%%%%
% Modificación de una plantilla de Latex para adaptarla al castellano.
%%%

%%%%%%%%%%%%%%%%%%%%%%%%%%%%%%%%%%%%%%%%%
% Thin Sectioned Essay
% LaTeX Template
% Version 1.0 (3/8/13)
%
% This template has been downloaded from:
% http://www.LaTeXTemplates.com
%
% Original Author:
% Nicolas Diaz (nsdiaz@uc.cl) with extensive modifications by:
% Vel (vel@latextemplates.com)
%
% License:
% CC BY-NC-SA 3.0 (http://creativecommons.org/licenses/by-nc-sa/3.0/)
%
%%%%%%%%%%%%%%%%%%%%%%%%%%%%%%%%%%%%%%%%%

%----------------------------------------------------------------------------------------
%	PACKAGES AND OTHER DOCUMENT CONFIGURATIONS
%----------------------------------------------------------------------------------------

\documentclass[a4paper, 11pt]{article} % Font size (can be 10pt, 11pt or 12pt) and paper size (remove a4paper for US letter paper)

\usepackage[protrusion=true,expansion=true]{microtype} % Better typography
\usepackage{graphicx} % Required for including pictures
\usepackage[usenames,dvipsnames]{color} % Coloring code
\usepackage{wrapfig} % Allows in-line images
\usepackage[utf8]{inputenc}

% Imágenes
\usepackage{graphicx} 

\usepackage{amsmath}
% para importar svg
%\usepackage[generate=all]{svgfig}

% sudo apt-get install texlive-lang-spanish
\usepackage[spanish]{babel} % English language/hyphenation
\selectlanguage{spanish}
% Hay que pelearse con babel-spanish para el alineamiento del punto decimal
\decimalpoint
\usepackage{dcolumn}
\newcolumntype{d}[1]{D{.}{\esperiod}{#1}}
\makeatletter
\addto\shorthandsspanish{\let\esperiod\es@period@code}
\makeatother

\usepackage{longtable}
\usepackage{tabu}
\usepackage{supertabular}

\usepackage{multicol}
\newsavebox\ltmcbox

% Para algoritmos
\usepackage{algorithm}
\usepackage{algorithmic}
\usepackage{amsthm}
\floatname{algorithm}{Algoritmo}
\renewcommand{\listalgorithmname}{Lista de algoritmos}
\renewcommand{\algorithmicrequire}{\textbf{Entrada:}}
\renewcommand{\algorithmicensure}{\textbf{Salida:}}
\renewcommand{\algorithmicend}{\textbf{fin}}
\renewcommand{\algorithmicif}{\textbf{si}}
\renewcommand{\algorithmicthen}{\textbf{entonces}}
\renewcommand{\algorithmicelse}{\textbf{en otro caso}}
\renewcommand{\algorithmicelsif}{\algorithmicelse,\ \algorithmicif}
\renewcommand{\algorithmicendif}{\algorithmicend\ \algorithmicif}
\renewcommand{\algorithmicfor}{\textbf{para }}
\renewcommand{\algorithmicforall}{\textbf{para cada}}
\renewcommand{\algorithmicdo}{\textbf{}}
\renewcommand{\algorithmicendfor}{\algorithmicend\ \algorithmicfor}
\renewcommand{\algorithmicwhile}{\textbf{mientras}}
\renewcommand{\algorithmicendwhile}{\algorithmicend\ \algorithmicwhile}
\renewcommand{\algorithmicloop}{\textbf{repetir}}
\renewcommand{\algorithmicendloop}{\algorithmicend\ \algorithmicloop}
\renewcommand{\algorithmicrepeat}{\textbf{repetir}}
\renewcommand{\algorithmicuntil}{\textbf{hasta que}}
\renewcommand{\algorithmicprint}{\textbf{imprimir}} 
\renewcommand{\algorithmicreturn}{\textbf{devolver}} 
\renewcommand{\algorithmictrue}{\textbf{true }} 
\renewcommand{\algorithmicfalse}{\textbf{false }} 
\renewcommand{\algorithmicand}{\textbf{y}}
\renewcommand{\algorithmicor}{\textbf{o}}


% Para matrices
\usepackage{amsmath}

% Símbolos matemáticos
\usepackage{amssymb}
\let\oldemptyset\emptyset
\let\emptyset\varnothing

\usepackage[section]{placeins} % Para gráficas en su sección.
\usepackage{mathpazo} % Use the Palatino font
\usepackage[T1]{fontenc} % Required for accented characters
\newenvironment{allintypewriter}{\ttfamily}{\par}
\setlength{\parindent}{0pt}
\parskip=8pt
\linespread{1.05} % Change line spacing here, Palatino benefits from a slight increase by default

\newcommand{\ef}[1]{$\mathcal{O}#1$}

\makeatletter
\renewcommand\@biblabel[1]{\textbf{#1.}} % Change the square brackets for each bibliography item from '[1]' to '1.'
\renewcommand{\@listI}{\itemsep=0pt} % Reduce the space between items in the itemize and enumerate environments and the bibliography
\newcommand{\imagen}[2]{\begin{center} \includegraphics[width=90mm]{#1} \\#2 \end{center}}

\renewcommand{\maketitle}{ % Customize the title - do not edit title and author name here, see the TITLE block below
\begin{flushright} % Right align
{\LARGE\@title} % Increase the font size of the title

\vspace{50pt} % Some vertical space between the title and author name

{\large\@author} % Author name
\\\@date % Date

\vspace{40pt} % Some vertical space between the author block and abstract
\end{flushright}
}

%----------------------------------------------------------------------------------------
%	TITLE
%----------------------------------------------------------------------------------------

\title{\textbf{Práctica 4}\\ % Title
QAP: Asignación cuadrática} % Subtitle

\author{\textsc{Óscar Bermúdez,\\Francisco David Charte,\\Ignacio Cordón,\\José Carlos Entrena,\\Mario Román} % Author
\\{\textit{Universidad de Granada}}} % Institution

\date{\today} % Date

%----------------------------------------------------------------------------------------

\begin{document}

\maketitle % Print the title section

\renewcommand{\abstractname}{Resumen} % Uncomment to change the name of the abstract to something else
\begin{abstract}
\end{abstract}
{\parskip=2pt
\tableofcontents
}
\pagebreak

  \section{El problema de la asignación cuadrática (QAP)}
    \subsection{Enunciado}
      Dado un conjunto de $n$ instalaciones con un flujo de transporte y
      un conste asociado entre dichas instalaciones, y dadas $n$ posibles localizaciones,
      el problema consiste en buscar ubicación a cada una de las instalaciones que
      minimice el coste del stransporte.
      
      Siendo $d(i,j)$ la distancia entre la localización $i$ y la $j$
      y siendo $w(i,j)$ el peso asociado al flujo de materiales entre la instalación
      $i$ y la $j$, se consiga minimizar:
      \begin{equation}
      \sum_{i=1}^n \sum_{j=1}^nw(i,j) d(\sigma(i),\sigma(j))
      \label{coste}
      \end{equation}

      donde $\sigma \in \mathcal{S}_n$ define una permutación sobre el conjunto de instalaciones.


    \subsection{Backtracking}
      En nuestro algoritmo de tipo Backtracking consideraremos todas las posibles opciones para quedarnos con la mejor de ellas.
      Tomaremos una instalación y la emplazaremos en un lugar cualquiera. A partir de aquí, colocaremos una nueva instalación
      en otro emplazamiento vacío hasta que se agoten. Llegados a este punto, volveremos hacia atrás y consideraremos soluciones
      alternativas, colocando las instalaciones en los demás emplazamientos no considerados hasta el momento. Una vez obtenida una
      posible solución, compararemos costes con la mejor solución hasta el momento y la actualizaremos si es mejorada.       
      
      Representaremos la solución como un vector de enteros, que nos indica en que emplazamiento hemos colocado cada ciudad a 
      través del índice. El tamaño del espacio de soluciones es de $n!$, que representa los emplazamientos posibles que tenemos 
      para cada ciudad.
      
      Dado un nodo intermedio $(i_0,i_1,\dots,i_k)$, los posibles hijos de la permutación parcial generada serán
      de la forma $(i_0,i_1,\dots,i_k, i_{k+1})$, cumpliendo que $i_{k+1} \notin \{i_0,i_1,\dots,i_k\}$. Nótese
      que la generación de la permutación es idéntica al caso del TSP.
    
    \subsection{Branch \& Bound}
      Para implementar la ramificación y poda necesitaremos definir una función que proporcione una
      cota inferior al coste de una permutación parcial. Como cota superior global del coste, usaremos el
      coste de la mejor solución que hayamos obtenido hasta el momento.
      
      La \textbf{estimación propuesta} en el guión se basa en colocar cada una de las fábricas que todavía no
      quedan fijas por la permutación en el punto que minimice su flujo de materiales con cada fábrica,
      permitiendo que cada fábrica pueda colocarse en un sitio distinto a cada paso.
      
      Calculamos como función de cota, para una permutación parcial en la que está definido $\sigma(i) \quad \forall i \leq k$,
      la siguiente suma:
      \begin{eqnarray*}
      \displaystyle
	  & \sum_{i,j \leq k}^n & d(\sigma(i),\sigma(j)) w(i,j) \\
	+ & \sum_{i \leq k \leq j \leq n}^n & \min_{c \notin \sigma(1..k)}{(d(\sigma(i),c))} w(i,j) \\
	+ & \sum_{j \leq k \leq i \leq n}^n & \min_{c \notin \sigma(1..k)}{(d(c,\sigma(j))} w(i,j) \\
	+ & \sum_{k \leq i,j \leq n}^n& \min_{c,e \notin \sigma(1..k)}{(d(c,e)} w(i,j) \\
      \end{eqnarray*}
      Donde podemos notar que el último término será siempre nulo, quedando:
      \begin{eqnarray*}
      \displaystyle
	  & \sum_{i,j \leq k}^n & d(\sigma(i),\sigma(j)) w(i,j) \\
	+ & \sum_{i \leq k \leq j \leq n}^n & \min_{c \notin \sigma(1..k)}{(d(\sigma(i),c))} w(i,j) \\
	+ & \sum_{j \leq k \leq i \leq n}^n & \min_{c \notin \sigma(1..k)}{(d(c,\sigma(j))} w(i,j) \\
      \end{eqnarray*}
      \medskip
      
      Como \textbf{propuesta} para mejorar esta estimación, podemos obligar a que una fábrica quede ubicada en un mismo sitio,
      y que no pueda cambiarse este a cada paso. En este caso, quedaría:
      \begin{eqnarray*}
      \displaystyle
	  & \sum_{i,j \leq k}^n & d(\sigma(i),\sigma(j)) w(i,j) \\
	+ & \sum_{k\leq j \leq n}^n & \min_{c_j \notin \sigma(1..k)}{\sum^n_{i\leq k} (d(\sigma(i),c_j))} w(i,j) \\
	+ & \sum_{k\leq i \leq n}^n & \min_{c_i \notin \sigma(1..k)}{\sum^n_{j\leq k} (d(c_i, \sigma(j)))} w(i,j) \\
      \end{eqnarray*}

      
      Y como última propuesta, considerar simplemente el coste obtenido hasta el momento también proporciona
      buenas acotaciones, en un compromiso entre el coste de calcular la cota y su bondad:
      \begin{eqnarray*}
      \displaystyle
	  & \sum_{i,j \leq k}^n & d(\sigma(i),\sigma(j)) w(i,j)
      \end{eqnarray*}
      
\end{document}