%%%%
% Modificación de una plantilla de Latex para adaptarla al castellano.
%%%

%%%%%%%%%%%%%%%%%%%%%%%%%%%%%%%%%%%%%%%%%
% Thin Sectioned Essay
% LaTeX Template
% Version 1.0 (3/8/13)
%
% This template has been downloaded from:
% http://www.LaTeXTemplates.com
%
% Original Author:
% Nicolas Diaz (nsdiaz@uc.cl) with extensive modifications by:
% Vel (vel@latextemplates.com)
%
% License:
% CC BY-NC-SA 3.0 (http://creativecommons.org/licenses/by-nc-sa/3.0/)
%
%%%%%%%%%%%%%%%%%%%%%%%%%%%%%%%%%%%%%%%%%

%----------------------------------------------------------------------------------------
%	PACKAGES AND OTHER DOCUMENT CONFIGURATIONS
%----------------------------------------------------------------------------------------

\documentclass[a4paper, 11pt]{article} % Font size (can be 10pt, 11pt or 12pt) and paper size (remove a4paper for US letter paper)

\usepackage[protrusion=true,expansion=true]{microtype} % Better typography
\usepackage{graphicx} % Required for including pictures
\usepackage[usenames,dvipsnames]{color} % Coloring code
\usepackage{wrapfig} % Allows in-line images
\usepackage[utf8]{inputenc}

% Imágenes
\usepackage{graphicx} 

\usepackage{amsmath}
% para importar svg
%\usepackage[generate=all]{svgfig}

% sudo apt-get install texlive-lang-spanish
\usepackage[spanish]{babel} % English language/hyphenation
\selectlanguage{spanish}
% Hay que pelearse con babel-spanish para el alineamiento del punto decimal
\decimalpoint
\usepackage{dcolumn}
\newcolumntype{d}[1]{D{.}{\esperiod}{#1}}
\makeatletter
\addto\shorthandsspanish{\let\esperiod\es@period@code}
\makeatother

\usepackage{longtable}
\usepackage{tabu}
\usepackage{supertabular}

\usepackage{multicol}
\newsavebox\ltmcbox

% Para algoritmos
\usepackage{algorithm}
\usepackage{algorithmic}
\usepackage{amsthm}
\floatname{algorithm}{Algoritmo}
\renewcommand{\listalgorithmname}{Lista de algoritmos}
\renewcommand{\algorithmicrequire}{\textbf{Entrada:}}
\renewcommand{\algorithmicensure}{\textbf{Salida:}}
\renewcommand{\algorithmicend}{\textbf{fin}}
\renewcommand{\algorithmicif}{\textbf{si}}
\renewcommand{\algorithmicthen}{\textbf{entonces}}
\renewcommand{\algorithmicelse}{\textbf{en otro caso}}
\renewcommand{\algorithmicelsif}{\algorithmicelse,\ \algorithmicif}
\renewcommand{\algorithmicendif}{\algorithmicend\ \algorithmicif}
\renewcommand{\algorithmicfor}{\textbf{para }}
\renewcommand{\algorithmicforall}{\textbf{para cada}}
\renewcommand{\algorithmicdo}{\textbf{}}
\renewcommand{\algorithmicendfor}{\algorithmicend\ \algorithmicfor}
\renewcommand{\algorithmicwhile}{\textbf{mientras}}
\renewcommand{\algorithmicendwhile}{\algorithmicend\ \algorithmicwhile}
\renewcommand{\algorithmicloop}{\textbf{repetir}}
\renewcommand{\algorithmicendloop}{\algorithmicend\ \algorithmicloop}
\renewcommand{\algorithmicrepeat}{\textbf{repetir}}
\renewcommand{\algorithmicuntil}{\textbf{hasta que}}
\renewcommand{\algorithmicprint}{\textbf{imprimir}} 
\renewcommand{\algorithmicreturn}{\textbf{devolver}} 
\renewcommand{\algorithmictrue}{\textbf{true }} 
\renewcommand{\algorithmicfalse}{\textbf{false }} 
\renewcommand{\algorithmicand}{\textbf{y}}
\renewcommand{\algorithmicor}{\textbf{o}}


% Para matrices
\usepackage{amsmath}

% Símbolos matemáticos
\usepackage{amssymb}
\let\oldemptyset\emptyset
\let\emptyset\varnothing

\usepackage[section]{placeins} % Para gráficas en su sección.
\usepackage{mathpazo} % Use the Palatino font
\usepackage[T1]{fontenc} % Required for accented characters
\newenvironment{allintypewriter}{\ttfamily}{\par}
\setlength{\parindent}{0pt}
\parskip=8pt
\linespread{1.05} % Change line spacing here, Palatino benefits from a slight increase by default

\makeatletter
\renewcommand\@biblabel[1]{\textbf{#1.}} % Change the square brackets for each bibliography item from '[1]' to '1.'
\renewcommand{\@listI}{\itemsep=0pt} % Reduce the space between items in the itemize and enumerate environments and the bibliography
\newcommand{\imagen}[2]{\begin{center} \includegraphics[width=90mm]{#1} \\#2 \end{center}}

\renewcommand{\maketitle}{ % Customize the title - do not edit title and author name here, see the TITLE block below
\begin{flushright} % Right align
{\LARGE\@title} % Increase the font size of the title

\vspace{50pt} % Some vertical space between the title and author name

{\large\@author} % Author name
\\\@date % Date

\vspace{40pt} % Some vertical space between the author block and abstract
\end{flushright}
}


%Basado en: http://en.wikibooks.org/wiki/LaTeX/Theorems
\usepackage{amsthm}
\newtheorem*{mydef}{Definición}
\newtheorem{mydefn}{Definición}
\newtheorem{theorem}{Teorema}
\everymath{\displaystyle} % Displaystyle por defecto

%----------------------------------------------------------------------------------------
%	TITLE
%----------------------------------------------------------------------------------------

\title{\textbf{Práctica 4}\\ % Title
Backtracking y Branch \& Bound} % Subtitle

\author{\textsc{Óscar Bermúdez,\\Francisco David Charte,\\Ignacio Cordón,\\José Carlos Entrena,\\Mario Román} % Author
\\{\textit{Universidad de Granada}}} % Institution

\date{\today} % Date

%----------------------------------------------------------------------------------------

\begin{document}

\maketitle % Print the title section

\renewcommand{\abstractname}{Resumen} % Uncomment to change the name of the abstract to something else
\begin{abstract}
\end{abstract}
{\parskip=2pt
\tableofcontents
}
\pagebreak


\section{El problema de la mochila}


\subsection{Algoritmo}

Definimos una mochila como un par \texttt{(vector<bool>,beneficios)} donde el vector
representa si un determinado objeto de una lista está o no en la solución.

Una mochila vacía será un par \texttt{([],0)}
\begin{algorithm}[H]
	\begin{algorithmic}[1]
		\REQUIRE \ \\
        	\texttt{limite}, de peso de la mochila \\
        	\texttt{pesos}, vector de pesos de objetos\\
        	\texttt{beneficios}, vector de beneficios de objetos\\\
     	\STATE{\texttt{tamanio = \# pesos}}
	\STATE{\texttt{solucion = empty Mochila}}
	\STATE{\texttt{max\_valor=0}}
     	\STATE{Creamos una cola \texttt{posibles\_mochilas}}
     	\STATE{\texttt{posibles\_mochilas.push(empty Mochila)}}
     	
     	\WHILE{\texttt{posibles\_mochilas}$\neq \emptyset$}
	  \STATE{\texttt{actual = posibles\_mochilas.pop()}}
	  \IF{\texttt{\#actual[0] = tamanio}}
	    \IF{\texttt{\#actual[1] > max\_valor}}
	      \STATE{\texttt{max\_valor = actual[1]}}
	      \STATE{\texttt{solucion = actual}}
	    \ENDIF
	  \ELSE
	    \STATE{\texttt{con\_nuevo = sin\_nuevo = actual}}
	    \STATE{\texttt{con\_nuevo[1] += beneficios[\#actual[0]]}}
	    \STATE{\texttt{con\_nuevo.push\_back(true)}}
	    \STATE{\texttt{nuevo\_peso = 'peso de' \texttt{con\_nuevo}}}
	    \IF{\texttt{nuevo\_peso <= limite}}
	      \STATE{\texttt{posibles\_mochilas.push(con\_nuevo)}}
	    \ENDIF
	    \STATE{\texttt{posibles\_mochilas.push(sin\_nuevo}}
	  \ENDIF
     	\ENDWHILE
     	
	\end{algorithmic}
    \caption{Algoritmo backtracking para el problema de la mochila}
    \label{monedas}
\end{algorithm}  


\section{Traveling Salesman Problem}
  \subsection{Enunciado}
    Dada una lista $S$ de $n$ ciudades como puntos en el plano:
    \begin{equation}
      S = [(x_0,y_0), (x_1,y_1), \dots (x_{n-1},y_{n-1})] \subset \mathbb{R}^2
    \end{equation}
    Y definiendo la longitud de recorrer una lista como la suma de las distancias de cada ciudad a la siguiente:
    \begin{equation}
     long(S) = \sum_{i \in \mathbb{Z}_n} dist((x_i,y_i), (x_{i+1}, y_{i+1})) = \sum_{i \in \mathbb{Z}_n} \sqrt{(x_i-x_{i+1})^2 + (y_i-y_{i+1})^2}
    \end{equation}
    Encontrar la permutación de la lista $\sigma : \mathbb{Z}_n \leftrightarrow \mathbb{Z}_n$, tal que su longitud sea mínima:
    \begin{equation}
     long(\sigma(S)) = long([(x_{\sigma(1)},y_{\sigma(1)}), (x_{\sigma(2)},y_{\sigma(2)}), \dots, (x_{\sigma(n)},y_{\sigma(n)})])
    \end{equation}


\section{3-Dimensional Matching}


\section{Implementaciones}

\subsection{El problema de la mochila}
    
\end{document}