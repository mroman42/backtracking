%%%%
% Modificación de una plantilla de Latex para adaptarla al castellano.
%%%

%%%%%%%%%%%%%%%%%%%%%%%%%%%%%%%%%%%%%%%%%
% Thin Sectioned Essay
% LaTeX Template
% Version 1.0 (3/8/13)
%
% This template has been downloaded from:
% http://www.LaTeXTemplates.com
%
% Original Author:
% Nicolas Diaz (nsdiaz@uc.cl) with extensive modifications by:
% Vel (vel@latextemplates.com)
%
% License:
% CC BY-NC-SA 3.0 (http://creativecommons.org/licenses/by-nc-sa/3.0/)
%
%%%%%%%%%%%%%%%%%%%%%%%%%%%%%%%%%%%%%%%%%

%----------------------------------------------------------------------------------------
%	PACKAGES AND OTHER DOCUMENT CONFIGURATIONS
%----------------------------------------------------------------------------------------

\documentclass[a4paper, 11pt]{article} % Font size (can be 10pt, 11pt or 12pt) and paper size (remove a4paper for US letter paper)

\usepackage[protrusion=true,expansion=true]{microtype} % Better typography
\usepackage{graphicx} % Required for including pictures
\usepackage[usenames,dvipsnames]{color} % Coloring code
\usepackage{wrapfig} % Allows in-line images
\usepackage[utf8]{inputenc}

% Imágenes
\usepackage{graphicx} 

\usepackage{amsmath}
% para importar svg
%\usepackage[generate=all]{svgfig}

% sudo apt-get install texlive-lang-spanish
\usepackage[spanish]{babel} % English language/hyphenation
\selectlanguage{spanish}
% Hay que pelearse con babel-spanish para el alineamiento del punto decimal
\decimalpoint
\usepackage{dcolumn}
\newcolumntype{d}[1]{D{.}{\esperiod}{#1}}
\makeatletter
\addto\shorthandsspanish{\let\esperiod\es@period@code}
\makeatother

\usepackage{longtable}
\usepackage{tabu}
\usepackage{supertabular}

\usepackage{multicol}
\newsavebox\ltmcbox

% Para algoritmos
\usepackage{algorithm}
\usepackage{algorithmic}
\usepackage{amsthm}
\floatname{algorithm}{Algoritmo}
\renewcommand{\listalgorithmname}{Lista de algoritmos}
\renewcommand{\algorithmicrequire}{\textbf{Entrada:}}
\renewcommand{\algorithmicensure}{\textbf{Salida:}}
\renewcommand{\algorithmicend}{\textbf{fin}}
\renewcommand{\algorithmicif}{\textbf{si}}
\renewcommand{\algorithmicthen}{\textbf{entonces}}
\renewcommand{\algorithmicelse}{\textbf{en otro caso}}
\renewcommand{\algorithmicelsif}{\algorithmicelse,\ \algorithmicif}
\renewcommand{\algorithmicendif}{\algorithmicend\ \algorithmicif}
\renewcommand{\algorithmicfor}{\textbf{para }}
\renewcommand{\algorithmicforall}{\textbf{para cada}}
\renewcommand{\algorithmicdo}{\textbf{}}
\renewcommand{\algorithmicendfor}{\algorithmicend\ \algorithmicfor}
\renewcommand{\algorithmicwhile}{\textbf{mientras}}
\renewcommand{\algorithmicendwhile}{\algorithmicend\ \algorithmicwhile}
\renewcommand{\algorithmicloop}{\textbf{repetir}}
\renewcommand{\algorithmicendloop}{\algorithmicend\ \algorithmicloop}
\renewcommand{\algorithmicrepeat}{\textbf{repetir}}
\renewcommand{\algorithmicuntil}{\textbf{hasta que}}
\renewcommand{\algorithmicprint}{\textbf{imprimir}} 
\renewcommand{\algorithmicreturn}{\textbf{devolver}} 
\renewcommand{\algorithmictrue}{\textbf{true }} 
\renewcommand{\algorithmicfalse}{\textbf{false }} 
\renewcommand{\algorithmicand}{\textbf{y}}
\renewcommand{\algorithmicor}{\textbf{o}}


% Para matrices
\usepackage{amsmath}

% Símbolos matemáticos
\usepackage{amssymb}
\let\oldemptyset\emptyset
\let\emptyset\varnothing

\usepackage[section]{placeins} % Para gráficas en su sección.
\usepackage{mathpazo} % Use the Palatino font
\usepackage[T1]{fontenc} % Required for accented characters
\newenvironment{allintypewriter}{\ttfamily}{\par}
\setlength{\parindent}{0pt}
\parskip=8pt
\linespread{1.05} % Change line spacing here, Palatino benefits from a slight increase by default

\makeatletter
\renewcommand\@biblabel[1]{\textbf{#1.}} % Change the square brackets for each bibliography item from '[1]' to '1.'
\renewcommand{\@listI}{\itemsep=0pt} % Reduce the space between items in the itemize and enumerate environments and the bibliography
\newcommand{\imagen}[2]{\begin{center} \includegraphics[width=90mm]{#1} \\#2 \end{center}}

\renewcommand{\maketitle}{ % Customize the title - do not edit title and author name here, see the TITLE block below
\begin{flushright} % Right align
{\LARGE\@title} % Increase the font size of the title

\vspace{50pt} % Some vertical space between the title and author name

{\large\@author} % Author name
\\\@date % Date

\vspace{40pt} % Some vertical space between the author block and abstract
\end{flushright}
}


%Basado en: http://en.wikibooks.org/wiki/LaTeX/Theorems
\usepackage{amsthm}
\newtheorem*{mydef}{Definición}
\newtheorem{mydefn}{Definición}
\newtheorem{theorem}{Teorema}
\everymath{\displaystyle} % Displaystyle por defecto

%----------------------------------------------------------------------------------------
%	TITLE
%----------------------------------------------------------------------------------------

\title{\textbf{Práctica 4}\\ % Title
Backtracking y Branch \& Bound} % Subtitle

\author{\textsc{Óscar Bermúdez,\\Francisco David Charte,\\Ignacio Cordón,\\José Carlos Entrena,\\Mario Román} % Author
\\{\textit{Universidad de Granada}}} % Institution

\date{\today} % Date

%----------------------------------------------------------------------------------------

\begin{document}

\maketitle % Print the title section

\renewcommand{\abstractname}{Resumen} % Uncomment to change the name of the abstract to something else
\begin{abstract}
\end{abstract}
{\parskip=2pt
\tableofcontents
}
\pagebreak


\section{El problema de la mochila}

\subsection{Enunciado}
El problema de la mochila es un problema de maximización, en el que tenemos un contenedor con un tamaño 
fijo $M$ y un conjunto de $n$ elementos que queremos almacenar en nuestro contenedor. Dichos elementos tendrán un peso $\omega_i$
determinado, y obtendremos un cierto beneficio $b_i$ por cada objeto almacenado.

Nuestro objetivo es maximizar la suma de beneficios, $\sum_{i=1}^{n} b_i\delta_i$, (siendo $\delta_i = 1$ si el i-ésimo objeto está en la mochila, 0 en caso contrario) con la restricción de que la mochila solo puede llevar un peso máximo, determinado por su tamaño: $\sum_{i=1}^{n} \omega_i\delta_i \leq M$.

  
    \subsection{Backtracking}
Representamos una mochila como un vector de datos booleanos, $\delta$, de tamaño $n$. Esto implica que nuestro espacio de soluciones tendrá un tamaño de $2^n$.

La función de poda toma un vector $\delta = (\delta_1, \dots, \delta_n)$ y devuelve falso en caso de que $\sum_{i=1}^{n} \omega_i\delta_i > M$, asegurando así que solo se van considerando en el proceso candidatas a solución.


	\subsubsection{Algoritmo}
Nuestro algoritmo es de tipo iterativo, por lo que usa una cola en la que almacenamos las mochilas que vamos procesando. La poda se produce cuando al añadir un objeto, la mochila resultante supera el peso límite, en cuyo caso esta no es añadida a la cola de candidatas. Cuando encontremos una mochila completa, se compara con la solución actual, y se sustituye en caso de mejorarla. 

\begin{algorithm}[H]
	\begin{algorithmic}[1]
		\REQUIRE \ \\
        	$M$, límite de peso de la mochila \\
        	$\omega$, vector de pesos de objetos\\
        	$B$, vector de beneficios de objetos\\\
     	\STATE{\texttt{$n$ = \#pesos}}
	\STATE{\texttt{solución = [false,\dots,false]}}\\\
     	\STATE{Creamos una cola \texttt{posibles\_mochilas}}
     	\STATE{\texttt{posibles\_mochilas.push([])}}
     	
     	\WHILE{\texttt{posibles\_mochilas}$\neq \emptyset$}
	  \STATE{\texttt{actual = posibles\_mochilas.pop()}}
	  \IF{\texttt{\#actual == $n$}}
	    \IF{beneficio(\texttt{actual}) >\ beneficio(\texttt{solución})}
	      \STATE{\texttt{solución = actual}}
	    \ENDIF
	  \ELSE
	    \STATE{\texttt{con\_nuevo = sin\_nuevo = actual}
	    \STATE{\texttt{con\_nuevo.push\_back(true)}}
	    \STATE{\texttt{sin\_nuevo.push\_back(false)}}}
	    \STATE{\texttt{nuevo\_peso = }peso(\texttt{con\_nuevo})}
	    \IF{\texttt{nuevo\_peso} $\leq M$}
	    \STATE{(\textit{Acotación: no analizamos el caso si \texttt{nuevo\_peso} > $M$})}
	      \STATE{\texttt{posibles\_mochilas.push(con\_nuevo)}}
	    \ENDIF
	    \STATE{\texttt{posibles\_mochilas.push(sin\_nuevo)}}
	  \ENDIF
     	\ENDWHILE
     \RETURN{\texttt{solución}}
	\end{algorithmic}
    \caption{Algoritmo backtracking para el problema de la mochila}
    \label{Back-Mochila}
\end{algorithm}

En el algoritmo, \textit{beneficio} representa el valor de la suma de beneficios de un vector de booleanos $\delta$, $\sum_{i=1}^{n}b_i\delta_i$; y \textit{peso} representa la suma de pesos del vector $\delta$, $\sum_{i=1}^{n}\omega_i\delta_i$.



  \subsection{Branch \& Bound}
  
  Para la versión Branch \& Bound del algoritmo, usamos la misma representación para la solución, un vector de booleanos, y por tanto tenemos el mismo espacio de soluciones.
  
  En este caso, un nodo intermedio será del tipo $(\delta_1, \dots, \delta_k)$, y sus hijos serán $(\delta_1, \dots, \delta_k, 1)$ y $(\delta_1, \dots, \delta_k, 0)$.
  
	\subsubsection{Algoritmo}


\begin{algorithm}[H]
	\begin{algorithmic}[1]
		\REQUIRE \ \\
        	$M$, límite de peso de la mochila \\
         	$\omega$, vector de pesos de objetos\\
          	$B$, vector de beneficios de objetos\\\
     	\STATE{\texttt{$n$ = \#pesos}}
     		\STATE{\texttt{solución = [false,\dots,false]}}\\\
     	     	\STATE{Creamos una cola \texttt{posibles\_mochilas}}
     	     	\STATE{\texttt{posibles\_mochilas.push([])}}
     	     	
     	     	\WHILE{\texttt{posibles\_mochilas}$\neq \emptyset$}
     		  \STATE{\texttt{actual = posibles\_mochilas.pop()}}
     		  \IF{\texttt{\#actual == $n$}}
     		    \IF{beneficio(\texttt{actual}) >\ beneficio(\texttt{solución})}
     		      \STATE{\texttt{solución = actual}}
     		    \ENDIF
     		  \ELSE
     		    \STATE{\texttt{con\_nuevo = sin\_nuevo = actual}
     		    \STATE{\texttt{con\_nuevo.push\_back(true)}}
     		    \STATE{\texttt{sin\_nuevo.push\_back(false)}}}
     		    \STATE{\texttt{nuevo\_peso = }peso(\texttt{con\_nuevo})}
     		    \IF{\texttt{nuevo\_peso} $\leq M$}
     		    \STATE{(\textit{Acotación: no analizamos el caso si \texttt{nuevo\_peso} > $M$})}
     		      \STATE{\texttt{posibles\_mochilas.push(con\_nuevo)}}
	    \ENDIF
	    \IF{beneficio(\texttt{solución}) <\ beneficio(\texttt{sin\_nuevo}) + calculaCota(\texttt{sin\_nuevo})}
	      \STATE{\texttt{posibles\_mochilas.push(sin\_nuevo)}}
	    \ENDIF
	  \ENDIF
     	\ENDWHILE
     	\RETURN{\texttt{solución}}
     	
	\end{algorithmic}
    \caption{Algoritmo Branch \& Bound para el problema de la mochila}
    \label{BBound-Mochila}
\end{algorithm}


\section{Traveling Salesman Problem}
  \subsection{Enunciado}
    Dada una lista $S$ de $n$ ciudades, representadas como puntos en el plano:
    \begin{equation}
      S = [(x_0,y_0), (x_1,y_1), \dots (x_{n-1},y_{n-1})] \subset \mathbb{R}^2
    \end{equation}
    Y definiendo la longitud de recorrer una lista como la suma de las distancias de cada ciudad a la siguiente:
    \begin{equation}
     long(S) = \sum_{i \in \mathbb{Z}_n} dist((x_i,y_i), (x_{i+1}, y_{i+1})) = \sum_{i \in \mathbb{Z}_n} \sqrt{(x_i-x_{i+1})^2 + (y_i-y_{i+1})^2}
    \end{equation}
    El objetivo es encontrar la permutación de la lista $\sigma : \mathbb{Z}_n \leftrightarrow \mathbb{Z}_n$, verificando que su longitud sea mínima:
    \begin{equation}
     long(\sigma(S)) = long([(x_{\sigma(1)},y_{\sigma(1)}), (x_{\sigma(2)},y_{\sigma(2)}), \dots, (x_{\sigma(n)},y_{\sigma(n)})])
    \end{equation}
    
  \subsection{Backtracking}
  
  
  Utilizaremos un vector de enteros para representar una solución, donde almacenaremos los índices de las ciudades que se van recorriendo en la ruta. Como únicamente podemos pasar una vez por cada ciudad, el espacio de soluciones tendrá un tamaño de $n!$
  
  
    \subsubsection{Algoritmo}
      Para aplicar backtracking al problema del TSP, utilizamos un algoritmo recursivo, el cual recibe la lista de 
      ciudades del problema, la ruta con las ciudades que llevamos hasta el momento, su coste, y el índice de la 
      última ciudad añadida. Calcularemos todas las ramificaciones posibles de la ruta actual, y tomaremos la mejor 
      de estas, comparando el coste de cada rama con el coste de la mejor solución hasta el momento.
      
  \begin{algorithm}[H]
  	\begin{algorithmic}[1]
  		\REQUIRE \ \\
          \texttt{ciudades}, lista con todas las ciudades \\
          \texttt{ruta}, ruta con algunas ciudades \\
          \texttt{coste\_actual}, coste de la ruta pasada \\
          \texttt{índice}, índice de la ciudad a partir de la cual tiene que permutar. \\\

		\STATE{\texttt{mejor\_coste = $\infty$}}

        \IF{\texttt{indice == \#ciudades}}
          \STATE{\texttt{coste\_actual += distancia(ruta[$n$-1], ruta[0])}}

	      \IF{\texttt{coste\_actual <\ mejor\_coste}}
	        \STATE{\texttt{mejor\_coste = coste\_actual\\
	        mejor\_ruta = ruta}}
	      \ENDIF
	    \ELSE
	      \FOR{\texttt{i} = índice hasta \texttt{\#ciudades}:}
	        \STATE{Producimos una permutación:\\
	        \texttt{intercambia(ruta[i], ruta[indice])\\
	        coste\_actual += distancia(ruta[indice-1], ruta[indice])}}
	        \STATE{Llamamos recursivamente a la función:\\
	        \texttt{TSP(ciudades, ruta, coste\_actual, indice+1)}}
	        \STATE{Deshacemos el cambio:\\ 
	        \texttt{coste\_actual --= distancia(ruta[indice - 1], ruta[indice])\\
	        intercambiar(ruta[indice], ruta[i])}}
	      \ENDFOR
	    \ENDIF
	    \RETURN{\texttt{ruta}}
  	\end{algorithmic}
      \caption{Algoritmo Branch \& Bound para el TSP}
      \label{Back-TSP}
  \end{algorithm}
  
 
  \subsection{Branch \& Bound}
  \subsubsection{Algoritmo}
  
  Para la ramificación y poda de este problema, utilizaremos una técnica basada en 2-OPT para desechar permutaciones: Si al añadir una nueva ciudad, la ruta resultante contiene algún cruce de caminos, es mejorable por 2-OPT y por tanto no es óptima, luego podemos pasar al siguiente candidato sin tener en cuenta las soluciones derivadas de la que hemos calculado.  
  
    \begin{algorithm}[H]
    	\begin{algorithmic}[1]
		\REQUIRE \ \\
          \texttt{ciudades}, lista con todas las ciudades \\
          \texttt{ruta}, ruta con algunas ciudades \\
          \texttt{coste\_actual}, coste de la ruta pasada \\
          \texttt{índice}, índice de la ciudad a partir de la cual tiene que permutar. \\\

		\STATE{\texttt{mejor\_coste = $\infty$}}

        \IF{\texttt{indice == \#ciudades}}
          \STATE{\texttt{coste\_actual += distancia(ruta[$n$-1], ruta[0])}}

	      \IF{\texttt{coste\_actual < mejor\_coste}}
	        \STATE{\texttt{mejor\_coste = coste\_actual\\
	        mejor\_ruta = ruta}}
	      \ENDIF
	    \ELSIF{\texttt{coste\_actual > mejor\_coste}}
	      \RETURN $\emptyset$
	    \ELSE
	      \FOR{\texttt{i} = indice hasta \texttt{\#ciudades}:}
	        \STATE{\texttt{Buscamos un cruce de caminos. En cuyo caso, lo desechamos porque no puede ser el óptimo.}}
	        \STATE{\texttt{Producimos una permutación:\\
	        intercambiar(ruta[i], ruta[indice])\\
	        coste\_actual += distancia(ruta[indice-1], ruta[indice])}}
	        \STATE{\texttt{TSP(ciudades, ruta, coste\_actual, indice+1)}}
	        \STATE{\texttt{Deshacemos el cambio:\\ coste\_actual --= distancia(ruta[indice - 1], ruta[indice])\\
	        intercambiar(ruta[indice], ruta[i])}}
	        
	        
	      \ENDFOR
	    \ENDIF
	    \RETURN{\texttt{ruta}}
    	\end{algorithmic}
        \caption{Algoritmo Branch \& Bound para el TSP}
        \label{BBound-TSP}
    \end{algorithm}

\section{Planificación en multiprocesadores}
  \subsection{Enunciado}
  Tenemos un conjunto de $n$ tareas con un tiempo de ejecución $t_i$ asociado a cada una, existiendo una relación de precedencia
  entre algunas tareas, que hace que tengan que ejecutarse unas antes que otras. Nuestro objetivo es obtener la asignación de 
  tareas que, manteniendo las relaciones de dependencia, resulte en un tiempo de ejecución mínimo. 
  
  
  \subsection{Backtracking}
  
  \subsection{Branch \& Bound}

\section{3-Dimensional Matching}
  \subsection{Enunciado}
    Sean $X$,$Y$,$Z$ tres conjuntos finitos y sea $T \subset X \times Y \times Z$, subconjunto de tripletas válidas.
    Una asignación válida $M$ es un subconjunto de elementos disjuntos de $T$. Es decir, $M \subset T$ tal que:
    \begin{equation}
     \forall (x_1,y_1,z_1), (x_2,y_2,z_2) \in M : \quad (x_1 \neq x_2) \vee (y_1 \neq y_2) \vee (z_1 \neq z_2)
    \end{equation} 
    Buscamos la asignación válida de mayor cardinal, es decir la $M$ asignación válida maximizando $|M|$.
    
  \subsection{Backtracking}
  En este algoritmo consideraremos todas las asignaciones posibles y nos quedaremos con la asignación que tenga mayor cardinal. Para ello, consideraremos todas las aristas e iremos comprobando para cada una si es posible añadirla. De ser así, distinguiremos dos casos: La arista puede pertenecer o no a la solución. Sobre estos dos casos aplicaremos el mismo algoritmo. Cuando no quede ninguna arista que pueda ser añadida, compararemos los tamaños de la solución obtenida y la mejor solución hasta el momento, actualizando la mejor solución si la actual la mejorase. 
  
  Como cada arista tiene la opción de estar o no estar en la solución, la dimensión del espacio de soluciones es de $2^n$.   
	\subsubsection{Algoritmo}
      \begin{algorithm}[H]
      	\begin{algorithmic}[1]
  		\REQUIRE \ \\
        	\texttt{aristas} \\
     	\STATE{\texttt{$n$ = \#aristas}}
     	\STATE{\texttt{solución = [[false,\dots,false],[false,\dots,false],[false,\dots,false]]}}\\\
     	\STATE{Creamos una cola \texttt{posibles\_particiones}}
     	\STATE{\texttt{posibles\_particiones.push([])}}
     	
     	\WHILE{\texttt{posibles\_particiones}$\neq \emptyset$}
		  \STATE{\texttt{actual = posibles\_particiones.pop()}}
		  \STATE{\texttt{indice = \#actual.aristas}}
		  \IF{\texttt{indice == $n$}}
		    \IF{valor(\texttt{actual}) >\ valor(\texttt{solución})}
		      \STATE{\texttt{solución = actual}}
		    \ENDIF
		  \ELSE
	        \STATE{\texttt{con\_nueva = sin\_nueva = actual}
	        \STATE{\texttt{con\_nueva.push\_back(true)}}
	        \STATE{\texttt{sin\_nueva.push\_back(false)}}}
	        
	        \STATE{\texttt{posibles\_mochilas.push(sin\_nueva)}}
	        \IF{Se puede añadir}
	        \STATE{\texttt{posibles\_mochilas.push(con\_nueva)}}
	        \ENDIF
	      \ENDIF
 		\ENDWHILE
     \RETURN{\texttt{solución}}
      	\end{algorithmic}
          \caption{Algoritmo Backtracking para el 3D Matching}
          \label{Back-3DMatch}
      \end{algorithm}

  \subsection{Branch \& Bound}
    \begin{algorithm}[H]
    	\begin{algorithmic}[1]
		\REQUIRE 
    	\end{algorithmic}
        \caption{Algoritmo Branch \& Bound para el 3D Matching}
        \label{BBound-3DMatch}
    \end{algorithm}
  

\section{El problema de la asignación cuadrática (QAP)}
  \subsection{Enunciado}
    Dado un conjunto de $n$ instalaciones con un flujo de transporte y un coste asociado entre ellas, y dadas $n$ posibles 
    localizaciones, el problema consiste en buscar la localización más adecuada para cada instalación, de forma que el 
    coste total del transporte entre instalaciones sea el mínimo posible. 

  \subsection{Backtracking}
  \subsection{Branch \& Bound}


\section{Implementaciones}
  \subsection{El problema de la mochila}
        Versión en C++:
        
        \small
  	\texttt{% Generator: GNU source-highlight, by Lorenzo Bettini, http://www.gnu.org/software/src-highlite
\noindent
\mbox{}\textit{\textcolor{Brown}{/**}} \\
\mbox{}\textit{\textcolor{Brown}{\ *\ mochila.cpp}} \\
\mbox{}\textit{\textcolor{Brown}{\ *\ Problema\ de\ la\ mochila.}} \\
\mbox{}\textit{\textcolor{Brown}{\ *\ Implementación\ de\ un\ algoritmo\ de\ backtracking\ en\ C++.}} \\
\mbox{}\textit{\textcolor{Brown}{\ *}} \\
\mbox{}\textit{\textcolor{Brown}{\ *\ Formato\ del\ problema:\ }} \\
\mbox{}\textit{\textcolor{Brown}{\ *\ \ [tamaño\ de\ mochila]}} \\
\mbox{}\textit{\textcolor{Brown}{\ *\ \ [número\ de\ objetos]}} \\
\mbox{}\textit{\textcolor{Brown}{\ *\ \ [vector\ de\ pesos]}} \\
\mbox{}\textit{\textcolor{Brown}{\ *\ \ [vector\ de\ beneficios]}} \\
\mbox{}\textit{\textcolor{Brown}{\ */}} \\
\mbox{} \\
\mbox{}\textit{\textcolor{Brown}{/**}} \\
\mbox{}\textit{\textcolor{Brown}{\ Representación\ de\ la\ solución:}} \\
\mbox{}\textit{\textcolor{Brown}{\ La\ solución\ será\ un\ objeto\ de\ tipo\ Mochila,\ es\ decir,\ un\ par\ (vector}}\textbf{\textcolor{Blue}{\textless{}bool\textgreater{}}}\textit{\textcolor{Brown}{,\ int)}} \\
\mbox{}\textit{\textcolor{Brown}{\ donde\ el\ vector\ contendrá\ true\ o\ false\ según\ si\ el\ elemento\ indizado\ correspondiente}} \\
\mbox{}\textit{\textcolor{Brown}{\ está\ contenido\ en\ la\ mochila\ solución;\ y\ el\ entero\ representa\ el\ beneficio\ asociado}} \\
\mbox{}\textit{\textcolor{Brown}{\ a\ la\ mochila.}} \\
\mbox{} \\
\mbox{}\textit{\textcolor{Brown}{\ Tamaño\ del\ espacio\ de\ soluciones:}} \\
\mbox{}\textit{\textcolor{Brown}{\ ????}} \\
\mbox{} \\
\mbox{}\textit{\textcolor{Brown}{\ Backtracking\ -\ \ Función\ de\ poda/acotación:}} \\
\mbox{}\textit{\textcolor{Brown}{\ peso$\_$actual\ \textgreater{}\ peso$\_$maximo}} \\
\mbox{} \\
\mbox{}\textit{\textcolor{Brown}{\ Branch\ \&\ Bound\ -\ Hijos\ posibles\ de\ un\ nodo\ intermedio}} \\
\mbox{}\textit{\textcolor{Brown}{\ Para\ una\ mochila\ sin\ llenar,\ la\ siguiente\ mochila\ puede\ o\ no\ contener\ }} \\
\mbox{}\textit{\textcolor{Brown}{\ el\ siguiente\ elemento,\ esto\ nos\ da\ dos\ hijos\ (en\ el\ código,\ `con$\_$nuevo`\ y}} \\
\mbox{}\textit{\textcolor{Brown}{\ `sin$\_$nuevo`)}} \\
\mbox{} \\
\mbox{}\textit{\textcolor{Brown}{\ Branch\ \&\ Bound\ -\ Cálculo\ de\ las\ cotas}} \\
\mbox{}\textit{\textcolor{Brown}{\ Calculamos\ una\ cota\ superior\ para\ el\ beneficio\ que\ puede\ reportar\ la}} \\
\mbox{}\textit{\textcolor{Brown}{\ mochila\ actual.}} \\
\mbox{}\textit{\textcolor{Brown}{\ }} \\
\mbox{}\textit{\textcolor{Brown}{\ Branch\ \&\ Bound\ -\ Estrategia\ de\ poda}} \\
\mbox{}\textit{\textcolor{Brown}{\ Solo\ introducimos\ la\ mochila\ entre\ las\ posibles\ (pendientes\ de\ completar)}} \\
\mbox{}\textit{\textcolor{Brown}{\ si\ la\ cota\ superior\ del\ beneficio\ es\ mayor\ que\ el\ mayor\ beneficio\ obtenido}} \\
\mbox{}\textit{\textcolor{Brown}{\ hasta\ el\ momento.}} \\
\mbox{}\textit{\textcolor{Brown}{\ }} \\
\mbox{}\textit{\textcolor{Brown}{\ Branch\ \&\ Bound\ -\ Estrategia\ de\ ramificación}} \\
\mbox{}\textit{\textcolor{Brown}{\ WTF?}} \\
\mbox{} \\
\mbox{}\textit{\textcolor{Brown}{\ */}} \\
\mbox{} \\
\mbox{}\textbf{\textcolor{RoyalBlue}{\#include}}\ \texttt{\textcolor{Red}{\textless{}iostream\textgreater{}}} \\
\mbox{}\textbf{\textcolor{RoyalBlue}{\#include}}\ \texttt{\textcolor{Red}{\textless{}vector\textgreater{}}} \\
\mbox{}\textbf{\textcolor{RoyalBlue}{\#include}}\ \texttt{\textcolor{Red}{\textless{}queue\textgreater{}}} \\
\mbox{}\textbf{\textcolor{RoyalBlue}{\#include}}\ \texttt{\textcolor{Red}{\textless{}algorithm\textgreater{}}} \\
\mbox{}\textbf{\textcolor{RoyalBlue}{\#include}}\ \texttt{\textcolor{Red}{\textless{}chrono\textgreater{}}} \\
\mbox{}\textbf{\textcolor{Blue}{using}}\ \textbf{\textcolor{Blue}{namespace}}\ std\textcolor{BrickRed}{;} \\
\mbox{} \\
\mbox{}\textbf{\textcolor{Blue}{typedef}}\ \textcolor{ForestGreen}{unsigned}\ \textcolor{ForestGreen}{int}\ uint\textcolor{BrickRed}{;} \\
\mbox{} \\
\mbox{}\textbf{\textcolor{Blue}{typedef}}\ \textcolor{TealBlue}{pair\textless{}vector\textless{}bool\textgreater{},int\textgreater{}}\ Mochila\textcolor{BrickRed}{;} \\
\mbox{} \\
\mbox{}\textbf{\textcolor{Blue}{template}}\textcolor{BrickRed}{\textless{}}\textbf{\textcolor{Blue}{class}}\ \textcolor{TealBlue}{T}\textcolor{BrickRed}{\textgreater{}} \\
\mbox{}ostream\textcolor{BrickRed}{\&}\ \textbf{\textcolor{Blue}{operator}}\textcolor{BrickRed}{\textless{}\textless{}}\ \textcolor{BrickRed}{(}ostream\textcolor{BrickRed}{\&}\ output\textcolor{BrickRed}{,}\ vector\textcolor{BrickRed}{\textless{}}T\textcolor{BrickRed}{\textgreater{}\&}\ v\textcolor{BrickRed}{)}\textcolor{Red}{\{} \\
\mbox{}\ \ \ \ \textbf{\textcolor{Blue}{for}}\ \textcolor{BrickRed}{(}\textbf{\textcolor{Blue}{auto}}\ i\ \textcolor{BrickRed}{:}\ v\textcolor{BrickRed}{)} \\
\mbox{}\ \ \ \ \ \ \ \ output\ \textcolor{BrickRed}{\textless{}\textless{}}\ i\ \textcolor{BrickRed}{\textless{}\textless{}}\ \texttt{\textcolor{Red}{'\ '}}\textcolor{BrickRed}{;} \\
\mbox{}\ \ \ \  \\
\mbox{}\ \ \ \ output\ \textcolor{BrickRed}{\textless{}\textless{}}\ endl\textcolor{BrickRed}{;} \\
\mbox{}\ \ \ \ \textbf{\textcolor{Blue}{return}}\ output\textcolor{BrickRed}{;} \\
\mbox{}\textcolor{Red}{\}} \\
\mbox{} \\
\mbox{}\textbf{\textcolor{RoyalBlue}{\#ifdef}}\ BBOUND \\
\mbox{}\textbf{\textcolor{Blue}{struct}}\ \textcolor{TealBlue}{cmp}\textcolor{Red}{\{} \\
\mbox{}\ \ \ \ \textcolor{ForestGreen}{bool}\ \textbf{\textcolor{Blue}{operator}}\textcolor{BrickRed}{()}\ \textcolor{BrickRed}{(}\textbf{\textcolor{Blue}{const}}\ Mochila\textcolor{BrickRed}{\&}\ una\textcolor{BrickRed}{,}\  \\
\mbox{}\ \ \ \ \ \ \ \ \ \ \ \ \ \ \ \ \ \ \ \ \ \textbf{\textcolor{Blue}{const}}\ Mochila\textcolor{BrickRed}{\&}\ otra\textcolor{BrickRed}{)}\textcolor{Red}{\{} \\
\mbox{}\ \ \ \ \ \ \ \ \textbf{\textcolor{Blue}{return}}\ una\textcolor{BrickRed}{.}second\ \textcolor{BrickRed}{\textless{}}\ otra\textcolor{BrickRed}{.}second\textcolor{BrickRed}{;} \\
\mbox{}\ \ \ \ \textcolor{Red}{\}} \\
\mbox{}\textcolor{Red}{\}}\textcolor{BrickRed}{;} \\
\mbox{} \\
\mbox{}\textcolor{ForestGreen}{bool}\ \textbf{\textcolor{Black}{srt}}\textcolor{BrickRed}{(}\textbf{\textcolor{Blue}{const}}\ pair\textcolor{BrickRed}{\textless{}}\textcolor{ForestGreen}{double}\textcolor{BrickRed}{,}\textcolor{ForestGreen}{int}\textcolor{BrickRed}{\textgreater{}\&}\ uno\textcolor{BrickRed}{,}\ \textbf{\textcolor{Blue}{const}}\ pair\textcolor{BrickRed}{\textless{}}\textcolor{ForestGreen}{double}\textcolor{BrickRed}{,}\textcolor{ForestGreen}{int}\textcolor{BrickRed}{\textgreater{}\&}\ otro\textcolor{BrickRed}{)}\textcolor{Red}{\{} \\
\mbox{}\ \ \ \ \textbf{\textcolor{Blue}{return}}\ uno\textcolor{BrickRed}{.}first\ \textcolor{BrickRed}{\textless{}}\ otro\textcolor{BrickRed}{.}first\textcolor{BrickRed}{;} \\
\mbox{}\textcolor{Red}{\}} \\
\mbox{}\textbf{\textcolor{RoyalBlue}{\#endif}} \\
\mbox{} \\
\mbox{}\textcolor{TealBlue}{Mochila}\ \textbf{\textcolor{Black}{resolver}}\textcolor{BrickRed}{(}\textcolor{ForestGreen}{int}\ limite\textcolor{BrickRed}{,}\ \textcolor{TealBlue}{vector\textless{}int\textgreater{}\ pesos,\ vector\textless{}int\textgreater{}}\ beneficios\textcolor{BrickRed}{)}\ \textcolor{Red}{\{} \\
\mbox{}\ \ \ \ \textit{\textcolor{Brown}{//\ Resolución\ del\ problema}} \\
\mbox{}\ \ \ \ \textcolor{TealBlue}{uint}\ tamanio\ \textcolor{BrickRed}{=}\ pesos\textcolor{BrickRed}{.}\textbf{\textcolor{Black}{size}}\textcolor{BrickRed}{();} \\
\mbox{}\ \ \ \  \\
\mbox{}\textbf{\textcolor{RoyalBlue}{\ \ \ \ \#ifdef}}\ BBOUND \\
\mbox{}\ \ \ \ \ \ \ \ \textcolor{TealBlue}{priority$\_$queue\textless{}Mochila,\ vector\textless{}Mochila\textgreater{},\ cmp\textgreater{}}\ posibles$\_$mochilas\textcolor{BrickRed}{;} \\
\mbox{}\textbf{\textcolor{RoyalBlue}{\ \ \ \ \#else}} \\
\mbox{}\ \ \ \ \ \ \ \ \textcolor{TealBlue}{queue\textless{}Mochila\textgreater{}}\ posibles$\_$mochilas\textcolor{BrickRed}{;} \\
\mbox{}\textbf{\textcolor{RoyalBlue}{\ \ \ \ \#endif}} \\
\mbox{} \\
\mbox{}\ \ \ \ posibles$\_$mochilas\textcolor{BrickRed}{.}\textbf{\textcolor{Black}{push}}\textcolor{BrickRed}{(}\textbf{\textcolor{Black}{Mochila}}\textcolor{BrickRed}{(}vector\textcolor{BrickRed}{\textless{}}\textcolor{ForestGreen}{bool}\textcolor{BrickRed}{\textgreater{}(),}\textcolor{Purple}{0}\textcolor{BrickRed}{));} \\
\mbox{}\ \ \ \  \\
\mbox{}\ \ \ \ \textcolor{TealBlue}{Mochila}\ solucion\textcolor{BrickRed}{;} \\
\mbox{}\ \ \ \ \textcolor{ForestGreen}{int}\ max$\_$valor\ \textcolor{BrickRed}{=}\ \textcolor{Purple}{0}\textcolor{BrickRed}{;} \\
\mbox{}\ \ \ \  \\
\mbox{}\ \ \ \ \textit{\textcolor{Brown}{//\ Prueba\ cada\ una\ de\ las\ posibles\ mochilas.}} \\
\mbox{}\ \ \ \ \textbf{\textcolor{Blue}{while}}\ \textcolor{BrickRed}{(!}posibles$\_$mochilas\textcolor{BrickRed}{.}\textbf{\textcolor{Black}{empty}}\textcolor{BrickRed}{())}\ \textcolor{Red}{\{} \\
\mbox{}\textbf{\textcolor{RoyalBlue}{\ \ \ \ \ \ \ \ \#ifdef}}\ BBOUND \\
\mbox{}\ \ \ \ \ \ \ \ \ \ \ \ \textcolor{TealBlue}{Mochila}\ actual\ \textcolor{BrickRed}{=}\ posibles$\_$mochilas\textcolor{BrickRed}{.}\textbf{\textcolor{Black}{top}}\textcolor{BrickRed}{();} \\
\mbox{}\textbf{\textcolor{RoyalBlue}{\ \ \ \ \ \ \ \ \#else}} \\
\mbox{}\ \ \ \ \ \ \ \ \ \ \ \ \textcolor{TealBlue}{Mochila}\ actual\ \textcolor{BrickRed}{=}\ posibles$\_$mochilas\textcolor{BrickRed}{.}\textbf{\textcolor{Black}{front}}\textcolor{BrickRed}{();} \\
\mbox{}\textbf{\textcolor{RoyalBlue}{\ \ \ \ \ \ \ \ \#endif}} \\
\mbox{} \\
\mbox{}\ \ \ \ \ \ \ \ posibles$\_$mochilas\textcolor{BrickRed}{.}\textbf{\textcolor{Black}{pop}}\textcolor{BrickRed}{();} \\
\mbox{}\ \ \ \ \ \ \ \  \\
\mbox{}\ \ \ \ \ \ \ \ \textit{\textcolor{Brown}{//\ Caso\ de\ mochila\ llena}} \\
\mbox{}\ \ \ \ \ \ \ \ \textit{\textcolor{Brown}{//\ Calculamos\ su\ valor\ y\ si\ es\ mejor\ que\ la\ mejor\ mochila\ actual.}} \\
\mbox{}\ \ \ \ \ \ \ \ \textbf{\textcolor{Blue}{if}}\ \textcolor{BrickRed}{(}actual\textcolor{BrickRed}{.}first\textcolor{BrickRed}{.}\textbf{\textcolor{Black}{size}}\textcolor{BrickRed}{()}\ \textcolor{BrickRed}{==}\ tamanio\textcolor{BrickRed}{)}\ \textcolor{Red}{\{} \\
\mbox{}\ \ \ \ \ \ \ \ \ \ \ \ \textbf{\textcolor{Blue}{if}}\ \textcolor{BrickRed}{(}actual\textcolor{BrickRed}{.}second\ \textcolor{BrickRed}{\textgreater{}}\ max$\_$valor\textcolor{BrickRed}{)}\ \textcolor{Red}{\{} \\
\mbox{}\ \ \ \ \ \ \ \ \ \ \ \ \ \ \ \ max$\_$valor\ \textcolor{BrickRed}{=}\ actual\textcolor{BrickRed}{.}second\textcolor{BrickRed}{;} \\
\mbox{}\ \ \ \ \ \ \ \ \ \ \ \ \ \ \ \ solucion\ \textcolor{BrickRed}{=}\ actual\textcolor{BrickRed}{;} \\
\mbox{}\ \ \ \ \ \ \ \ \ \ \ \ \textcolor{Red}{\}} \\
\mbox{}\ \ \ \ \ \ \ \ \textcolor{Red}{\}} \\
\mbox{}\ \ \ \ \ \ \ \  \\
\mbox{}\ \ \ \ \ \ \ \ \textit{\textcolor{Brown}{//\ Caso\ de\ la\ mochila\ sin\ llenar}} \\
\mbox{}\ \ \ \ \ \ \ \ \textit{\textcolor{Brown}{//\ Rellenamos\ o\ no\ con\ el\ siguiente\ objeto.}} \\
\mbox{}\ \ \ \ \ \ \ \ \textit{\textcolor{Brown}{//\ Añadimos\ el\ nuevo\ si\ no\ excede\ el\ peso.}} \\
\mbox{}\ \ \ \ \ \ \ \ \textbf{\textcolor{Blue}{else}}\ \textcolor{Red}{\{} \\
\mbox{}\ \ \ \ \ \ \ \ \ \ \ \ \textcolor{TealBlue}{Mochila}\ con$\_$nuevo\ \textcolor{BrickRed}{=}\ actual\textcolor{BrickRed}{;} \\
\mbox{}\ \ \ \ \ \ \ \ \ \ \ \ \textcolor{TealBlue}{Mochila}\ sin$\_$nuevo\ \textcolor{BrickRed}{=}\ actual\textcolor{BrickRed}{;} \\
\mbox{}\ \ \ \ \ \ \ \ \ \ \ \  \\
\mbox{}\ \ \ \ \ \ \ \ \ \ \ \ con$\_$nuevo\textcolor{BrickRed}{.}second\ \textcolor{BrickRed}{+=}\ beneficios\textcolor{BrickRed}{[}actual\textcolor{BrickRed}{.}first\textcolor{BrickRed}{.}\textbf{\textcolor{Black}{size}}\textcolor{BrickRed}{()];} \\
\mbox{}\ \ \ \ \ \ \ \ \ \ \ \  \\
\mbox{}\ \ \ \ \ \ \ \ \ \ \ \ con$\_$nuevo\textcolor{BrickRed}{.}first\textcolor{BrickRed}{.}\textbf{\textcolor{Black}{push$\_$back}}\textcolor{BrickRed}{(}\textbf{\textcolor{Blue}{true}}\textcolor{BrickRed}{);} \\
\mbox{}\ \ \ \ \ \ \ \ \ \ \ \ sin$\_$nuevo\textcolor{BrickRed}{.}first\textcolor{BrickRed}{.}\textbf{\textcolor{Black}{push$\_$back}}\textcolor{BrickRed}{(}\textbf{\textcolor{Blue}{false}}\textcolor{BrickRed}{);} \\
\mbox{}\ \ \ \ \ \ \ \ \ \ \ \  \\
\mbox{}\ \ \ \ \ \ \ \ \ \ \ \ \textcolor{ForestGreen}{int}\ nuevo$\_$peso\ \textcolor{BrickRed}{=}\ \textcolor{Purple}{0}\textcolor{BrickRed}{;} \\
\mbox{}\ \ \ \ \ \ \ \ \ \ \ \ \textbf{\textcolor{Blue}{for}}\ \textcolor{BrickRed}{(}\textcolor{TealBlue}{uint}\ i\textcolor{BrickRed}{=}\textcolor{Purple}{0}\textcolor{BrickRed}{;}\ i\textcolor{BrickRed}{\textless{}}con$\_$nuevo\textcolor{BrickRed}{.}first\textcolor{BrickRed}{.}\textbf{\textcolor{Black}{size}}\textcolor{BrickRed}{();}\ i\textcolor{BrickRed}{++)}\textcolor{Red}{\{} \\
\mbox{}\ \ \ \ \ \ \ \ \ \ \ \ \ \ \ \ nuevo$\_$peso\ \textcolor{BrickRed}{+=}\ con$\_$nuevo\textcolor{BrickRed}{.}first\textcolor{BrickRed}{[}i\textcolor{BrickRed}{]}\ \textcolor{BrickRed}{?}\ pesos\textcolor{BrickRed}{[}i\textcolor{BrickRed}{]}\ \textcolor{BrickRed}{:}\ \textcolor{Purple}{0}\textcolor{BrickRed}{;} \\
\mbox{}\ \ \ \ \ \ \ \ \ \ \ \ \textcolor{Red}{\}} \\
\mbox{} \\
\mbox{}\ \ \ \ \ \ \ \ \ \ \ \ \textbf{\textcolor{Blue}{if}}\ \textcolor{BrickRed}{(}nuevo$\_$peso\ \textcolor{BrickRed}{\textless{}=}\ limite\textcolor{BrickRed}{)}\ \textcolor{Red}{\{} \\
\mbox{}\ \ \ \ \ \ \ \ \ \ \ \ \ \ \ \ posibles$\_$mochilas\textcolor{BrickRed}{.}\textbf{\textcolor{Black}{push}}\textcolor{BrickRed}{(}con$\_$nuevo\textcolor{BrickRed}{);} \\
\mbox{}\ \ \ \ \ \ \ \ \ \ \ \ \textcolor{Red}{\}} \\
\mbox{}\ \ \ \ \ \ \ \ \ \ \ \  \\
\mbox{}\textbf{\textcolor{RoyalBlue}{\ \ \ \ \ \ \ \ \ \ \ \ \#ifdef}}\ BBOUND \\
\mbox{}\ \ \ \ \ \ \ \ \ \ \ \ \textit{\textcolor{Brown}{//\ Calculamos\ una\ cota\ superior\ para\ el\ llenado\ de\ la\ parte\ de\ la\ mochila\ que\ falta}} \\
\mbox{}\ \ \ \ \ \ \ \ \ \ \ \ \textcolor{TealBlue}{vector\textless{}pair\textless{}double,int\textgreater{}\textgreater{}}\ w\textcolor{BrickRed}{;} \\
\mbox{}\ \ \ \ \ \ \ \ \ \ \ \ \textcolor{ForestGreen}{int}\ n\ \textcolor{BrickRed}{=}\ sin$\_$nuevo\textcolor{BrickRed}{.}first\textcolor{BrickRed}{.}\textbf{\textcolor{Black}{size}}\textcolor{BrickRed}{();} \\
\mbox{}\ \ \ \ \ \ \ \ \ \ \ \ \textcolor{ForestGreen}{int}\ restante\ \textcolor{BrickRed}{=}\ limite\ \textcolor{BrickRed}{-}\ nuevo$\_$peso\ \textcolor{BrickRed}{+}\ pesos\textcolor{BrickRed}{[}n\textcolor{BrickRed}{-}\textcolor{Purple}{1}\textcolor{BrickRed}{];} \\
\mbox{}\ \ \ \ \ \ \ \ \ \ \ \ \textcolor{ForestGreen}{int}\ \textbf{\textcolor{Black}{max$\_$beneficio}}\textcolor{BrickRed}{(}\textcolor{Purple}{0}\textcolor{BrickRed}{);} \\
\mbox{}\ \ \ \ \ \ \ \ \ \ \ \ \textcolor{ForestGreen}{double}\ coef\textcolor{BrickRed}{;} \\
\mbox{}\ \ \ \ \ \ \ \ \ \ \ \ \ \ \ \  \\
\mbox{}\ \ \ \ \ \ \ \ \ \ \ \ \textbf{\textcolor{Blue}{for}}\ \textcolor{BrickRed}{(}\textcolor{TealBlue}{uint}\ i\textcolor{BrickRed}{=}\textcolor{Purple}{0}\textcolor{BrickRed}{;}\ i\ \textcolor{BrickRed}{\textless{}}\ tamanio\ \textcolor{BrickRed}{-}\ n\textcolor{BrickRed}{;}\ \textcolor{BrickRed}{++}i\textcolor{BrickRed}{)}\ \ \ \ \ \ \ \ \ \ \ \ \ \ \ \  \\
\mbox{}\ \ \ \ \ \ \ \ \ \ \ \ \ \ \ \ w\textcolor{BrickRed}{.}\textbf{\textcolor{Black}{push$\_$back}}\textcolor{BrickRed}{(}\textbf{\textcolor{Black}{make$\_$pair}}\ \textcolor{BrickRed}{(}beneficios\textcolor{BrickRed}{[}i\textcolor{BrickRed}{+}n\textcolor{BrickRed}{]/}pesos\textcolor{BrickRed}{[}i\textcolor{BrickRed}{+}n\textcolor{BrickRed}{],}\ i\textcolor{BrickRed}{+}n\textcolor{BrickRed}{));} \\
\mbox{}\ \ \ \ \ \ \ \ \ \ \ \ \ \ \ \  \\
\mbox{}\ \ \ \ \ \ \ \ \ \ \ \ \textbf{\textcolor{Black}{sort}}\textcolor{BrickRed}{(}w\textcolor{BrickRed}{.}\textbf{\textcolor{Black}{begin}}\textcolor{BrickRed}{(),}\ w\textcolor{BrickRed}{.}\textbf{\textcolor{Black}{end}}\textcolor{BrickRed}{(),}\ srt\textcolor{BrickRed}{);} \\
\mbox{}\ \ \ \ \ \ \ \ \ \ \ \ \ \ \ \  \\
\mbox{}\ \ \ \ \ \ \ \ \ \ \ \ \textbf{\textcolor{Blue}{while}}\ \textcolor{BrickRed}{(}restante\ \textcolor{BrickRed}{!=}\ \textcolor{Purple}{0}\ \textcolor{BrickRed}{\&\&}\ \textcolor{BrickRed}{!}w\textcolor{BrickRed}{.}\textbf{\textcolor{Black}{empty}}\textcolor{BrickRed}{())}\textcolor{Red}{\{} \\
\mbox{}\ \ \ \ \ \ \ \ \ \ \ \ \ \ \ \ coef\ \textcolor{BrickRed}{=}\ \textbf{\textcolor{Black}{min}}\textcolor{BrickRed}{(}\textcolor{Purple}{1.0}\textcolor{BrickRed}{,}restante\textcolor{BrickRed}{*}\textcolor{Purple}{1.0}\textcolor{BrickRed}{/}pesos\textcolor{BrickRed}{[}w\textcolor{BrickRed}{.}\textbf{\textcolor{Black}{back}}\textcolor{BrickRed}{().}second\textcolor{BrickRed}{]);} \\
\mbox{}\ \ \ \ \ \ \ \ \ \ \ \ \ \ \ \ restante\ \textcolor{BrickRed}{-=}\ pesos\textcolor{BrickRed}{[}w\textcolor{BrickRed}{.}\textbf{\textcolor{Black}{back}}\textcolor{BrickRed}{().}second\textcolor{BrickRed}{]}\ \textcolor{BrickRed}{*}\ coef\textcolor{BrickRed}{;} \\
\mbox{}\ \ \ \ \ \ \ \ \ \ \ \ \ \ \ \ max$\_$beneficio\ \textcolor{BrickRed}{+=}\ beneficios\textcolor{BrickRed}{[}w\textcolor{BrickRed}{.}\textbf{\textcolor{Black}{back}}\textcolor{BrickRed}{().}second\textcolor{BrickRed}{];} \\
\mbox{}\ \ \ \ \ \ \ \ \ \ \ \ \ \ \ \ w\textcolor{BrickRed}{.}\textbf{\textcolor{Black}{pop$\_$back}}\textcolor{BrickRed}{();} \\
\mbox{}\ \ \ \ \ \ \ \ \ \ \ \ \textcolor{Red}{\}} \\
\mbox{}\ \ \ \ \ \ \ \ \ \ \ \ \ \ \ \  \\
\mbox{}\ \ \ \ \ \ \ \ \ \ \ \ \textit{\textcolor{Brown}{//\ Si\ dicha\ cota\ superior\ supera\ al\ mejor\ valor\ hasta\ el\ momento,}} \\
\mbox{}\ \ \ \ \ \ \ \ \ \ \ \ \textit{\textcolor{Brown}{//\ introducimos\ la\ nueva\ mochila,\ en\ caso\ opuesto\ no}} \\
\mbox{}\ \ \ \ \ \ \ \ \ \ \ \ \textbf{\textcolor{Blue}{if}}\ \textcolor{BrickRed}{(}max$\_$valor\ \textcolor{BrickRed}{\textless{}}\ sin$\_$nuevo\textcolor{BrickRed}{.}second\ \textcolor{BrickRed}{+}\ max$\_$beneficio\textcolor{BrickRed}{)} \\
\mbox{}\textbf{\textcolor{RoyalBlue}{\ \ \ \ \ \ \ \ \ \ \ \ \#endif}} \\
\mbox{}\ \ \ \ \ \ \ \ \ \ \ \ \ \ \ \ posibles$\_$mochilas\textcolor{BrickRed}{.}\textbf{\textcolor{Black}{push}}\textcolor{BrickRed}{(}sin$\_$nuevo\textcolor{BrickRed}{);} \\
\mbox{}\ \ \ \ \ \ \ \ \ \ \ \  \\
\mbox{}\ \ \ \ \ \ \ \ \textcolor{Red}{\}} \\
\mbox{}\ \ \ \ \textcolor{Red}{\}} \\
\mbox{} \\
\mbox{}\ \ \ \ \textbf{\textcolor{Blue}{return}}\ solucion\textcolor{BrickRed}{;} \\
\mbox{}\textcolor{Red}{\}}\  \\
\mbox{} \\
\mbox{}\textcolor{ForestGreen}{int}\ \textbf{\textcolor{Black}{main}}\ \textcolor{BrickRed}{()}\ \textcolor{Red}{\{} \\
\mbox{}\ \ \ \ \textit{\textcolor{Brown}{//\ Lectura\ del\ problema}} \\
\mbox{}\ \ \ \ \textcolor{ForestGreen}{int}\ limite\textcolor{BrickRed}{,}\ leido\textcolor{BrickRed}{;} \\
\mbox{}\ \ \ \ \textcolor{TealBlue}{uint}\ tamanio\textcolor{BrickRed}{;} \\
\mbox{}\ \ \ \ \textcolor{TealBlue}{vector\textless{}int\textgreater{}}\ pesos\textcolor{BrickRed}{;} \\
\mbox{}\ \ \ \ \textcolor{TealBlue}{vector\textless{}int\textgreater{}}\ beneficios\textcolor{BrickRed}{;} \\
\mbox{} \\
\mbox{}\ \ \ \ cin\ \textcolor{BrickRed}{\textgreater{}\textgreater{}}\ limite\textcolor{BrickRed}{;} \\
\mbox{}\ \ \ \ cin\ \textcolor{BrickRed}{\textgreater{}\textgreater{}}\ tamanio\textcolor{BrickRed}{;} \\
\mbox{}\ \ \ \  \\
\mbox{}\ \ \ \ \textbf{\textcolor{Blue}{for}}\ \textcolor{BrickRed}{(}\textcolor{TealBlue}{uint}\ i\textcolor{BrickRed}{=}\textcolor{Purple}{0}\textcolor{BrickRed}{;}\ i\textcolor{BrickRed}{\textless{}}tamanio\textcolor{BrickRed}{;}\ i\textcolor{BrickRed}{++)}\ \textcolor{Red}{\{} \\
\mbox{}\ \ \ \ \ \ \ \ cin\ \textcolor{BrickRed}{\textgreater{}\textgreater{}}\ leido\textcolor{BrickRed}{;} \\
\mbox{}\ \ \ \ \ \ \ \ pesos\textcolor{BrickRed}{.}\textbf{\textcolor{Black}{push$\_$back}}\textcolor{BrickRed}{(}leido\textcolor{BrickRed}{);} \\
\mbox{}\ \ \ \ \textcolor{Red}{\}} \\
\mbox{}\ \ \ \ \textbf{\textcolor{Blue}{for}}\ \textcolor{BrickRed}{(}\textcolor{TealBlue}{uint}\ i\textcolor{BrickRed}{=}\textcolor{Purple}{0}\textcolor{BrickRed}{;}\ i\textcolor{BrickRed}{\textless{}}tamanio\textcolor{BrickRed}{;}\ i\textcolor{BrickRed}{++)}\ \textcolor{Red}{\{} \\
\mbox{}\ \ \ \ \ \ \ \ cin\ \textcolor{BrickRed}{\textgreater{}\textgreater{}}\ leido\textcolor{BrickRed}{;} \\
\mbox{}\ \ \ \ \ \ \ \ beneficios\textcolor{BrickRed}{.}\textbf{\textcolor{Black}{push$\_$back}}\textcolor{BrickRed}{(}leido\textcolor{BrickRed}{);} \\
\mbox{}\ \ \ \ \textcolor{Red}{\}} \\
\mbox{}\ \ \ \  \\
\mbox{}\ \ \ \ \textbf{\textcolor{Blue}{auto}}\ time1\ \textcolor{BrickRed}{=}\ chrono\textcolor{BrickRed}{::}high$\_$resolution$\_$clock\textcolor{BrickRed}{::}\textbf{\textcolor{Black}{now}}\textcolor{BrickRed}{();} \\
\mbox{}\ \ \ \ \textcolor{TealBlue}{Mochila}\ resultado\ \textcolor{BrickRed}{=}\ \textbf{\textcolor{Black}{resolver}}\textcolor{BrickRed}{(}limite\textcolor{BrickRed}{,}\ pesos\textcolor{BrickRed}{,}\ beneficios\textcolor{BrickRed}{);} \\
\mbox{}\ \ \ \ \textbf{\textcolor{Blue}{auto}}\ time2\ \textcolor{BrickRed}{=}\ chrono\textcolor{BrickRed}{::}high$\_$resolution$\_$clock\textcolor{BrickRed}{::}\textbf{\textcolor{Black}{now}}\textcolor{BrickRed}{();} \\
\mbox{}\ \ \ \ chrono\textcolor{BrickRed}{::}\textcolor{TealBlue}{duration\textless{}double\textgreater{}}\ time$\_$span\ \textcolor{BrickRed}{=}\ chrono\textcolor{BrickRed}{::}duration$\_$cast\textcolor{BrickRed}{\textless{}}chrono\textcolor{BrickRed}{::}duration\textcolor{BrickRed}{\textless{}}\textcolor{ForestGreen}{double}\textcolor{BrickRed}{\textgreater{}\textgreater{}(}time2\ \textcolor{BrickRed}{-}\ time1\textcolor{BrickRed}{);} \\
\mbox{}\ \ \ \ \textcolor{ForestGreen}{double}\ time\ \textcolor{BrickRed}{=}\ time$\_$span\textcolor{BrickRed}{.}\textbf{\textcolor{Black}{count}}\textcolor{BrickRed}{();} \\
\mbox{}\ \ \ \ \textit{\textcolor{Brown}{//\ Muestra\ la\ solución.}} \\
\mbox{}\ \ \ \  \\
\mbox{}\ \ \ \ cout\ \textcolor{BrickRed}{\textless{}\textless{}}\ endl\ \textcolor{BrickRed}{\textless{}\textless{}}\ \texttt{\textcolor{Red}{"{}SOLUCIÓN:"{}}}\ \textcolor{BrickRed}{\textless{}\textless{}}\ endl\  \\
\mbox{}\ \ \ \ \ \ \ \ \textcolor{BrickRed}{\textless{}\textless{}}\ \texttt{\textcolor{Red}{"{}Valor\ obtenido:\ "{}}}\ \textcolor{BrickRed}{\textless{}\textless{}}\ resultado\textcolor{BrickRed}{.}second\  \\
\mbox{}\ \ \ \ \ \ \ \ \textcolor{BrickRed}{\textless{}\textless{}}\ endl\ \textcolor{BrickRed}{\textless{}\textless{}}\ \texttt{\textcolor{Red}{"{}Mochila:\ "{}}}\ \textcolor{BrickRed}{\textless{}\textless{}}\ resultado\textcolor{BrickRed}{.}first\  \\
\mbox{}\ \ \ \ \ \ \ \ \textcolor{BrickRed}{\textless{}\textless{}}\ endl\ \textcolor{BrickRed}{\textless{}\textless{}}\ \texttt{\textcolor{Red}{"{}Tiempo\ de\ cómputo:\ "{}}}\ \textcolor{BrickRed}{\textless{}\textless{}}\ time\ \textcolor{BrickRed}{\textless{}\textless{}}\ endl\textcolor{BrickRed}{;} \\
\mbox{}\textcolor{Red}{\}} \\
\mbox{}
}
        \normalsize
        
        Versión en Haskell:
        
        \small
  	\texttt{% Generator: GNU source-highlight, by Lorenzo Bettini, http://www.gnu.org/software/src-highlite
\noindent
\mbox{}\textit{\textcolor{Brown}{-\/-\ Función\ de\ Backtracking}} \\
\mbox{}\textit{\textcolor{Brown}{\{-\ Genera\ todas\ las\ posibles\ soluciones\ a\ un\ problema\ de\ backtracking.}} \\
\mbox{}\textit{\textcolor{Brown}{\ \ \ Pide\ una\ función\ que\ devuelve\ los\ nodos\ hijos\ de\ una\ pseudosolución}} \\
\mbox{}\textit{\textcolor{Brown}{\ \ \ y\ una\ función\ de\ poda\ que\ devuelva\ verdadero\ o\ falso\ sobre\ cada}} \\
\mbox{}\textit{\textcolor{Brown}{\ \ \ pseudosolución.}} \\
\mbox{} \\
\mbox{}\textit{\textcolor{Brown}{\ \ \ Las\ soluciones\ sin\ pseudosoluciones\ hijas\ se\ consideran\ soluciones\ finales.}} \\
\mbox{}\textit{\textcolor{Brown}{\ -\}}} \\
\mbox{}backtracking\ \textcolor{BrickRed}{::}\ \textcolor{BrickRed}{(}a\ \textcolor{BrickRed}{-\textgreater{}}\ \textcolor{BrickRed}{[}a\textcolor{BrickRed}{])}\ \textcolor{BrickRed}{-\textgreater{}}\ \textcolor{BrickRed}{(}a\ \textcolor{BrickRed}{-\textgreater{}}\ \textcolor{ForestGreen}{Bool}\textcolor{BrickRed}{)}\ \textcolor{BrickRed}{-\textgreater{}}\ a\ \textcolor{BrickRed}{-\textgreater{}}\ \textcolor{BrickRed}{[}a\textcolor{BrickRed}{]} \\
\mbox{}backtracking\ nodes\ pred\ x \\
\mbox{}\ \ \ \ \textcolor{BrickRed}{$|$}\ \textcolor{BrickRed}{(}pred\ x\ \textcolor{BrickRed}{\&\&}\ \textcolor{BrickRed}{(}length\ \textcolor{BrickRed}{(}nodes\ x\textcolor{BrickRed}{)}\ \textcolor{BrickRed}{==}\ \textcolor{Purple}{0}\textcolor{BrickRed}{))}\ \textcolor{BrickRed}{=}\ \textcolor{BrickRed}{[}x\textcolor{BrickRed}{]} \\
\mbox{}\ \ \ \ \textcolor{BrickRed}{$|$}\ pred\ x\ \textcolor{BrickRed}{=}\ concat\ \textcolor{BrickRed}{\$}\ map\ \textcolor{BrickRed}{(}backtracking\ nodes\ pred\textcolor{BrickRed}{)}\ \textcolor{BrickRed}{(}nodes\ x\textcolor{BrickRed}{)} \\
\mbox{}\ \ \ \ \textcolor{BrickRed}{$|$}\ otherwise\ \textcolor{BrickRed}{=}\ \textcolor{BrickRed}{[]} \\
\mbox{}\ \ \ \  \\
\mbox{}\textit{\textcolor{Brown}{\{-\ Mochila\ -\}}} \\
\mbox{}\textbf{\textcolor{Blue}{type}}\ \textcolor{ForestGreen}{Mochila}\ \textcolor{BrickRed}{=}\ \textcolor{BrickRed}{([}\textcolor{ForestGreen}{Int}\textcolor{BrickRed}{],}\ \textcolor{ForestGreen}{Int}\textcolor{BrickRed}{)} \\
\mbox{}mochilaBT\ \textcolor{BrickRed}{::}\ \textcolor{BrickRed}{[}\textcolor{ForestGreen}{Int}\textcolor{BrickRed}{]}\ \textcolor{BrickRed}{-\textgreater{}}\ \textcolor{BrickRed}{[}\textcolor{ForestGreen}{Int}\textcolor{BrickRed}{]}\ \textcolor{BrickRed}{-\textgreater{}}\ \textcolor{ForestGreen}{Int}\ \textcolor{BrickRed}{-\textgreater{}}\ \textcolor{ForestGreen}{Mochila}\ \textcolor{BrickRed}{-\textgreater{}}\ \textcolor{BrickRed}{[}\textcolor{ForestGreen}{Mochila}\textcolor{BrickRed}{]} \\
\mbox{}mochilaBT\ weights\ benefits\ limit\ x\ \textcolor{BrickRed}{=}\ backtracking\ push\ isFull\ x \\
\mbox{}\ \ \ \ \ \ \ \ \ \ \textbf{\textcolor{Blue}{where}} \\
\mbox{}\ \ \ \ \ \ \ \ \ \ \ \ totalWeight\ x\ \textcolor{BrickRed}{=}\ foldr\ \textcolor{BrickRed}{(+)}\ \textcolor{Purple}{0}\ \textcolor{BrickRed}{(}map\ \textcolor{BrickRed}{(}weights\ \textcolor{BrickRed}{!!)}\ x\textcolor{BrickRed}{)} \\
\mbox{}\ \ \ \ \ \ \ \ \ \ \ \ isFull\ \textcolor{BrickRed}{(}xs\textcolor{BrickRed}{,}k\textcolor{BrickRed}{)}\ \textcolor{BrickRed}{=}\ \textcolor{BrickRed}{((}totalWeight\ xs\textcolor{BrickRed}{)}\ \textcolor{BrickRed}{\textgreater{}}\ limit\textcolor{BrickRed}{)} \\
\mbox{}\ \ \ \ \ \ \ \ \ \ \ \ push\ \textcolor{BrickRed}{(}xs\textcolor{BrickRed}{,}k\textcolor{BrickRed}{)} \\
\mbox{}\ \ \ \ \ \ \ \ \ \ \ \ \ \ \textcolor{BrickRed}{$|$}\ k\ \textcolor{BrickRed}{\textless{}}\ \textcolor{BrickRed}{(}length\ benefits\textcolor{BrickRed}{)}\ \textcolor{BrickRed}{=}\ \textcolor{BrickRed}{[((}xs\textcolor{BrickRed}{:}k\textcolor{BrickRed}{),}k\textcolor{BrickRed}{+}\textcolor{Purple}{1}\textcolor{BrickRed}{),}\ \textcolor{BrickRed}{(}xs\textcolor{BrickRed}{,}k\textcolor{BrickRed}{+}\textcolor{Purple}{1}\textcolor{BrickRed}{)]} \\
\mbox{}\ \ \ \ \ \ \ \ \ \ \ \ \ \ \textcolor{BrickRed}{$|$}\ otherwise\ \textcolor{BrickRed}{=}\ \textcolor{BrickRed}{[]} \\
\mbox{}
}
        \normalsize
        
  \subsection{Travelling Salesman Problem}
        \small
  	\texttt{% Generator: GNU source-highlight, by Lorenzo Bettini, http://www.gnu.org/software/src-highlite
\noindent
\mbox{}\textit{\textcolor{Brown}{/**}} \\
\mbox{}\textit{\textcolor{Brown}{\ *\ tsp.cpp}} \\
\mbox{}\textit{\textcolor{Brown}{\ *\ Traveling\ Salesman\ Problem.}} \\
\mbox{}\textit{\textcolor{Brown}{\ *\ Implementación\ de\ un\ algoritmo\ de\ backtracking\ en\ C++.}} \\
\mbox{}\textit{\textcolor{Brown}{\ */}} \\
\mbox{} \\
\mbox{}\textbf{\textcolor{RoyalBlue}{\#include}}\ \texttt{\textcolor{Red}{\textless{}iostream\textgreater{}}} \\
\mbox{}\textbf{\textcolor{RoyalBlue}{\#include}}\ \texttt{\textcolor{Red}{\textless{}vector\textgreater{}}} \\
\mbox{}\textbf{\textcolor{RoyalBlue}{\#include}}\ \texttt{\textcolor{Red}{\textless{}cmath\textgreater{}}} \\
\mbox{}\textbf{\textcolor{RoyalBlue}{\#include}}\ \texttt{\textcolor{Red}{\textless{}limits\textgreater{}}} \\
\mbox{}\textbf{\textcolor{RoyalBlue}{\#include}}\ \texttt{\textcolor{Red}{\textless{}numeric\textgreater{}}} \\
\mbox{}\textbf{\textcolor{RoyalBlue}{\#include}}\ \texttt{\textcolor{Red}{\textless{}chrono\textgreater{}}} \\
\mbox{}\textbf{\textcolor{Blue}{using}}\ \textbf{\textcolor{Blue}{namespace}}\ std\textcolor{BrickRed}{;} \\
\mbox{} \\
\mbox{}\textbf{\textcolor{Blue}{typedef}}\ \textcolor{ForestGreen}{unsigned}\ \textcolor{ForestGreen}{int}\ uint\textcolor{BrickRed}{;} \\
\mbox{}\textbf{\textcolor{Blue}{typedef}}\ \textcolor{TealBlue}{pair\textless{}double,double\textgreater{}}\ Point\textcolor{BrickRed}{;} \\
\mbox{}\textbf{\textcolor{Blue}{typedef}}\ \textcolor{TealBlue}{vector\textless{}int\textgreater{}}\ Ruta\textcolor{BrickRed}{;} \\
\mbox{}\textbf{\textcolor{Blue}{typedef}}\ \textcolor{ForestGreen}{float}\ Coste\textcolor{BrickRed}{;} \\
\mbox{} \\
\mbox{} \\
\mbox{}\textit{\textcolor{Brown}{//\ Función\ de\ impresión\ de\ vectores}} \\
\mbox{}\textbf{\textcolor{Blue}{template}}\textcolor{BrickRed}{\textless{}}\textbf{\textcolor{Blue}{class}}\ \textcolor{TealBlue}{T}\textcolor{BrickRed}{\textgreater{}} \\
\mbox{}ostream\textcolor{BrickRed}{\&}\ \textbf{\textcolor{Blue}{operator}}\textcolor{BrickRed}{\textless{}\textless{}}\ \textcolor{BrickRed}{(}ostream\textcolor{BrickRed}{\&}\ output\textcolor{BrickRed}{,}\ vector\textcolor{BrickRed}{\textless{}}T\textcolor{BrickRed}{\textgreater{}\&}\ v\textcolor{BrickRed}{)}\textcolor{Red}{\{} \\
\mbox{}\ \ \ \ \textbf{\textcolor{Blue}{for}}\ \textcolor{BrickRed}{(}\textbf{\textcolor{Blue}{auto}}\ i\ \textcolor{BrickRed}{:}\ v\textcolor{BrickRed}{)} \\
\mbox{}\ \ \ \ \ \ \ \ output\ \textcolor{BrickRed}{\textless{}\textless{}}\ i\ \textcolor{BrickRed}{\textless{}\textless{}}\ \texttt{\textcolor{Red}{'\ '}}\textcolor{BrickRed}{;} \\
\mbox{}\ \ \ \  \\
\mbox{}\ \ \ \ output\ \textcolor{BrickRed}{\textless{}\textless{}}\ endl\textcolor{BrickRed}{;} \\
\mbox{}\ \ \ \ \textbf{\textcolor{Blue}{return}}\ output\textcolor{BrickRed}{;} \\
\mbox{}\textcolor{Red}{\}} \\
\mbox{} \\
\mbox{} \\
\mbox{}\textcolor{TealBlue}{vector\textless{}Point\textgreater{}}\ ciudades\textcolor{BrickRed}{;} \\
\mbox{}\textcolor{TealBlue}{uint}\ dimension\textcolor{BrickRed}{;} \\
\mbox{}\textcolor{TealBlue}{Ruta}\ mejor$\_$ruta\textcolor{BrickRed}{;} \\
\mbox{}\textcolor{TealBlue}{Coste}\ mejor$\_$coste\textcolor{BrickRed}{;} \\
\mbox{} \\
\mbox{} \\
\mbox{}\textcolor{TealBlue}{Coste}\ \textbf{\textcolor{Black}{distancia}}\ \textcolor{BrickRed}{(}\textcolor{ForestGreen}{int}\ i\textcolor{BrickRed}{,}\ \textcolor{ForestGreen}{int}\ j\textcolor{BrickRed}{)}\ \textcolor{Red}{\{} \\
\mbox{}\ \ \ \ \textit{\textcolor{Brown}{//\ Calcula\ la\ distancia\ entre\ dos\ puntos.}} \\
\mbox{}\ \ \ \ \textcolor{TealBlue}{Point}\ a\ \textcolor{BrickRed}{=}\ ciudades\textcolor{BrickRed}{[}i\textcolor{BrickRed}{];} \\
\mbox{}\ \ \ \ \textcolor{TealBlue}{Point}\ b\ \textcolor{BrickRed}{=}\ ciudades\textcolor{BrickRed}{[}j\textcolor{BrickRed}{];} \\
\mbox{}\ \ \ \  \\
\mbox{}\ \ \ \ \textcolor{TealBlue}{Coste}\ x\ \textcolor{BrickRed}{=}\ a\textcolor{BrickRed}{.}first\ \textcolor{BrickRed}{-}\ b\textcolor{BrickRed}{.}first\textcolor{BrickRed}{;} \\
\mbox{}\ \ \ \ \textcolor{TealBlue}{Coste}\ y\ \textcolor{BrickRed}{=}\ a\textcolor{BrickRed}{.}second\ \textcolor{BrickRed}{-}\ b\textcolor{BrickRed}{.}second\textcolor{BrickRed}{;} \\
\mbox{}\ \ \ \  \\
\mbox{}\ \ \ \ \textbf{\textcolor{Blue}{return}}\ \textbf{\textcolor{Black}{sqrt}}\textcolor{BrickRed}{(}x\textcolor{BrickRed}{*}x\ \textcolor{BrickRed}{+}\ y\textcolor{BrickRed}{*}y\textcolor{BrickRed}{);} \\
\mbox{}\textcolor{Red}{\}} \\
\mbox{} \\
\mbox{}\textbf{\textcolor{RoyalBlue}{\#ifdef}}\ OPTBOUND \\
\mbox{}\textcolor{ForestGreen}{bool}\ \textbf{\textcolor{Black}{cruce}}\ \textcolor{BrickRed}{(}\textcolor{ForestGreen}{int}\ u\textcolor{BrickRed}{,}\ \textcolor{ForestGreen}{int}\ v\textcolor{BrickRed}{,}\ \textcolor{ForestGreen}{int}\ w\textcolor{BrickRed}{,}\ \textcolor{ForestGreen}{int}\ z\textcolor{BrickRed}{)}\ \textcolor{Red}{\{} \\
\mbox{}\ \ \ \ \textit{\textcolor{Brown}{//\ Calcula\ si\ se\ cruzan\ los\ segmentos\ uv,\ wz.}} \\
\mbox{}\ \ \ \ \textbf{\textcolor{Blue}{return}}\ \textbf{\textcolor{Black}{distancia}}\textcolor{BrickRed}{(}u\textcolor{BrickRed}{,}v\textcolor{BrickRed}{)}\ \textcolor{BrickRed}{+}\ \textbf{\textcolor{Black}{distancia}}\textcolor{BrickRed}{(}w\textcolor{BrickRed}{,}z\textcolor{BrickRed}{)}\ \textcolor{BrickRed}{\textgreater{}}\ \textbf{\textcolor{Black}{distancia}}\textcolor{BrickRed}{(}u\textcolor{BrickRed}{,}w\textcolor{BrickRed}{)}\ \textcolor{BrickRed}{+}\ \textbf{\textcolor{Black}{distancia}}\textcolor{BrickRed}{(}v\textcolor{BrickRed}{,}z\textcolor{BrickRed}{);} \\
\mbox{}\textcolor{Red}{\}} \\
\mbox{}\textbf{\textcolor{RoyalBlue}{\#endif}} \\
\mbox{} \\
\mbox{}\textcolor{ForestGreen}{void}\ \textbf{\textcolor{Black}{permutaciones}}\textcolor{BrickRed}{(}Ruta\textcolor{BrickRed}{\&}\ ruta\textcolor{BrickRed}{,}\ \textcolor{TealBlue}{Coste}\ coste$\_$actual\textcolor{BrickRed}{,}\ \textcolor{TealBlue}{uint}\ indice\textcolor{BrickRed}{)}\textcolor{Red}{\{} \\
\mbox{}\ \ \ \ \textit{\textcolor{Brown}{//\ Caso\ de\ la\ ruta\ finalizada}} \\
\mbox{}\ \ \ \ \textit{\textcolor{Brown}{//\ Comprueba\ si\ se\ mejora\ el\ óptimo.\ \ \ \ }} \\
\mbox{}\ \ \ \ \textbf{\textcolor{Blue}{if}}\ \textcolor{BrickRed}{(}indice\ \textcolor{BrickRed}{==}\ dimension\textcolor{BrickRed}{)}\ \textcolor{Red}{\{} \\
\mbox{}\ \ \ \ \ \ \ \ \ \ \ \ \textcolor{TealBlue}{Coste}\ coste$\_$total\ \textcolor{BrickRed}{=}\ coste$\_$actual\ \textcolor{BrickRed}{+}\ \textbf{\textcolor{Black}{distancia}}\textcolor{BrickRed}{(}ruta\textcolor{BrickRed}{[}indice\textcolor{BrickRed}{-}\textcolor{Purple}{1}\textcolor{BrickRed}{],}\ ruta\textcolor{BrickRed}{[}\textcolor{Purple}{0}\textcolor{BrickRed}{]);} \\
\mbox{} \\
\mbox{}\ \ \ \ \ \ \ \ \ \ \ \ \textbf{\textcolor{Blue}{if}}\ \textcolor{BrickRed}{(}coste$\_$total\ \textcolor{BrickRed}{\textless{}}\ mejor$\_$coste\textcolor{BrickRed}{)}\ \textcolor{Red}{\{} \\
\mbox{}\ \ \ \ \ \ \ \ \ \ \ \ \ \ \ \ mejor$\_$ruta\ \textcolor{BrickRed}{=}\ ruta\textcolor{BrickRed}{;} \\
\mbox{}\ \ \ \ \ \ \ \ \ \ \ \ \ \ \ \ mejor$\_$coste\ \textcolor{BrickRed}{=}\ coste$\_$total\textcolor{BrickRed}{;} \\
\mbox{}\ \ \ \ \ \ \ \ \ \ \ \ \textcolor{Red}{\}} \\
\mbox{}\ \ \ \ \textcolor{Red}{\}} \\
\mbox{}\ \ \ \  \\
\mbox{}\textbf{\textcolor{RoyalBlue}{\ \ \ \ \#ifdef}}\ BBOUND \\
\mbox{}\ \ \ \ \textit{\textcolor{Brown}{//\ Caso\ de\ superar\ el\ coste\ óptimo.}} \\
\mbox{}\ \ \ \ \textit{\textcolor{Brown}{//\ No\ es\ necesario\ seguir\ estudiando\ este\ caso.}} \\
\mbox{}\ \ \ \ \textbf{\textcolor{Blue}{else}}\ \textbf{\textcolor{Blue}{if}}\ \textcolor{BrickRed}{(}coste$\_$actual\ \textcolor{BrickRed}{\textgreater{}}\ mejor$\_$coste\textcolor{BrickRed}{)} \\
\mbox{}\ \ \ \ \ \ \ \ \textbf{\textcolor{Blue}{return}}\textcolor{BrickRed}{;} \\
\mbox{}\textbf{\textcolor{RoyalBlue}{\ \ \ \ \#endif}} \\
\mbox{}\ \ \ \  \\
\mbox{} \\
\mbox{}\ \ \ \ \textit{\textcolor{Brown}{//\ Caso\ de\ recorrido\ intermedio}} \\
\mbox{}\ \ \ \ \textit{\textcolor{Brown}{//\ Prueba\ posibles\ permutaciones\ para\ los\ restantes\ elementos.}} \\
\mbox{}\ \ \ \ \textbf{\textcolor{Blue}{else}}\ \textcolor{Red}{\{} \\
\mbox{}\ \ \ \ \ \ \ \ \textbf{\textcolor{Blue}{for}}\ \textcolor{BrickRed}{(}\textcolor{TealBlue}{uint}\ i\ \textcolor{BrickRed}{=}\ indice\textcolor{BrickRed}{;}\ i\ \textcolor{BrickRed}{\textless{}}\ dimension\textcolor{BrickRed}{;}\ \textcolor{BrickRed}{++}i\textcolor{BrickRed}{)}\ \textcolor{Red}{\{} \\
\mbox{}\textbf{\textcolor{RoyalBlue}{\ \ \ \ \ \ \ \ \ \ \ \ \ \ \ \ \#ifdef}}\ OPTBOUND \\
\mbox{}\ \ \ \ \ \ \ \ \ \ \ \ \ \ \ \ \textit{\textcolor{Brown}{//\ Caso\ en\ el\ que\ la\ permutación\ introduciría\ un\ cruce\ de\ caminos.}} \\
\mbox{}\ \ \ \ \ \ \ \ \ \ \ \ \ \ \ \ \textit{\textcolor{Brown}{//\ Por\ optimización\ OPT-2,\ no\ puede\ ser\ el\ óptimo.}} \\
\mbox{}\ \ \ \ \ \ \ \ \ \ \ \ \ \ \ \ \textcolor{ForestGreen}{bool}\ opt2\ \textcolor{BrickRed}{=}\ \textbf{\textcolor{Blue}{false}}\textcolor{BrickRed}{;} \\
\mbox{}\ \ \ \ \ \ \ \ \ \ \ \ \ \ \ \ \textbf{\textcolor{Blue}{for}}\ \textcolor{BrickRed}{(}\textcolor{TealBlue}{uint}\ j\ \textcolor{BrickRed}{=}\ \textcolor{Purple}{1}\textcolor{BrickRed}{;}\ j\ \textcolor{BrickRed}{\textless{}}\ \textcolor{TealBlue}{indice}\ and\ \textcolor{BrickRed}{!}opt2\textcolor{BrickRed}{;}\ j\textcolor{BrickRed}{++)} \\
\mbox{}\ \ \ \ \ \ \ \ \ \ \ \ \ \ \ \ \ \ \ \ \ \ \ \ opt2\ \textcolor{BrickRed}{=}\ \textbf{\textcolor{Black}{cruce}}\textcolor{BrickRed}{(}ruta\textcolor{BrickRed}{[}i\textcolor{BrickRed}{],}ruta\textcolor{BrickRed}{[}indice\textcolor{BrickRed}{-}\textcolor{Purple}{1}\textcolor{BrickRed}{],}\ ruta\textcolor{BrickRed}{[}j\textcolor{BrickRed}{],}ruta\textcolor{BrickRed}{[}j\textcolor{BrickRed}{-}\textcolor{Purple}{1}\textcolor{BrickRed}{]);} \\
\mbox{} \\
\mbox{}\ \ \ \ \ \ \ \ \ \ \ \ \ \ \ \ \textbf{\textcolor{Blue}{if}}\ \textcolor{BrickRed}{(}opt2\textcolor{BrickRed}{)} \\
\mbox{}\ \ \ \ \ \ \ \ \ \ \ \ \ \ \ \ \ \ \ \ \ \ \ \ \textbf{\textcolor{Blue}{continue}}\textcolor{BrickRed}{;} \\
\mbox{}\textbf{\textcolor{RoyalBlue}{\ \ \ \ \ \ \ \ \ \ \ \ \#endif}} \\
\mbox{} \\
\mbox{} \\
\mbox{}\ \ \ \ \ \ \ \ \ \ \ \ \textit{\textcolor{Brown}{//\ Produce\ una\ permutación\ en\ la\ ruta.}} \\
\mbox{}\ \ \ \ \ \ \ \ \ \ \ \ \textcolor{TealBlue}{uint}\ temp\ \textcolor{BrickRed}{=}\ ruta\textcolor{BrickRed}{[}i\textcolor{BrickRed}{];} \\
\mbox{}\ \ \ \ \ \ \ \ \ \ \ \ ruta\textcolor{BrickRed}{[}i\textcolor{BrickRed}{]}\ \textcolor{BrickRed}{=}\ ruta\textcolor{BrickRed}{[}indice\textcolor{BrickRed}{];} \\
\mbox{}\ \ \ \ \ \ \ \ \ \ \ \ ruta\textcolor{BrickRed}{[}indice\textcolor{BrickRed}{]}\ \textcolor{BrickRed}{=}\ temp\textcolor{BrickRed}{;} \\
\mbox{}\ \ \ \ \ \ \ \ \ \ \ \  \\
\mbox{}\ \ \ \ \ \ \ \ \ \ \ \ coste$\_$actual\ \textcolor{BrickRed}{+=}\ \textbf{\textcolor{Black}{distancia}}\textcolor{BrickRed}{(}ruta\textcolor{BrickRed}{[}indice\ \textcolor{BrickRed}{-}\ \textcolor{Purple}{1}\textcolor{BrickRed}{],}\ ruta\textcolor{BrickRed}{[}indice\textcolor{BrickRed}{]);} \\
\mbox{}\ \ \ \ \ \ \ \ \ \ \ \  \\
\mbox{}\ \ \ \ \ \ \ \ \ \ \ \ \textit{\textcolor{Brown}{//\ Estudia\ permutaciones\ con\ ese\ cambio.}} \\
\mbox{}\ \ \ \ \ \ \ \ \ \ \ \ \textbf{\textcolor{Black}{permutaciones}}\ \textcolor{BrickRed}{(}ruta\textcolor{BrickRed}{,}\ coste$\_$actual\textcolor{BrickRed}{,}\ indice\ \textcolor{BrickRed}{+}\ \textcolor{Purple}{1}\textcolor{BrickRed}{);} \\
\mbox{}\ \ \ \ \ \ \ \ \ \ \ \  \\
\mbox{}\ \ \ \ \ \ \ \ \ \ \ \ \textit{\textcolor{Brown}{//\ Deshace\ la\ permutación.}} \\
\mbox{}\ \ \ \ \ \ \ \ \ \ \ \ coste$\_$actual\ \textcolor{BrickRed}{-=}\ \textbf{\textcolor{Black}{distancia}}\textcolor{BrickRed}{(}ruta\textcolor{BrickRed}{[}indice\ \textcolor{BrickRed}{-}\ \textcolor{Purple}{1}\textcolor{BrickRed}{],}\ ruta\textcolor{BrickRed}{[}indice\textcolor{BrickRed}{]);} \\
\mbox{}\ \ \ \ \ \ \ \ \ \ \ \ ruta\textcolor{BrickRed}{[}indice\textcolor{BrickRed}{]}\ \textcolor{BrickRed}{=}\ ruta\textcolor{BrickRed}{[}i\textcolor{BrickRed}{];} \\
\mbox{}\ \ \ \ \ \ \ \ \ \ \ \ ruta\textcolor{BrickRed}{[}i\textcolor{BrickRed}{]}\ \textcolor{BrickRed}{=}\ temp\textcolor{BrickRed}{;} \\
\mbox{}\ \ \ \ \ \ \ \ \textcolor{Red}{\}} \\
\mbox{}\ \ \ \ \textcolor{Red}{\}} \\
\mbox{}\textcolor{Red}{\}} \\
\mbox{} \\
\mbox{} \\
\mbox{}\textcolor{ForestGreen}{int}\ \textbf{\textcolor{Black}{main}}\textcolor{BrickRed}{()}\ \textcolor{Red}{\{} \\
\mbox{}\ \ \ \ \textcolor{TealBlue}{Coste}\ \textbf{\textcolor{Black}{coste$\_$actual}}\textcolor{BrickRed}{(}\textcolor{Purple}{0}\textcolor{BrickRed}{);} \\
\mbox{}\ \ \ \ mejor$\_$coste\ \textcolor{BrickRed}{=}\ numeric$\_$limits\textcolor{BrickRed}{\textless{}}Coste\textcolor{BrickRed}{\textgreater{}::}\textbf{\textcolor{Black}{infinity}}\textcolor{BrickRed}{();} \\
\mbox{}\ \ \ \  \\
\mbox{}\ \ \ \ \textit{\textcolor{Brown}{//\ Lectura\ del\ problema}} \\
\mbox{}\ \ \ \ cin\ \textcolor{BrickRed}{\textgreater{}\textgreater{}}\ dimension\textcolor{BrickRed}{;} \\
\mbox{}\ \ \ \ ciudades\textcolor{BrickRed}{.}\textbf{\textcolor{Black}{resize}}\textcolor{BrickRed}{(}dimension\textcolor{BrickRed}{);} \\
\mbox{}\ \ \ \  \\
\mbox{}\ \ \ \ \textbf{\textcolor{Blue}{for}}\ \textcolor{BrickRed}{(}\textbf{\textcolor{Blue}{auto}}\textcolor{BrickRed}{\&}\ p\ \textcolor{BrickRed}{:}\ ciudades\textcolor{BrickRed}{)} \\
\mbox{}\ \ \ \ \ \ \ \ cin\ \textcolor{BrickRed}{\textgreater{}\textgreater{}}\ p\textcolor{BrickRed}{.}first\ \textcolor{BrickRed}{\textgreater{}\textgreater{}}\ p\textcolor{BrickRed}{.}second\textcolor{BrickRed}{;} \\
\mbox{}\ \ \ \  \\
\mbox{}\ \ \ \ \textit{\textcolor{Brown}{//\ Resolución\ del\ problema}} \\
\mbox{}\ \ \ \ \textit{\textcolor{Brown}{//\ Recorre\ las\ posibles\ permutaciones\ dejando\ fija\ la\ primera\ ciudad.}} \\
\mbox{}\ \ \ \ \textit{\textcolor{Brown}{//\ Crea\ una\ primera\ ruta\ con\ la\ permutación\ identidad.}} \\
\mbox{}\ \ \ \ \textcolor{TealBlue}{Ruta}\ \textbf{\textcolor{Black}{ruta}}\textcolor{BrickRed}{(}dimension\textcolor{BrickRed}{);} \\
\mbox{}\ \ \ \ \textbf{\textcolor{Black}{iota}}\textcolor{BrickRed}{(}ruta\textcolor{BrickRed}{.}\textbf{\textcolor{Black}{begin}}\textcolor{BrickRed}{(),}ruta\textcolor{BrickRed}{.}\textbf{\textcolor{Black}{end}}\textcolor{BrickRed}{(),}\textcolor{Purple}{0}\textcolor{BrickRed}{);} \\
\mbox{} \\
\mbox{}\ \ \ \ \textbf{\textcolor{Blue}{auto}}\ time1\ \textcolor{BrickRed}{=}\ chrono\textcolor{BrickRed}{::}high$\_$resolution$\_$clock\textcolor{BrickRed}{::}\textbf{\textcolor{Black}{now}}\textcolor{BrickRed}{();} \\
\mbox{}\ \ \ \ \textbf{\textcolor{Black}{permutaciones}}\textcolor{BrickRed}{(}ruta\textcolor{BrickRed}{,}\ coste$\_$actual\textcolor{BrickRed}{,}\ \textcolor{Purple}{1}\textcolor{BrickRed}{);} \\
\mbox{}\ \ \ \ \textbf{\textcolor{Blue}{auto}}\ time2\ \textcolor{BrickRed}{=}\ chrono\textcolor{BrickRed}{::}high$\_$resolution$\_$clock\textcolor{BrickRed}{::}\textbf{\textcolor{Black}{now}}\textcolor{BrickRed}{();} \\
\mbox{}\ \ \ \ chrono\textcolor{BrickRed}{::}\textcolor{TealBlue}{duration\textless{}double\textgreater{}}\ time$\_$span\ \textcolor{BrickRed}{=}\ chrono\textcolor{BrickRed}{::}duration$\_$cast\textcolor{BrickRed}{\textless{}}chrono\textcolor{BrickRed}{::}duration\textcolor{BrickRed}{\textless{}}\textcolor{ForestGreen}{double}\textcolor{BrickRed}{\textgreater{}\textgreater{}(}time2\ \textcolor{BrickRed}{-}\ time1\textcolor{BrickRed}{);} \\
\mbox{}\ \ \ \ \textcolor{ForestGreen}{double}\ time\ \textcolor{BrickRed}{=}\ time$\_$span\textcolor{BrickRed}{.}\textbf{\textcolor{Black}{count}}\textcolor{BrickRed}{();} \\
\mbox{} \\
\mbox{}\ \ \ \ \textit{\textcolor{Brown}{//\ Muestra\ la\ solución}} \\
\mbox{}\ \ \ \ cout\ \textcolor{BrickRed}{\textless{}\textless{}}\ \texttt{\textcolor{Red}{"{}Mejor\ coste\ obtenido:\ "{}}}\ \textcolor{BrickRed}{\textless{}\textless{}}\ mejor$\_$coste\ \textcolor{BrickRed}{\textless{}\textless{}}\ endl \\
\mbox{}\ \ \ \ \ \ \ \ \ \ \ \ \ \ \textcolor{BrickRed}{\textless{}\textless{}}\ \texttt{\textcolor{Red}{"{}Mejor\ ruta:\ "{}}}\ \textcolor{BrickRed}{\textless{}\textless{}}\ endl\ \textcolor{BrickRed}{\textless{}\textless{}}\ mejor$\_$ruta \\
\mbox{}\ \ \ \ \ \ \ \ \ \ \ \ \ \ \textcolor{BrickRed}{\textless{}\textless{}}\ \texttt{\textcolor{Red}{"{}Tiempo\ de\ cómputo:\ "{}}}\ \textcolor{BrickRed}{\textless{}\textless{}}\ time\ \textcolor{BrickRed}{\textless{}\textless{}}\ endl\textcolor{BrickRed}{;} \\
\mbox{}\ \ \ \  \\
\mbox{}\ \ \ \  \\
\mbox{}\ \ \ \ \textit{\textcolor{Brown}{//\ Depuración}} \\
\mbox{}\ \ \ \ \textit{\textcolor{Brown}{//cout\ \textless{}\textless{}\ "{}Recalculado:\ "{}\ \textless{}\textless{}\ total(mejor$\_$ruta);}} \\
\mbox{}\textcolor{Red}{\}} \\
\mbox{}
}
        \normalsize
  \subsection{Planificación en multiprocesadores}
        \small
  	\texttt{% Generator: GNU source-highlight, by Lorenzo Bettini, http://www.gnu.org/software/src-highlite
\noindent
\mbox{}\textit{\textcolor{Brown}{/**}} \\
\mbox{}\textit{\textcolor{Brown}{\ *\ Planificacion.cpp}} \\
\mbox{}\textit{\textcolor{Brown}{\ *\ Problema\ de\ planificación\ en\ multiprocesadores.}} \\
\mbox{}\textit{\textcolor{Brown}{\ *\ Implementación\ de\ un\ algoritmo\ backtracking\ en\ C++.}} \\
\mbox{}\textit{\textcolor{Brown}{\ */}} \\
\mbox{}\textbf{\textcolor{RoyalBlue}{\#include}}\ \texttt{\textcolor{Red}{\textless{}vector\textgreater{}}} \\
\mbox{}\textbf{\textcolor{RoyalBlue}{\#include}}\ \texttt{\textcolor{Red}{\textless{}queue\textgreater{}}} \\
\mbox{}\textbf{\textcolor{RoyalBlue}{\#include}}\ \texttt{\textcolor{Red}{\textless{}algorithm\textgreater{}}} \\
\mbox{}\textbf{\textcolor{RoyalBlue}{\#include}}\ \texttt{\textcolor{Red}{\textless{}utility\textgreater{}}} \\
\mbox{}\textbf{\textcolor{RoyalBlue}{\#include}}\ \texttt{\textcolor{Red}{\textless{}numeric\textgreater{}}} \\
\mbox{}\textbf{\textcolor{RoyalBlue}{\#include}}\ \texttt{\textcolor{Red}{\textless{}iostream\textgreater{}}} \\
\mbox{}\textbf{\textcolor{Blue}{using}}\ \textbf{\textcolor{Blue}{namespace}}\ std\textcolor{BrickRed}{;} \\
\mbox{} \\
\mbox{}\textbf{\textcolor{Blue}{typedef}}\ \textcolor{ForestGreen}{double}\ tiempo\textcolor{BrickRed}{;} \\
\mbox{}\textbf{\textcolor{Blue}{typedef}}\ \textcolor{ForestGreen}{unsigned}\ \textcolor{ForestGreen}{int}\ uint\textcolor{BrickRed}{;} \\
\mbox{} \\
\mbox{}\textbf{\textcolor{Blue}{class}}\ \textcolor{TealBlue}{Planificador}\ \textcolor{Red}{\{} \\
\mbox{}\textbf{\textcolor{Blue}{public}}\textcolor{BrickRed}{:} \\
\mbox{}\ \ \ \ \textbf{\textcolor{Blue}{struct}}\ \textcolor{TealBlue}{Tarea}\textcolor{BrickRed}{;} \\
\mbox{}\ \ \ \ \textbf{\textcolor{Blue}{struct}}\ \textcolor{TealBlue}{Asignacion}\textcolor{BrickRed}{;} \\
\mbox{}\ \ \ \ \textbf{\textcolor{Blue}{struct}}\ \textcolor{TealBlue}{Planificacion}\textcolor{BrickRed}{;} \\
\mbox{}\ \ \ \ \textbf{\textcolor{Blue}{static}}\ \textbf{\textcolor{Blue}{const}}\ \textcolor{ForestGreen}{int}\ num$\_$cores\ \textcolor{BrickRed}{=}\ \textcolor{Purple}{4}\textcolor{BrickRed}{;} \\
\mbox{}\ \ \ \ \textbf{\textcolor{Blue}{const}}\ \textcolor{TealBlue}{vector\textless{}Tarea\textgreater{}}\ problema\textcolor{BrickRed}{;} \\
\mbox{}\textbf{\textcolor{Blue}{private}}\textcolor{BrickRed}{:} \\
\mbox{}\ \ \ \ \textcolor{ForestGreen}{bool}\ \textbf{\textcolor{Black}{depende}}\textcolor{BrickRed}{(}\textcolor{ForestGreen}{int}\ una\textcolor{BrickRed}{,}\ \textcolor{ForestGreen}{int}\ otra\textcolor{BrickRed}{);} \\
\mbox{}\ \ \ \ \textcolor{ForestGreen}{bool}\ \textbf{\textcolor{Black}{empty}}\textcolor{BrickRed}{(}\textbf{\textcolor{Blue}{const}}\ \textcolor{TealBlue}{vector\textless{}Tarea\textgreater{}}\ \textcolor{BrickRed}{\&}procesador\textcolor{BrickRed}{);} \\
\mbox{}\ \ \ \ \textcolor{TealBlue}{uint}\ \textbf{\textcolor{Black}{gap}}\textcolor{BrickRed}{(}\textcolor{TealBlue}{vector\textless{}Tarea\textgreater{}}\ \textcolor{BrickRed}{\&}procesador\textcolor{BrickRed}{);} \\
\mbox{}\textbf{\textcolor{Blue}{public}}\textcolor{BrickRed}{:} \\
\mbox{}\ \ \ \ \textcolor{TealBlue}{Planificacion}\ \textbf{\textcolor{Black}{planifica}}\textcolor{BrickRed}{();} \\
\mbox{}\ \ \ \ \textbf{\textcolor{Black}{Planificador}}\textcolor{BrickRed}{(}\textcolor{TealBlue}{vector\textless{}Tarea\textgreater{}}\ tareas\textcolor{BrickRed}{)}\ \textcolor{BrickRed}{:}\textbf{\textcolor{Black}{problema}}\textcolor{BrickRed}{(}tareas\textcolor{BrickRed}{)}\ \textcolor{Red}{\{\}} \\
\mbox{}\textcolor{Red}{\}}\textcolor{BrickRed}{;} \\
\mbox{} \\
\mbox{}\textbf{\textcolor{Blue}{struct}}\ \textcolor{TealBlue}{Planificador}\textcolor{BrickRed}{::}Tarea\ \textcolor{Red}{\{} \\
\mbox{}\ \ \ \ \textcolor{TealBlue}{uint}\ index\textcolor{BrickRed}{;} \\
\mbox{}\ \ \ \ \textcolor{TealBlue}{tiempo}\ ejecucion\textcolor{BrickRed}{;} \\
\mbox{}\ \ \ \ \textit{\textcolor{Brown}{//\ Dependencias\ de\ otras\ tareas}} \\
\mbox{}\ \ \ \ \textcolor{TealBlue}{vector\textless{}int\textgreater{}}\ dependencias\textcolor{BrickRed}{;} \\
\mbox{} \\
\mbox{}\ \ \ \ \textbf{\textcolor{Black}{Tarea}}\textcolor{BrickRed}{(}\textcolor{TealBlue}{uint}\ i\textcolor{BrickRed}{,}\ \textcolor{TealBlue}{tiempo}\ t\textcolor{BrickRed}{,}\ \textcolor{TealBlue}{vector\textless{}int\textgreater{}}\ deps\textcolor{BrickRed}{)} \\
\mbox{}\ \ \ \ \ \ \ \ \textcolor{BrickRed}{:}\textbf{\textcolor{Black}{index}}\textcolor{BrickRed}{(}i\textcolor{BrickRed}{),}\ \textbf{\textcolor{Black}{ejecucion}}\textcolor{BrickRed}{(}t\textcolor{BrickRed}{),}\ \textbf{\textcolor{Black}{dependencias}}\textcolor{BrickRed}{(}deps\textcolor{BrickRed}{)}\ \textcolor{Red}{\{\}} \\
\mbox{}\ \ \ \ \textbf{\textcolor{Black}{Tarea}}\textcolor{BrickRed}{(}\textcolor{TealBlue}{tiempo}\ t\textcolor{BrickRed}{)}\ \textcolor{BrickRed}{:}\textbf{\textcolor{Black}{ejecucion}}\textcolor{BrickRed}{(}t\textcolor{BrickRed}{)}\ \textcolor{Red}{\{\}} \\
\mbox{}\ \ \ \ \textbf{\textcolor{Black}{Tarea}}\textcolor{BrickRed}{()}\ \textcolor{BrickRed}{:}\textbf{\textcolor{Black}{ejecucion}}\textcolor{BrickRed}{(}\textcolor{Purple}{0}\textcolor{BrickRed}{)}\ \textcolor{Red}{\{\}} \\
\mbox{} \\
\mbox{}\ \ \ \ \textcolor{ForestGreen}{bool}\ \textbf{\textcolor{Black}{empty}}\textcolor{BrickRed}{()}\textcolor{Red}{\{} \\
\mbox{}\ \ \ \ \ \ \ \ \textbf{\textcolor{Blue}{return}}\ ejecucion\ \textcolor{BrickRed}{==}\ \textcolor{Purple}{0}\textcolor{BrickRed}{;} \\
\mbox{}\ \ \ \ \textcolor{Red}{\}} \\
\mbox{}\textcolor{Red}{\}}\textcolor{BrickRed}{;} \\
\mbox{} \\
\mbox{}\textbf{\textcolor{Blue}{struct}}\ \textcolor{TealBlue}{Planificador}\textcolor{BrickRed}{::}Asignacion\textcolor{Red}{\{} \\
\mbox{}\ \ \ \ \textcolor{TealBlue}{uint}\ core\textcolor{BrickRed}{;} \\
\mbox{}\ \ \ \ \textcolor{TealBlue}{Tarea}\ tarea\textcolor{BrickRed}{;} \\
\mbox{}\ \ \ \ \textcolor{TealBlue}{tiempo}\ t$\_$inicio\textcolor{BrickRed}{;} \\
\mbox{} \\
\mbox{}\ \ \ \ \textbf{\textcolor{Black}{Asignacion}}\textcolor{BrickRed}{(}\textcolor{TealBlue}{uint}\ i\textcolor{BrickRed}{,}\ \textcolor{TealBlue}{Tarea}\ \textcolor{BrickRed}{\&}t\textcolor{BrickRed}{,}\ \textcolor{TealBlue}{tiempo}\ init\textcolor{BrickRed}{)} \\
\mbox{}\ \ \ \ \ \ \ \textcolor{BrickRed}{:}\textbf{\textcolor{Black}{core}}\textcolor{BrickRed}{(}i\textcolor{BrickRed}{),}\ \textbf{\textcolor{Black}{tarea}}\textcolor{BrickRed}{(}t\textcolor{BrickRed}{),}\ \textbf{\textcolor{Black}{t$\_$inicio}}\textcolor{BrickRed}{(}init\textcolor{BrickRed}{)} \\
\mbox{}\ \ \ \ \textcolor{Red}{\{\}} \\
\mbox{}\textcolor{Red}{\}}\textcolor{BrickRed}{;} \\
\mbox{} \\
\mbox{}\textbf{\textcolor{Blue}{struct}}\ \textcolor{TealBlue}{Planificador}\textcolor{BrickRed}{::}Planificacion\textcolor{Red}{\{} \\
\mbox{}\ \ \ \ \textit{\textcolor{Brown}{//\ Asignaciones\ de\ tareas\ a\ cores\ en\ orden}} \\
\mbox{}\ \ \ \ vector\ \textcolor{BrickRed}{\textless{}}Asignacion\textcolor{BrickRed}{\textgreater{}}\ historial\textcolor{BrickRed}{;} \\
\mbox{}\ \ \ \ \textit{\textcolor{Brown}{//\ Estado\ del\ procesador\ en\ un\ momento\ determinado,\ estará}} \\
\mbox{}\ \ \ \ \textit{\textcolor{Brown}{//\ lleno\ de\ Tareas\ vacías\ cuando\ se\ haya\ terminado\ la\ planificación}} \\
\mbox{}\ \ \ \ vector\ \textcolor{BrickRed}{\textless{}}Tarea\textcolor{BrickRed}{\textgreater{}}\ procesador$\_$actual\textcolor{BrickRed}{;} \\
\mbox{}\ \ \ \ \textit{\textcolor{Brown}{//\ Tareas\ que\ faltan\ por\ planificar}} \\
\mbox{}\ \ \ \ vector\ \textcolor{BrickRed}{\textless{}}Tarea\textcolor{BrickRed}{\textgreater{}}\ restantes\textcolor{BrickRed}{;} \\
\mbox{}\ \ \ \ \textcolor{TealBlue}{tiempo}\ t$\_$ejecucion\textcolor{BrickRed}{;} \\
\mbox{} \\
\mbox{}\ \ \ \ \textbf{\textcolor{Black}{Planificacion}}\textcolor{BrickRed}{()}\textcolor{Red}{\{\}} \\
\mbox{} \\
\mbox{}\ \ \ \ \textbf{\textcolor{Black}{Planificacion}}\textcolor{BrickRed}{(}vector\ \textcolor{BrickRed}{\textless{}}Tarea\textcolor{BrickRed}{\textgreater{}}\ tareas\textcolor{BrickRed}{)} \\
\mbox{}\ \ \ \ \ \ \ \ \textcolor{BrickRed}{:}\textbf{\textcolor{Black}{restantes}}\textcolor{BrickRed}{(}tareas\textcolor{BrickRed}{),} \\
\mbox{}\ \ \ \ \ \ \ \ \textbf{\textcolor{Black}{t$\_$ejecucion}}\textcolor{BrickRed}{(}\textcolor{Purple}{0}\textcolor{BrickRed}{)} \\
\mbox{}\ \ \ \ \textcolor{Red}{\{} \\
\mbox{}\ \ \ \ \ \ \ \ procesador$\_$actual\textcolor{BrickRed}{.}\textbf{\textcolor{Black}{resize}}\textcolor{BrickRed}{(}num$\_$cores\textcolor{BrickRed}{);} \\
\mbox{}\ \ \ \ \textcolor{Red}{\}} \\
\mbox{}\textcolor{Red}{\}}\textcolor{BrickRed}{;} \\
\mbox{} \\
\mbox{} \\
\mbox{}\textit{\textcolor{Brown}{//\ Decimos\ que\ la\ tarea\ 'otra'\ depende\ de\ 'una'\ si\ desciende\ directamente}} \\
\mbox{}\textit{\textcolor{Brown}{//\ de\ ella,\ o\ aguna\ de\ sus\ dependencias\ depende\ de\ 'una'}} \\
\mbox{}\textcolor{ForestGreen}{bool}\ Planificador\textcolor{BrickRed}{::}\textbf{\textcolor{Black}{depende}}\textcolor{BrickRed}{(}\textcolor{ForestGreen}{int}\ una\textcolor{BrickRed}{,}\ \textcolor{ForestGreen}{int}\ otra\textcolor{BrickRed}{)}\ \textcolor{Red}{\{} \\
\mbox{}\ \ \ \ \textbf{\textcolor{Blue}{if}}\ \textcolor{BrickRed}{(}\textbf{\textcolor{Black}{find}}\textcolor{BrickRed}{(}problema\textcolor{BrickRed}{[}otra\textcolor{BrickRed}{].}dependencias\textcolor{BrickRed}{.}\textbf{\textcolor{Black}{begin}}\textcolor{BrickRed}{(),}\ problema\textcolor{BrickRed}{[}otra\textcolor{BrickRed}{].}dependencias\textcolor{BrickRed}{.}\textbf{\textcolor{Black}{end}}\textcolor{BrickRed}{(),}\ una\textcolor{BrickRed}{)} \\
\mbox{}\ \ \ \ \ \ \ \ \textcolor{BrickRed}{!=}\ problema\textcolor{BrickRed}{[}otra\textcolor{BrickRed}{].}dependencias\textcolor{BrickRed}{.}\textbf{\textcolor{Black}{end}}\textcolor{BrickRed}{())}\textcolor{Red}{\{} \\
\mbox{}\ \ \ \ \ \ \ \ \textbf{\textcolor{Blue}{return}}\ \textbf{\textcolor{Blue}{true}}\textcolor{BrickRed}{;} \\
\mbox{}\ \ \ \ \textcolor{Red}{\}} \\
\mbox{}\ \ \ \ \textbf{\textcolor{Blue}{else}}\textcolor{Red}{\{} \\
\mbox{}\ \ \ \ \ \ \ \ \textit{\textcolor{Brown}{//\ Subimos\ un\ nivel}} \\
\mbox{}\ \ \ \ \ \ \ \ \textbf{\textcolor{Blue}{for}}\ \textcolor{BrickRed}{(}\textbf{\textcolor{Blue}{auto}}\ super\ \textcolor{BrickRed}{:}\ problema\textcolor{BrickRed}{[}otra\textcolor{BrickRed}{].}dependencias\textcolor{BrickRed}{)}\textcolor{Red}{\{} \\
\mbox{}\ \ \ \ \ \ \ \ \ \ \ \ \textbf{\textcolor{Blue}{if}}\ \textcolor{BrickRed}{(}\textbf{\textcolor{Black}{depende}}\textcolor{BrickRed}{(}una\textcolor{BrickRed}{,}\ super\textcolor{BrickRed}{))} \\
\mbox{}\ \ \ \ \ \ \ \ \ \ \ \ \ \ \ \ \textbf{\textcolor{Blue}{return}}\ \textbf{\textcolor{Blue}{true}}\textcolor{BrickRed}{;} \\
\mbox{}\ \ \ \ \ \ \ \ \textcolor{Red}{\}} \\
\mbox{}\ \ \ \ \textcolor{Red}{\}} \\
\mbox{}\ \ \ \ \textbf{\textcolor{Blue}{return}}\ \textbf{\textcolor{Blue}{false}}\textcolor{BrickRed}{;} \\
\mbox{}\textcolor{Red}{\}} \\
\mbox{} \\
\mbox{}\textcolor{ForestGreen}{bool}\ Planificador\textcolor{BrickRed}{::}\textbf{\textcolor{Black}{empty}}\textcolor{BrickRed}{(}\textbf{\textcolor{Blue}{const}}\ \textcolor{TealBlue}{vector\textless{}Tarea\textgreater{}}\ \textcolor{BrickRed}{\&}procesador\textcolor{BrickRed}{)}\textcolor{Red}{\{} \\
\mbox{}\ \ \ \ \textbf{\textcolor{Blue}{for}}\ \textcolor{BrickRed}{(}\textbf{\textcolor{Blue}{auto}}\ t\ \textcolor{BrickRed}{:}\ procesador\textcolor{BrickRed}{)}\textcolor{Red}{\{} \\
\mbox{}\ \ \ \ \ \ \ \ \textbf{\textcolor{Blue}{if}}\ \textcolor{BrickRed}{(!}t\textcolor{BrickRed}{.}\textbf{\textcolor{Black}{empty}}\textcolor{BrickRed}{())}\textcolor{Red}{\{} \\
\mbox{}\ \ \ \ \ \ \ \ \ \ \ \ \textbf{\textcolor{Blue}{return}}\ \textbf{\textcolor{Blue}{false}}\textcolor{BrickRed}{;} \\
\mbox{}\ \ \ \ \ \ \ \ \textcolor{Red}{\}} \\
\mbox{}\ \ \ \ \textcolor{Red}{\}} \\
\mbox{}\ \ \ \ \textbf{\textcolor{Blue}{return}}\ \textbf{\textcolor{Blue}{true}}\textcolor{BrickRed}{;} \\
\mbox{}\textcolor{Red}{\}} \\
\mbox{} \\
\mbox{}\textcolor{TealBlue}{uint}\ Planificador\textcolor{BrickRed}{::}\textbf{\textcolor{Black}{gap}}\textcolor{BrickRed}{(}\textcolor{TealBlue}{vector\textless{}Tarea\textgreater{}}\ \textcolor{BrickRed}{\&}procesador\textcolor{BrickRed}{)}\textcolor{Red}{\{} \\
\mbox{}\ \ \ \ \textbf{\textcolor{Blue}{for}}\ \textcolor{BrickRed}{(}\textcolor{TealBlue}{uint}\ i\textcolor{BrickRed}{=}\textcolor{Purple}{0}\textcolor{BrickRed}{;}\ i\textcolor{BrickRed}{\textless{}}num$\_$cores\textcolor{BrickRed}{;}\ \textcolor{BrickRed}{++}i\textcolor{BrickRed}{)}\textcolor{Red}{\{} \\
\mbox{}\ \ \ \ \ \ \ \ \textbf{\textcolor{Blue}{if}}\ \textcolor{BrickRed}{(}procesador\textcolor{BrickRed}{[}i\textcolor{BrickRed}{].}\textbf{\textcolor{Black}{empty}}\textcolor{BrickRed}{())}\textcolor{Red}{\{} \\
\mbox{}\ \ \ \ \ \ \ \ \ \ \ \ \textbf{\textcolor{Blue}{return}}\ i\textcolor{BrickRed}{+}\textcolor{Purple}{1}\textcolor{BrickRed}{;} \\
\mbox{}\ \ \ \ \ \ \ \ \textcolor{Red}{\}} \\
\mbox{}\ \ \ \ \textcolor{Red}{\}} \\
\mbox{}\ \ \ \ \textbf{\textcolor{Blue}{return}}\ \textcolor{Purple}{0}\textcolor{BrickRed}{;} \\
\mbox{}\textcolor{Red}{\}} \\
\mbox{} \\
\mbox{}Planificador\textcolor{BrickRed}{::}\textcolor{TealBlue}{Planificacion}\ Planificador\textcolor{BrickRed}{::}\textbf{\textcolor{Black}{planifica}}\textcolor{BrickRed}{()}\ \textcolor{Red}{\{} \\
\mbox{}\ \ \ \ \textcolor{TealBlue}{queue\textless{}Planificacion\textgreater{}}\ posibles\textcolor{BrickRed}{;} \\
\mbox{}\ \ \ \ \textcolor{TealBlue}{Planificacion}\ solucion\textcolor{BrickRed}{;} \\
\mbox{}\ \ \ \ solucion\textcolor{BrickRed}{.}t$\_$ejecucion\ \textcolor{BrickRed}{=}\ numeric$\_$limits\textcolor{BrickRed}{\textless{}}tiempo\textcolor{BrickRed}{\textgreater{}::}\textbf{\textcolor{Black}{infinity}}\textcolor{BrickRed}{();} \\
\mbox{} \\
\mbox{}\ \ \ \ posibles\textcolor{BrickRed}{.}\textbf{\textcolor{Black}{push}}\textcolor{BrickRed}{(}\textbf{\textcolor{Black}{Planificacion}}\textcolor{BrickRed}{(}problema\textcolor{BrickRed}{));} \\
\mbox{} \\
\mbox{} \\
\mbox{}\ \ \ \ \textbf{\textcolor{Blue}{while}}\ \textcolor{BrickRed}{(!}posibles\textcolor{BrickRed}{.}\textbf{\textcolor{Black}{empty}}\textcolor{BrickRed}{())}\ \textcolor{Red}{\{} \\
\mbox{}\ \ \ \ \ \ \ \ \textcolor{TealBlue}{Planificacion}\ actual\ \textcolor{BrickRed}{=}\ posibles\textcolor{BrickRed}{.}\textbf{\textcolor{Black}{front}}\textcolor{BrickRed}{();} \\
\mbox{}\ \ \ \ \ \ \ \ posibles\textcolor{BrickRed}{.}\textbf{\textcolor{Black}{pop}}\textcolor{BrickRed}{();} \\
\mbox{} \\
\mbox{}\ \ \ \ \ \ \ \ \textbf{\textcolor{Blue}{if}}\ \textcolor{BrickRed}{(}actual\textcolor{BrickRed}{.}historial\textcolor{BrickRed}{.}\textbf{\textcolor{Black}{size}}\textcolor{BrickRed}{()}\ \textcolor{BrickRed}{==}\ problema\textcolor{BrickRed}{.}\textbf{\textcolor{Black}{size}}\textcolor{BrickRed}{()}\ \textcolor{BrickRed}{\&\&}\ \textbf{\textcolor{Black}{empty}}\textcolor{BrickRed}{(}actual\textcolor{BrickRed}{.}procesador$\_$actual\textcolor{BrickRed}{))}\ \textcolor{Red}{\{} \\
\mbox{}\ \ \ \ \ \ \ \ \ \ \ \ \textbf{\textcolor{Blue}{if}}\ \textcolor{BrickRed}{(}actual\textcolor{BrickRed}{.}t$\_$ejecucion\ \textcolor{BrickRed}{\textless{}}\ solucion\textcolor{BrickRed}{.}t$\_$ejecucion\textcolor{BrickRed}{)}\ \textcolor{Red}{\{} \\
\mbox{}\ \ \ \ \ \ \ \ \ \ \ \ \ \ \ \ solucion\ \textcolor{BrickRed}{=}\ actual\textcolor{BrickRed}{;} \\
\mbox{}\ \ \ \ \ \ \ \ \ \ \ \ \textcolor{Red}{\}} \\
\mbox{}\ \ \ \ \ \ \ \ \textcolor{Red}{\}} \\
\mbox{}\ \ \ \ \ \ \ \ \textbf{\textcolor{Blue}{else}}\ \textcolor{Red}{\{} \\
\mbox{}\ \ \ \ \ \ \ \ \ \ \ \ \textcolor{ForestGreen}{bool}\ dependencia\textcolor{BrickRed}{;} \\
\mbox{}\ \ \ \ \ \ \ \ \ \ \ \ \textcolor{TealBlue}{uint}\ core\ \textcolor{BrickRed}{=}\ \textbf{\textcolor{Black}{gap}}\textcolor{BrickRed}{(}actual\textcolor{BrickRed}{.}procesador$\_$actual\textcolor{BrickRed}{);} \\
\mbox{}\ \ \ \ \ \ \ \ \ \ \ \ \textit{\textcolor{Brown}{/*}} \\
\mbox{}\textit{\textcolor{Brown}{\ \ \ \ \ \ \ \ \ \ \ \ \ Si\ hay\ core\ libre,\ intentamos\ planificar\ algún\ proceso\ en\ dicho\ core}} \\
\mbox{}\textit{\textcolor{Brown}{\ \ \ \ \ \ \ \ \ \ \ \ \ */}} \\
\mbox{} \\
\mbox{} \\
\mbox{}\ \ \ \ \ \ \ \ \ \ \ \ \textbf{\textcolor{Blue}{if}}\ \textcolor{BrickRed}{(}core\textcolor{BrickRed}{)}\textcolor{Red}{\{} \\
\mbox{}\ \ \ \ \ \ \ \ \ \ \ \ \ \ \ \ core\textcolor{BrickRed}{-\/-;} \\
\mbox{}\ \ \ \ \ \ \ \ \ \ \ \ \ \ \ \ \textbf{\textcolor{Blue}{for}}\ \textcolor{BrickRed}{(}\textcolor{TealBlue}{uint}\ j\textcolor{BrickRed}{=}\textcolor{Purple}{0}\textcolor{BrickRed}{;}\ j\textcolor{BrickRed}{\textless{}}actual\textcolor{BrickRed}{.}restantes\textcolor{BrickRed}{.}\textbf{\textcolor{Black}{size}}\textcolor{BrickRed}{();}\ \textcolor{BrickRed}{++}j\textcolor{BrickRed}{)}\textcolor{Red}{\{} \\
\mbox{}\ \ \ \ \ \ \ \ \ \ \ \ \ \ \ \ \ \ \ \ dependencia\ \textcolor{BrickRed}{=}\ \textbf{\textcolor{Blue}{false}}\textcolor{BrickRed}{;} \\
\mbox{}\ \ \ \ \ \ \ \ \ \ \ \ \ \ \ \ \ \ \ \ \textbf{\textcolor{Blue}{for}}\ \textcolor{BrickRed}{(}\textcolor{TealBlue}{uint}\ i\textcolor{BrickRed}{=}\textcolor{Purple}{0}\textcolor{BrickRed}{;}\ i\textcolor{BrickRed}{\textless{}}num$\_$cores\ \textcolor{BrickRed}{\&\&}\ \textcolor{BrickRed}{!}dependencia\textcolor{BrickRed}{;}\ \textcolor{BrickRed}{++}i\textcolor{BrickRed}{)}\textcolor{Red}{\{} \\
\mbox{} \\
\mbox{}\ \ \ \ \ \ \ \ \ \ \ \ \ \ \ \ \ \ \ \ \ \ \ \ \textbf{\textcolor{Blue}{if}}\ \textcolor{BrickRed}{(!}actual\textcolor{BrickRed}{.}procesador$\_$actual\textcolor{BrickRed}{[}i\textcolor{BrickRed}{].}\textbf{\textcolor{Black}{empty}}\textcolor{BrickRed}{())}\textcolor{Red}{\{} \\
\mbox{}\ \ \ \ \ \ \ \ \ \ \ \ \ \ \ \ \ \ \ \ \ \ \ \ \ \ \ \ dependencia\ \textcolor{BrickRed}{=}\ \textbf{\textcolor{Black}{depende}}\textcolor{BrickRed}{(}actual\textcolor{BrickRed}{.}procesador$\_$actual\textcolor{BrickRed}{[}i\textcolor{BrickRed}{].}index\textcolor{BrickRed}{,}\ actual\textcolor{BrickRed}{.}restantes\textcolor{BrickRed}{[}j\textcolor{BrickRed}{].}index\textcolor{BrickRed}{);} \\
\mbox{}\ \ \ \ \ \ \ \ \ \ \ \ \ \ \ \ \ \ \ \ \ \ \ \ \textcolor{Red}{\}} \\
\mbox{}\ \ \ \ \ \ \ \ \ \ \ \ \ \ \ \ \ \ \ \ \textcolor{Red}{\}} \\
\mbox{}\ \ \ \ \ \ \ \ \ \ \ \ \ \ \ \ \ \ \ \ \textbf{\textcolor{Blue}{for}}\ \textcolor{BrickRed}{(}\textcolor{TealBlue}{uint}\ k\textcolor{BrickRed}{=}\textcolor{Purple}{0}\textcolor{BrickRed}{;}\ k\textcolor{BrickRed}{\textless{}}actual\textcolor{BrickRed}{.}restantes\textcolor{BrickRed}{.}\textbf{\textcolor{Black}{size}}\textcolor{BrickRed}{()}\ \textcolor{BrickRed}{\&\&}\ \textcolor{BrickRed}{!}dependencia\textcolor{BrickRed}{;}\ \textcolor{BrickRed}{++}k\textcolor{BrickRed}{)}\textcolor{Red}{\{} \\
\mbox{}\ \ \ \ \ \ \ \ \ \ \ \ \ \ \ \ \ \ \ \ \ \ \ \ \textbf{\textcolor{Blue}{if}}\ \textcolor{BrickRed}{(}k\textcolor{BrickRed}{!=}j\textcolor{BrickRed}{)}\textcolor{Red}{\{} \\
\mbox{}\ \ \ \ \ \ \ \ \ \ \ \ \ \ \ \ \ \ \ \ \ \ \ \ \ \ \ \ \textbf{\textcolor{Blue}{if}}\ \textcolor{BrickRed}{(!}actual\textcolor{BrickRed}{.}restantes\textcolor{BrickRed}{[}j\textcolor{BrickRed}{].}\textbf{\textcolor{Black}{empty}}\textcolor{BrickRed}{())} \\
\mbox{}\ \ \ \ \ \ \ \ \ \ \ \ \ \ \ \ \ \ \ \ \ \ \ \ \ \ \ \ \ \ \ \ dependencia\ \textcolor{BrickRed}{=}\ \textbf{\textcolor{Black}{depende}}\textcolor{BrickRed}{(}actual\textcolor{BrickRed}{.}restantes\textcolor{BrickRed}{[}k\textcolor{BrickRed}{].}index\textcolor{BrickRed}{,}\ actual\textcolor{BrickRed}{.}restantes\textcolor{BrickRed}{[}j\textcolor{BrickRed}{].}index\textcolor{BrickRed}{);} \\
\mbox{}\ \ \ \ \ \ \ \ \ \ \ \ \ \ \ \ \ \ \ \ \ \ \ \ \textcolor{Red}{\}} \\
\mbox{}\ \ \ \ \ \ \ \ \ \ \ \ \ \ \ \ \ \ \ \ \textcolor{Red}{\}} \\
\mbox{} \\
\mbox{}\ \ \ \ \ \ \ \ \ \ \ \ \ \ \ \ \ \ \ \ \textbf{\textcolor{Blue}{if}}\ \textcolor{BrickRed}{(!}dependencia\textcolor{BrickRed}{)}\textcolor{Red}{\{} \\
\mbox{}\ \ \ \ \ \ \ \ \ \ \ \ \ \ \ \ \ \ \ \ \ \ \ \ \textcolor{TealBlue}{Planificacion}\ copia$\_$actual\ \textcolor{BrickRed}{=}\ actual\textcolor{BrickRed}{;} \\
\mbox{}\ \ \ \ \ \ \ \ \ \ \ \ \ \ \ \ \ \ \ \ \ \ \ \ copia$\_$actual\textcolor{BrickRed}{.}procesador$\_$actual\textcolor{BrickRed}{[}core\textcolor{BrickRed}{]}\ \textcolor{BrickRed}{=}\ actual\textcolor{BrickRed}{.}restantes\textcolor{BrickRed}{[}j\textcolor{BrickRed}{];} \\
\mbox{}\ \ \ \ \ \ \ \ \ \ \ \ \ \ \ \ \ \ \ \ \ \ \ \ vector\ \textcolor{BrickRed}{\textless{}}Tarea\textcolor{BrickRed}{\textgreater{}::}\textcolor{TealBlue}{iterator}\ it\ \textcolor{BrickRed}{=}\ copia$\_$actual\textcolor{BrickRed}{.}restantes\textcolor{BrickRed}{.}\textbf{\textcolor{Black}{begin}}\textcolor{BrickRed}{();} \\
\mbox{}\ \ \ \ \ \ \ \ \ \ \ \ \ \ \ \ \ \ \ \ \ \ \ \ \textbf{\textcolor{Black}{advance}}\ \textcolor{BrickRed}{(}it\textcolor{BrickRed}{,}j\textcolor{BrickRed}{);} \\
\mbox{}\ \ \ \ \ \ \ \ \ \ \ \ \ \ \ \ \ \ \ \ \ \ \ \ copia$\_$actual\textcolor{BrickRed}{.}restantes\textcolor{BrickRed}{.}\textbf{\textcolor{Black}{erase}}\textcolor{BrickRed}{(}it\textcolor{BrickRed}{);} \\
\mbox{}\ \ \ \ \ \ \ \ \ \ \ \ \ \ \ \ \ \ \ \ \ \ \ \ copia$\_$actual\textcolor{BrickRed}{.}historial\textcolor{BrickRed}{.}push$\_$back \\
\mbox{}\ \ \ \ \ \ \ \ \ \ \ \ \ \ \ \ \ \ \ \ \ \ \ \ \ \ \ \ \textcolor{BrickRed}{(}\textbf{\textcolor{Black}{Asignacion}}\textcolor{BrickRed}{(}core\textcolor{BrickRed}{,}\ actual\textcolor{BrickRed}{.}restantes\textcolor{BrickRed}{[}j\textcolor{BrickRed}{],}\ copia$\_$actual\textcolor{BrickRed}{.}t$\_$ejecucion\textcolor{BrickRed}{));} \\
\mbox{}\ \ \ \ \ \ \ \ \ \ \ \ \ \ \ \ \ \ \ \ \ \ \ \ posibles\textcolor{BrickRed}{.}\textbf{\textcolor{Black}{push}}\textcolor{BrickRed}{(}copia$\_$actual\textcolor{BrickRed}{);} \\
\mbox{}\ \ \ \ \ \ \ \ \ \ \ \ \ \ \ \ \ \ \ \ \textcolor{Red}{\}} \\
\mbox{}\ \ \ \ \ \ \ \ \ \ \ \ \ \ \ \ \textcolor{Red}{\}} \\
\mbox{}\ \ \ \ \ \ \ \ \ \ \ \ \textcolor{Red}{\}} \\
\mbox{}\ \ \ \ \ \ \ \ \ \ \ \ \textit{\textcolor{Brown}{//\ Si\ el\ procesador\ no\ estaba\ vacío}} \\
\mbox{}\ \ \ \ \ \ \ \ \ \ \ \ \textbf{\textcolor{Blue}{if}}\ \textcolor{BrickRed}{(!}\textbf{\textcolor{Black}{empty}}\textcolor{BrickRed}{(}actual\textcolor{BrickRed}{.}procesador$\_$actual\textcolor{BrickRed}{))}\textcolor{Red}{\{} \\
\mbox{}\ \ \ \ \ \ \ \ \ \ \ \ \ \ \ \ \textcolor{TealBlue}{tiempo}\ minimo\ \textcolor{BrickRed}{=}\ numeric$\_$limits\textcolor{BrickRed}{\textless{}}tiempo\textcolor{BrickRed}{\textgreater{}::}\textbf{\textcolor{Black}{infinity}}\textcolor{BrickRed}{();} \\
\mbox{}\ \ \ \ \ \ \ \ \ \ \ \ \ \ \ \ \textit{\textcolor{Brown}{//\ Buscamos\ la\ tarea\ en\ el\ procesador\ de\ menor\ tiempo\ de\ ejecución\ restante}} \\
\mbox{}\ \ \ \ \ \ \ \ \ \ \ \ \ \ \ \ \textbf{\textcolor{Blue}{for}}\ \textcolor{BrickRed}{(}\textbf{\textcolor{Blue}{auto}}\ \textcolor{BrickRed}{\&}tarea\ \textcolor{BrickRed}{:}\ actual\textcolor{BrickRed}{.}procesador$\_$actual\textcolor{BrickRed}{)}\textcolor{Red}{\{} \\
\mbox{}\ \ \ \ \ \ \ \ \ \ \ \ \ \ \ \ \ \ \ \ \textbf{\textcolor{Blue}{if}}\ \textcolor{BrickRed}{(!}tarea\textcolor{BrickRed}{.}\textbf{\textcolor{Black}{empty}}\textcolor{BrickRed}{())}\textcolor{Red}{\{} \\
\mbox{}\ \ \ \ \ \ \ \ \ \ \ \ \ \ \ \ \ \ \ \ \ \ \ \ \textbf{\textcolor{Blue}{if}}\ \textcolor{BrickRed}{(}tarea\textcolor{BrickRed}{.}ejecucion\ \textcolor{BrickRed}{\textless{}}\ minimo\textcolor{BrickRed}{)}\textcolor{Red}{\{} \\
\mbox{}\ \ \ \ \ \ \ \ \ \ \ \ \ \ \ \ \ \ \ \ \ \ \ \ \ \ \ \ minimo\ \textcolor{BrickRed}{=}\ tarea\textcolor{BrickRed}{.}ejecucion\textcolor{BrickRed}{;} \\
\mbox{}\ \ \ \ \ \ \ \ \ \ \ \ \ \ \ \ \ \ \ \ \ \ \ \ \textcolor{Red}{\}} \\
\mbox{}\ \ \ \ \ \ \ \ \ \ \ \ \ \ \ \ \ \ \ \ \textcolor{Red}{\}} \\
\mbox{}\ \ \ \ \ \ \ \ \ \ \ \ \ \ \ \ \textcolor{Red}{\}} \\
\mbox{}\ \ \ \ \ \ \ \ \ \ \ \ \ \ \ \ \textit{\textcolor{Brown}{//\ Actualizamos\ tiempos\ de\ ejecución\ del\ procesador}} \\
\mbox{}\ \ \ \ \ \ \ \ \ \ \ \ \ \ \ \ \textbf{\textcolor{Blue}{for}}\ \textcolor{BrickRed}{(}\textcolor{TealBlue}{Tarea}\ \textcolor{BrickRed}{\&}tarea\ \textcolor{BrickRed}{:}\ actual\textcolor{BrickRed}{.}procesador$\_$actual\textcolor{BrickRed}{)}\textcolor{Red}{\{} \\
\mbox{}\ \ \ \ \ \ \ \ \ \ \ \ \ \ \ \ \ \ \ \ tarea\textcolor{BrickRed}{.}ejecucion\ \textcolor{BrickRed}{-=}\ minimo\textcolor{BrickRed}{;} \\
\mbox{} \\
\mbox{}\ \ \ \ \ \ \ \ \ \ \ \ \ \ \ \ \ \ \ \ \textbf{\textcolor{Blue}{if}}\ \textcolor{BrickRed}{(}tarea\textcolor{BrickRed}{.}ejecucion\ \textcolor{BrickRed}{\textless{}}\ \textcolor{Purple}{0}\textcolor{BrickRed}{)} \\
\mbox{}\ \ \ \ \ \ \ \ \ \ \ \ \ \ \ \ \ \ \ \ \ \ \ \ tarea\textcolor{BrickRed}{.}ejecucion\ \textcolor{BrickRed}{=}\ \textcolor{Purple}{0}\textcolor{BrickRed}{;} \\
\mbox{}\ \ \ \ \ \ \ \ \ \ \ \ \ \ \ \ \textcolor{Red}{\}} \\
\mbox{} \\
\mbox{}\ \ \ \ \ \ \ \ \ \ \ \ \ \ \ \ actual\textcolor{BrickRed}{.}t$\_$ejecucion\ \textcolor{BrickRed}{+=}\ minimo\textcolor{BrickRed}{;} \\
\mbox{}\ \ \ \ \ \ \ \ \ \ \ \ \ \ \ \  \\
\mbox{}\textbf{\textcolor{RoyalBlue}{\ \ \ \ \ \ \ \ \ \ \ \ \ \ \ \ \#ifdef}}\ BBOUND \\
\mbox{}\ \ \ \ \ \ \ \ \ \ \ \ \ \ \ \ \textbf{\textcolor{Blue}{if}}\ \textcolor{BrickRed}{(}actual\textcolor{BrickRed}{.}t$\_$ejecucion\ \textcolor{BrickRed}{\textless{}}\ solucion\textcolor{BrickRed}{.}t$\_$ejecucion\textcolor{BrickRed}{)} \\
\mbox{}\textbf{\textcolor{RoyalBlue}{\ \ \ \ \ \ \ \ \ \ \ \ \ \ \ \ \#endif}} \\
\mbox{}\ \ \ \ \ \ \ \ \ \ \ \ \ \ \ \ \ \ \ \ posibles\textcolor{BrickRed}{.}\textbf{\textcolor{Black}{push}}\textcolor{BrickRed}{(}actual\textcolor{BrickRed}{);} \\
\mbox{}\ \ \ \ \ \ \ \ \ \ \ \ \textcolor{Red}{\}} \\
\mbox{}\ \ \ \ \ \ \ \ \textcolor{Red}{\}} \\
\mbox{}\ \ \ \ \textcolor{Red}{\}} \\
\mbox{}\ \ \ \ \textbf{\textcolor{Blue}{return}}\ solucion\textcolor{BrickRed}{;} \\
\mbox{}\textcolor{Red}{\}} \\
\mbox{} \\
\mbox{}ostream\textcolor{BrickRed}{\&}\ \textbf{\textcolor{Blue}{operator}}\textcolor{BrickRed}{\textless{}\textless{}(}ostream\textcolor{BrickRed}{\&}\ out\textcolor{BrickRed}{,}\ \textbf{\textcolor{Blue}{const}}\ Planificador\textcolor{BrickRed}{::}Tarea\textcolor{BrickRed}{\&}\ t\textcolor{BrickRed}{)}\ \textcolor{Red}{\{} \\
\mbox{}\ \ \ \ out\ \textcolor{BrickRed}{\textless{}\textless{}}\ t\textcolor{BrickRed}{.}index\ \textcolor{BrickRed}{\textless{}\textless{}}\ \texttt{\textcolor{Red}{"{}\ [tiempo\ "{}}}\ \textcolor{BrickRed}{\textless{}\textless{}}\ t\textcolor{BrickRed}{.}ejecucion\ \textcolor{BrickRed}{\textless{}\textless{}}\ \texttt{\textcolor{Red}{"{};\ dependencias\ "{}}}\textcolor{BrickRed}{;} \\
\mbox{} \\
\mbox{}\ \ \ \ \textbf{\textcolor{Blue}{for}}\ \textcolor{BrickRed}{(}\textbf{\textcolor{Blue}{auto}}\textcolor{BrickRed}{\&}\ d\ \textcolor{BrickRed}{:}\ t\textcolor{BrickRed}{.}dependencias\textcolor{BrickRed}{)} \\
\mbox{}\ \ \ \ \ \ \ \ out\ \textcolor{BrickRed}{\textless{}\textless{}}\ d\ \textcolor{BrickRed}{\textless{}\textless{}}\ \texttt{\textcolor{Red}{"{}\ "{}}}\textcolor{BrickRed}{;} \\
\mbox{} \\
\mbox{}\ \ \ \ out\ \textcolor{BrickRed}{\textless{}\textless{}}\ \texttt{\textcolor{Red}{"{}]"{}}}\textcolor{BrickRed}{;} \\
\mbox{} \\
\mbox{}\ \ \ \ \textbf{\textcolor{Blue}{return}}\ out\textcolor{BrickRed}{;} \\
\mbox{}\textcolor{Red}{\}} \\
\mbox{} \\
\mbox{}\textcolor{ForestGreen}{int}\ \textbf{\textcolor{Black}{main}}\ \textcolor{BrickRed}{(}\textcolor{ForestGreen}{int}\ argc\textcolor{BrickRed}{,}\ \textcolor{ForestGreen}{char}\ \textbf{\textcolor{Blue}{const}}\ \textcolor{BrickRed}{*}argv\textcolor{BrickRed}{[])}\ \textcolor{Red}{\{} \\
\mbox{}\ \ \ \ \textcolor{TealBlue}{vector\textless{}Planificador::Tarea\textgreater{}}\ tareas\textcolor{BrickRed}{;} \\
\mbox{}\ \ \ \ \textit{\textcolor{Brown}{//Tarea\ t;}} \\
\mbox{}\ \ \ \ \textcolor{TealBlue}{tiempo}\ ej\textcolor{BrickRed}{;} \\
\mbox{}\ \ \ \ \textcolor{ForestGreen}{int}\ dep\textcolor{BrickRed}{;} \\
\mbox{}\ \ \ \ \textcolor{ForestGreen}{int}\ index\textcolor{BrickRed}{=}\textcolor{Purple}{0}\textcolor{BrickRed}{;} \\
\mbox{} \\
\mbox{}\ \ \ \ \textbf{\textcolor{Blue}{while}}\ \textcolor{BrickRed}{(}cin\ \textcolor{BrickRed}{\textgreater{}\textgreater{}}\ ej\textcolor{BrickRed}{)}\ \textcolor{Red}{\{} \\
\mbox{}\ \ \ \ \ \ \ \ \textcolor{TealBlue}{vector\textless{}int\textgreater{}}\ dependencias\textcolor{BrickRed}{;} \\
\mbox{} \\
\mbox{}\ \ \ \ \ \ \ \ \textbf{\textcolor{Blue}{while}}\ \textcolor{BrickRed}{((}cin\ \textcolor{BrickRed}{\textgreater{}\textgreater{}}\ dep\textcolor{BrickRed}{)}\ \textcolor{BrickRed}{\&\&}\ dep\ \textcolor{BrickRed}{\textgreater{}}\ \textcolor{BrickRed}{-}\textcolor{Purple}{1}\textcolor{BrickRed}{)}\ \textcolor{Red}{\{} \\
\mbox{}\ \ \ \ \ \ \ \ \ \ \ \ dependencias\textcolor{BrickRed}{.}\textbf{\textcolor{Black}{push$\_$back}}\textcolor{BrickRed}{(}dep\textcolor{BrickRed}{);} \\
\mbox{}\ \ \ \ \ \ \ \ \textcolor{Red}{\}} \\
\mbox{} \\
\mbox{}\ \ \ \ \ \ \ \ tareas\textcolor{BrickRed}{.}\textbf{\textcolor{Black}{push$\_$back}}\textcolor{BrickRed}{(}Planificador\textcolor{BrickRed}{::}\textbf{\textcolor{Black}{Tarea}}\textcolor{BrickRed}{(}index\textcolor{BrickRed}{,}\ ej\textcolor{BrickRed}{,}\ dependencias\textcolor{BrickRed}{));} \\
\mbox{}\ \ \ \ \ \ \ \ index\textcolor{BrickRed}{++;} \\
\mbox{}\ \ \ \ \textcolor{Red}{\}} \\
\mbox{}\ \ \ \ \textcolor{TealBlue}{Planificador}\ \textbf{\textcolor{Black}{instancia}}\textcolor{BrickRed}{(}tareas\textcolor{BrickRed}{);} \\
\mbox{} \\
\mbox{}\ \ \ \ Planificador\textcolor{BrickRed}{::}\textcolor{TealBlue}{Planificacion}\ solucion\ \textcolor{BrickRed}{=}\ instancia\textcolor{BrickRed}{.}\textbf{\textcolor{Black}{planifica}}\textcolor{BrickRed}{();} \\
\mbox{} \\
\mbox{}\ \ \ \ \textbf{\textcolor{Blue}{for}}\ \textcolor{BrickRed}{(}\textbf{\textcolor{Blue}{auto}}\textcolor{BrickRed}{\&}\ asig\ \textcolor{BrickRed}{:}\ solucion\textcolor{BrickRed}{.}historial\textcolor{BrickRed}{)} \\
\mbox{}\ \ \ \ \ \ \ \ cout\ \textcolor{BrickRed}{\textless{}\textless{}}\ \texttt{\textcolor{Red}{"{}Core\ "{}}}\ \textcolor{BrickRed}{\textless{}\textless{}}\ asig\textcolor{BrickRed}{.}core\ \textcolor{BrickRed}{\textless{}\textless{}}\ \texttt{\textcolor{Red}{"{}:\ tarea\ "{}}}\ \textcolor{BrickRed}{\textless{}\textless{}}\ asig\textcolor{BrickRed}{.}tarea\ \textcolor{BrickRed}{\textless{}\textless{}}\ \texttt{\textcolor{Red}{"{}\ (comenzando\ en\ "{}}}\ \textcolor{BrickRed}{\textless{}\textless{}}\ asig\textcolor{BrickRed}{.}t$\_$inicio\ \textcolor{BrickRed}{\textless{}\textless{}}\ \texttt{\textcolor{Red}{"{})"{}}}\ \textcolor{BrickRed}{\textless{}\textless{}}\ endl\textcolor{BrickRed}{;} \\
\mbox{} \\
\mbox{}\ \ \ \ cout\ \textcolor{BrickRed}{\textless{}\textless{}}\ \texttt{\textcolor{Red}{"{}Tiempo\ total:\ "{}}}\ \textcolor{BrickRed}{\textless{}\textless{}}\ solucion\textcolor{BrickRed}{.}t$\_$ejecucion\ \textcolor{BrickRed}{\textless{}\textless{}}\ endl\textcolor{BrickRed}{;} \\
\mbox{}\textcolor{Red}{\}} \\
\mbox{}
}
        \normalsize
  \subsection{3-Dimensional Matching}
        \small
  	\texttt{% Generator: GNU source-highlight, by Lorenzo Bettini, http://www.gnu.org/software/src-highlite
\noindent
\mbox{}\textit{\textcolor{Brown}{/**}} \\
\mbox{}\textit{\textcolor{Brown}{\ *\ 3matching.cpp}} \\
\mbox{}\textit{\textcolor{Brown}{\ *\ 3\ dimensional\ matching\ problem.}} \\
\mbox{}\textit{\textcolor{Brown}{\ *\ Implementación\ del\ algoritmo\ de\ backtracking\ en\ C++.}} \\
\mbox{}\textit{\textcolor{Brown}{\ */}} \\
\mbox{} \\
\mbox{}\textbf{\textcolor{RoyalBlue}{\#include}}\ \texttt{\textcolor{Red}{\textless{}iostream\textgreater{}}} \\
\mbox{}\textbf{\textcolor{RoyalBlue}{\#include}}\ \texttt{\textcolor{Red}{\textless{}vector\textgreater{}}} \\
\mbox{}\textbf{\textcolor{RoyalBlue}{\#include}}\ \texttt{\textcolor{Red}{\textless{}queue\textgreater{}}} \\
\mbox{}\textbf{\textcolor{RoyalBlue}{\#include}}\ \texttt{\textcolor{Red}{\textless{}chrono\textgreater{}}} \\
\mbox{}\textbf{\textcolor{RoyalBlue}{\#include}}\ \texttt{\textcolor{Red}{\textless{}utility\textgreater{}}} \\
\mbox{}\textbf{\textcolor{Blue}{using}}\ \textbf{\textcolor{Blue}{namespace}}\ std\textcolor{BrickRed}{;} \\
\mbox{}\textbf{\textcolor{Blue}{typedef}}\ \textcolor{ForestGreen}{unsigned}\ \textcolor{ForestGreen}{int}\ uint\textcolor{BrickRed}{;} \\
\mbox{} \\
\mbox{}\textit{\textcolor{Brown}{//\ Estructura\ para\ representar\ una\ arista.}} \\
\mbox{}\textit{\textcolor{Brown}{//\ Indica\ los\ tres\ puntos\ en\ los\ tres\ conjuntos\ unidos\ por\ la\ arista.}} \\
\mbox{}\textbf{\textcolor{Blue}{struct}}\ \textcolor{TealBlue}{Arista}\ \textcolor{Red}{\{} \\
\mbox{}\ \ \ \ \textcolor{ForestGreen}{int}\ a\textcolor{BrickRed}{;} \\
\mbox{}\ \ \ \ \textcolor{ForestGreen}{int}\ b\textcolor{BrickRed}{;} \\
\mbox{}\ \ \ \ \textcolor{ForestGreen}{int}\ c\textcolor{BrickRed}{;} \\
\mbox{} \\
\mbox{}\ \ \ \ \textbf{\textcolor{Black}{Arista}}\ \textcolor{BrickRed}{(}\textcolor{ForestGreen}{int}\ x\textcolor{BrickRed}{,}\ \textcolor{ForestGreen}{int}\ y\textcolor{BrickRed}{,}\ \textcolor{ForestGreen}{int}\ z\textcolor{BrickRed}{)} \\
\mbox{}\ \ \ \ \textcolor{BrickRed}{:}\textbf{\textcolor{Black}{a}}\textcolor{BrickRed}{(}x\textcolor{BrickRed}{),}\ \textbf{\textcolor{Black}{b}}\textcolor{BrickRed}{(}y\textcolor{BrickRed}{),}\ \textbf{\textcolor{Black}{c}}\textcolor{BrickRed}{(}z\textcolor{BrickRed}{)} \\
\mbox{}\ \ \ \ \textcolor{Red}{\{\}} \\
\mbox{}\textcolor{Red}{\}}\textcolor{BrickRed}{;} \\
\mbox{} \\
\mbox{}ostream\textcolor{BrickRed}{\&}\ \textbf{\textcolor{Blue}{operator}}\ \textcolor{BrickRed}{\textless{}\textless{}}\ \textcolor{BrickRed}{(}ostream\textcolor{BrickRed}{\&}\ output\textcolor{BrickRed}{,}\ Arista\textcolor{BrickRed}{\&}\ arista\textcolor{BrickRed}{)}\ \textcolor{Red}{\{} \\
\mbox{}\ \ \ \ cout\ \textcolor{BrickRed}{\textless{}\textless{}}\ \texttt{\textcolor{Red}{"{}("{}}}\ \textcolor{BrickRed}{\textless{}\textless{}}\ arista\textcolor{BrickRed}{.}a\ \textcolor{BrickRed}{\textless{}\textless{}}\ \texttt{\textcolor{Red}{"{},"{}}}\ \textcolor{BrickRed}{\textless{}\textless{}}\ arista\textcolor{BrickRed}{.}b\ \textcolor{BrickRed}{\textless{}\textless{}}\ \texttt{\textcolor{Red}{"{},"{}}}\ \textcolor{BrickRed}{\textless{}\textless{}}\ arista\textcolor{BrickRed}{.}c\ \textcolor{BrickRed}{\textless{}\textless{}}\ \texttt{\textcolor{Red}{"{})"{}}}\textcolor{BrickRed}{;} \\
\mbox{}\ \ \ \ \textbf{\textcolor{Blue}{return}}\ output\textcolor{BrickRed}{;} \\
\mbox{}\textcolor{Red}{\}} \\
\mbox{} \\
\mbox{}\textbf{\textcolor{Blue}{template}}\textcolor{BrickRed}{\textless{}}\textbf{\textcolor{Blue}{class}}\ \textcolor{TealBlue}{T}\textcolor{BrickRed}{\textgreater{}} \\
\mbox{}ostream\textcolor{BrickRed}{\&}\ \textbf{\textcolor{Blue}{operator}}\textcolor{BrickRed}{\textless{}\textless{}}\ \textcolor{BrickRed}{(}ostream\textcolor{BrickRed}{\&}\ output\textcolor{BrickRed}{,}\ vector\textcolor{BrickRed}{\textless{}}T\textcolor{BrickRed}{\textgreater{}\&}\ v\textcolor{BrickRed}{)}\textcolor{Red}{\{} \\
\mbox{}\ \ \ \ \textbf{\textcolor{Blue}{for}}\ \textcolor{BrickRed}{(}\textbf{\textcolor{Blue}{auto}}\ i\ \textcolor{BrickRed}{:}\ v\textcolor{BrickRed}{)} \\
\mbox{}\ \ \ \ \ \ \ \ output\ \textcolor{BrickRed}{\textless{}\textless{}}\ i\textcolor{BrickRed}{;} \\
\mbox{}\ \ \ \ output\ \textcolor{BrickRed}{\textless{}\textless{}}\ endl\textcolor{BrickRed}{;} \\
\mbox{}\ \ \ \ \textbf{\textcolor{Blue}{return}}\ output\textcolor{BrickRed}{;} \\
\mbox{}\textcolor{Red}{\}} \\
\mbox{} \\
\mbox{}\textit{\textcolor{Brown}{//\ Tamaños\ de\ las\ tablas\ de\ nodos}} \\
\mbox{}\textcolor{ForestGreen}{int}\ sizea\ \textcolor{BrickRed}{=}\ \textcolor{Purple}{0}\textcolor{BrickRed}{;} \\
\mbox{}\textcolor{ForestGreen}{int}\ sizeb\ \textcolor{BrickRed}{=}\ \textcolor{Purple}{0}\textcolor{BrickRed}{;} \\
\mbox{}\textcolor{ForestGreen}{int}\ sizec\ \textcolor{BrickRed}{=}\ \textcolor{Purple}{0}\textcolor{BrickRed}{;} \\
\mbox{} \\
\mbox{}\textit{\textcolor{Brown}{//\ Estructura\ para\ representar\ una\ asignación.}} \\
\mbox{}\textit{\textcolor{Brown}{//\ Señala\ las\ aristas\ seleccionadas\ y\ los\ nodos\ usados\ de\ cada\ conjunto.}} \\
\mbox{}\textit{\textcolor{Brown}{//\ Señala\ además\ la\ cardinalidad\ de\ la\ asignación\ (su\ valor).}} \\
\mbox{}\textbf{\textcolor{Blue}{struct}}\ \textcolor{TealBlue}{Matching}\ \textcolor{Red}{\{} \\
\mbox{}\ \ \ \ \textcolor{TealBlue}{vector\textless{}bool\textgreater{}}\ aristas\textcolor{BrickRed}{;} \\
\mbox{}\ \ \ \ \textcolor{TealBlue}{vector\textless{}bool\textgreater{}}\ nodosa\textcolor{BrickRed}{;} \\
\mbox{}\ \ \ \ \textcolor{TealBlue}{vector\textless{}bool\textgreater{}}\ nodosb\textcolor{BrickRed}{;} \\
\mbox{}\ \ \ \ \textcolor{TealBlue}{vector\textless{}bool\textgreater{}}\ nodosc\textcolor{BrickRed}{;} \\
\mbox{}\ \ \ \ \textcolor{ForestGreen}{int}\ valor\textcolor{BrickRed}{;} \\
\mbox{} \\
\mbox{}\ \ \ \ \textbf{\textcolor{Black}{Matching}}\ \textcolor{BrickRed}{()} \\
\mbox{}\ \ \ \ \textcolor{BrickRed}{:}\ \textbf{\textcolor{Black}{valor}}\textcolor{BrickRed}{(}\textcolor{Purple}{0}\textcolor{BrickRed}{),} \\
\mbox{}\ \ \ \ \textbf{\textcolor{Black}{nodosa}}\textcolor{BrickRed}{(}sizea\textcolor{BrickRed}{,}\ \textbf{\textcolor{Blue}{false}}\textcolor{BrickRed}{),} \\
\mbox{}\ \ \ \ \textbf{\textcolor{Black}{nodosb}}\textcolor{BrickRed}{(}sizeb\textcolor{BrickRed}{,}\ \textbf{\textcolor{Blue}{false}}\textcolor{BrickRed}{),} \\
\mbox{}\ \ \ \ \textbf{\textcolor{Black}{nodosc}}\textcolor{BrickRed}{(}sizec\textcolor{BrickRed}{,}\ \textbf{\textcolor{Blue}{false}}\textcolor{BrickRed}{)} \\
\mbox{}\ \ \ \ \textcolor{Red}{\{\}} \\
\mbox{}\textcolor{Red}{\}}\textcolor{BrickRed}{;} \\
\mbox{} \\
\mbox{}\textbf{\textcolor{Blue}{struct}}\ \textcolor{TealBlue}{cmp}\textcolor{Red}{\{} \\
\mbox{}\ \ \ \ \textcolor{ForestGreen}{bool}\ \textbf{\textcolor{Blue}{operator}}\textcolor{BrickRed}{()}\ \textcolor{BrickRed}{(}\textbf{\textcolor{Blue}{const}}\ Matching\textcolor{BrickRed}{\&}\ una\textcolor{BrickRed}{,} \\
\mbox{}\ \ \ \ \ \ \ \ \ \ \ \ \ \ \ \ \ \ \ \ \ \textbf{\textcolor{Blue}{const}}\ Matching\textcolor{BrickRed}{\&}\ otra\textcolor{BrickRed}{)}\textcolor{Red}{\{} \\
\mbox{}\ \ \ \ \ \ \ \ \textbf{\textcolor{Blue}{return}}\ una\textcolor{BrickRed}{.}valor\ \textcolor{BrickRed}{\textless{}}\ otra\textcolor{BrickRed}{.}valor\textcolor{BrickRed}{;} \\
\mbox{}\ \ \ \ \ \ \ \ \ \ \ \ \ \ \ \ \ \ \ \ \ \textcolor{Red}{\}} \\
\mbox{}\textcolor{Red}{\}}\textcolor{BrickRed}{;} \\
\mbox{} \\
\mbox{}\textcolor{TealBlue}{Matching}\ \textbf{\textcolor{Black}{resolver}}\textcolor{BrickRed}{(}\textcolor{TealBlue}{vector\textless{}Arista\textgreater{}\ aristas,\ vector\textless{}int\textgreater{}}\ preferencias\textcolor{BrickRed}{)}\ \textcolor{Red}{\{} \\
\mbox{}\ \ \ \ \textit{\textcolor{Brown}{//\ Prueba\ combinaciones\ de\ aristas.}} \\
\mbox{}\ \ \ \ \textit{\textcolor{Brown}{//\ Marca\ como\ true\ las\ aristas\ escogidas.}} \\
\mbox{}\textbf{\textcolor{RoyalBlue}{\ \ \ \ \#ifdef}}\ BBOUND \\
\mbox{}\ \ \ \ \textcolor{TealBlue}{priority$\_$queue\textless{}Matching,\ vector\textless{}Matching\textgreater{},\ cmp\textgreater{}}\ posibles$\_$particiones\textcolor{BrickRed}{;} \\
\mbox{}\textbf{\textcolor{RoyalBlue}{\ \ \ \ \#else}} \\
\mbox{}\ \ \ \ \textcolor{TealBlue}{queue\textless{}Matching\textgreater{}}\ posibles$\_$particiones\textcolor{BrickRed}{;} \\
\mbox{}\textbf{\textcolor{RoyalBlue}{\ \ \ \ \#endif}} \\
\mbox{}\ \ \ \ \textcolor{TealBlue}{Matching}\ solucion\textcolor{BrickRed}{;} \\
\mbox{}\ \ \ \ \textcolor{TealBlue}{uint}\ tamanio\ \textcolor{BrickRed}{=}\ aristas\textcolor{BrickRed}{.}\textbf{\textcolor{Black}{size}}\textcolor{BrickRed}{();} \\
\mbox{} \\
\mbox{}\ \ \ \ \textit{\textcolor{Brown}{//\ Prueba\ cada\ posible\ asignación,\ empezando\ por\ la\ vacía.}} \\
\mbox{}\ \ \ \ posibles$\_$particiones\textcolor{BrickRed}{.}\textbf{\textcolor{Black}{push}}\textcolor{BrickRed}{(}\textbf{\textcolor{Black}{Matching}}\textcolor{BrickRed}{());} \\
\mbox{}\ \ \ \ \textbf{\textcolor{Blue}{while}}\ \textcolor{BrickRed}{(!}posibles$\_$particiones\textcolor{BrickRed}{.}\textbf{\textcolor{Black}{empty}}\textcolor{BrickRed}{())}\ \textcolor{Red}{\{} \\
\mbox{}\textbf{\textcolor{RoyalBlue}{\ \ \ \ \ \ \ \ \#ifdef}}\ BBOUND \\
\mbox{}\ \ \ \ \ \ \ \ \textcolor{TealBlue}{Matching}\ actual\ \textcolor{BrickRed}{=}\ posibles$\_$particiones\textcolor{BrickRed}{.}\textbf{\textcolor{Black}{top}}\textcolor{BrickRed}{();} \\
\mbox{}\textbf{\textcolor{RoyalBlue}{\ \ \ \ \ \ \ \ \#else}} \\
\mbox{}\ \ \ \ \ \ \ \ \textcolor{TealBlue}{Matching}\ actual\ \textcolor{BrickRed}{=}\ posibles$\_$particiones\textcolor{BrickRed}{.}\textbf{\textcolor{Black}{front}}\textcolor{BrickRed}{();} \\
\mbox{}\textbf{\textcolor{RoyalBlue}{\ \ \ \ \ \ \ \ \#endif}} \\
\mbox{}\ \ \ \ \ \ \ \ posibles$\_$particiones\textcolor{BrickRed}{.}\textbf{\textcolor{Black}{pop}}\textcolor{BrickRed}{();} \\
\mbox{}\ \ \ \ \ \ \ \ \textcolor{TealBlue}{uint}\ indice\ \textcolor{BrickRed}{=}\ actual\textcolor{BrickRed}{.}aristas\textcolor{BrickRed}{.}\textbf{\textcolor{Black}{size}}\textcolor{BrickRed}{();} \\
\mbox{} \\
\mbox{}\ \ \ \ \ \ \ \ \textit{\textcolor{Brown}{//\ Caso\ de\ matching\ completo}} \\
\mbox{}\ \ \ \ \ \ \ \ \textbf{\textcolor{Blue}{if}}\ \textcolor{BrickRed}{(}indice\ \textcolor{BrickRed}{==}\ tamanio\textcolor{BrickRed}{)}\ \textcolor{Red}{\{} \\
\mbox{}\ \ \ \ \ \ \ \ \ \ \ \ \textbf{\textcolor{Blue}{if}}\ \textcolor{BrickRed}{(}actual\textcolor{BrickRed}{.}valor\ \textcolor{BrickRed}{\textgreater{}}\ solucion\textcolor{BrickRed}{.}valor\textcolor{BrickRed}{)} \\
\mbox{}\ \ \ \ \ \ \ \ \ \ \ \ \ \ \ \ solucion\ \textcolor{BrickRed}{=}\ actual\textcolor{BrickRed}{;} \\
\mbox{}\ \ \ \ \ \ \ \ \textcolor{Red}{\}} \\
\mbox{} \\
\mbox{}\ \ \ \ \ \ \ \ \textit{\textcolor{Brown}{//\ Caso\ de\ matching\ por\ completar.}} \\
\mbox{}\ \ \ \ \ \ \ \ \textit{\textcolor{Brown}{//\ Añade\ el\ caso\ de\ que\ se\ obtenga\ la\ arista\ o\ no.}} \\
\mbox{}\ \ \ \ \ \ \ \ \textbf{\textcolor{Blue}{else}}\ \textcolor{Red}{\{} \\
\mbox{}\ \ \ \ \ \ \ \ \ \ \ \ \textcolor{TealBlue}{Matching}\ con$\_$nueva\ \textcolor{BrickRed}{=}\ actual\textcolor{BrickRed}{;} \\
\mbox{}\ \ \ \ \ \ \ \ \ \ \ \ \textcolor{TealBlue}{Matching}\ sin$\_$nueva\ \textcolor{BrickRed}{=}\ actual\textcolor{BrickRed}{;} \\
\mbox{}\ \ \ \ \ \ \ \ \ \ \ \ con$\_$nueva\textcolor{BrickRed}{.}aristas\textcolor{BrickRed}{.}\textbf{\textcolor{Black}{push$\_$back}}\textcolor{BrickRed}{(}\textbf{\textcolor{Blue}{true}}\textcolor{BrickRed}{);} \\
\mbox{}\ \ \ \ \ \ \ \ \ \ \ \ sin$\_$nueva\textcolor{BrickRed}{.}aristas\textcolor{BrickRed}{.}\textbf{\textcolor{Black}{push$\_$back}}\textcolor{BrickRed}{(}\textbf{\textcolor{Blue}{false}}\textcolor{BrickRed}{);} \\
\mbox{} \\
\mbox{}\ \ \ \ \ \ \ \ \ \ \ \ \textcolor{TealBlue}{Arista}\ nueva$\_$arista\ \textcolor{BrickRed}{=}\ aristas\textcolor{BrickRed}{[}indice\textcolor{BrickRed}{];} \\
\mbox{} \\
\mbox{}\textbf{\textcolor{RoyalBlue}{\ \ \ \ \ \ \ \ \ \ \ \ \#ifdef}}\ BBOUND \\
\mbox{}\ \ \ \ \ \ \ \ \ \ \ \ \textcolor{ForestGreen}{int}\ \textbf{\textcolor{Black}{sum$\_$pref}}\textcolor{BrickRed}{(}\textcolor{Purple}{0}\textcolor{BrickRed}{);} \\
\mbox{} \\
\mbox{}\ \ \ \ \ \ \ \ \ \ \ \ \textit{\textcolor{Brown}{//\ Condición\ de\ poda:\ poder\ mejorar\ la\ satisfacción\ de\ la\ solución\ actual}} \\
\mbox{}\ \ \ \ \ \ \ \ \ \ \ \ \textbf{\textcolor{Blue}{for}}\ \textcolor{BrickRed}{(}\textcolor{TealBlue}{uint}\ i\textcolor{BrickRed}{=}indice\textcolor{BrickRed}{+}\textcolor{Purple}{1}\textcolor{BrickRed}{;}\ i\textcolor{BrickRed}{\textless{}}tamanio\textcolor{BrickRed}{;}\ \textcolor{BrickRed}{++}i\textcolor{BrickRed}{)} \\
\mbox{}\ \ \ \ \ \ \ \ \ \ \ \ \ \ \ \ \textbf{\textcolor{Blue}{if}}\ \textcolor{BrickRed}{((}\textcolor{TealBlue}{not}\ actual\textcolor{BrickRed}{.}nodosa\textcolor{BrickRed}{.}\textbf{\textcolor{Black}{at}}\textcolor{BrickRed}{(}nueva$\_$arista\textcolor{BrickRed}{.}a\textcolor{BrickRed}{))}\ and \\
\mbox{}\ \ \ \ \ \ \ \ \ \ \ \ \ \ \ \ \ \ \ \ \textcolor{BrickRed}{(}\textcolor{TealBlue}{not}\ actual\textcolor{BrickRed}{.}nodosb\textcolor{BrickRed}{.}\textbf{\textcolor{Black}{at}}\textcolor{BrickRed}{(}nueva$\_$arista\textcolor{BrickRed}{.}b\textcolor{BrickRed}{))}\ and \\
\mbox{}\ \ \ \ \ \ \ \ \ \ \ \ \ \ \ \ \ \ \ \ \textcolor{BrickRed}{(}\textcolor{TealBlue}{not}\ actual\textcolor{BrickRed}{.}nodosc\textcolor{BrickRed}{.}\textbf{\textcolor{Black}{at}}\textcolor{BrickRed}{(}nueva$\_$arista\textcolor{BrickRed}{.}c\textcolor{BrickRed}{)))} \\
\mbox{}\ \ \ \ \ \ \ \ \ \ \ \ \ \ \ \ \ \ \ \ sum$\_$pref\ \textcolor{BrickRed}{+=}\ preferencias\textcolor{BrickRed}{[}i\textcolor{BrickRed}{];} \\
\mbox{} \\
\mbox{}\ \ \ \ \ \ \ \ \ \ \ \ \textbf{\textcolor{Blue}{if}}\ \textcolor{BrickRed}{(}sin$\_$nueva\textcolor{BrickRed}{.}valor\ \textcolor{BrickRed}{+}\ sum$\_$pref\ \textcolor{BrickRed}{\textgreater{}}\ solucion\textcolor{BrickRed}{.}valor\textcolor{BrickRed}{)} \\
\mbox{}\ \ \ \ \ \ \ \ \ \ \ \ \ \ \ \ posibles$\_$particiones\textcolor{BrickRed}{.}\textbf{\textcolor{Black}{push}}\textcolor{BrickRed}{(}sin$\_$nueva\textcolor{BrickRed}{);} \\
\mbox{} \\
\mbox{}\textbf{\textcolor{RoyalBlue}{\ \ \ \ \ \ \ \ \ \ \ \ \#else}} \\
\mbox{}\ \ \ \ \ \ \ \ \ \ \ \ \textit{\textcolor{Brown}{//\ Siempre\ puede\ continuarse\ sin\ añadir\ nada.}} \\
\mbox{}\ \ \ \ \ \ \ \ \ \ \ \ posibles$\_$particiones\textcolor{BrickRed}{.}\textbf{\textcolor{Black}{push}}\textcolor{BrickRed}{(}sin$\_$nueva\textcolor{BrickRed}{);} \\
\mbox{}\textbf{\textcolor{RoyalBlue}{\ \ \ \ \ \ \ \ \ \ \ \ \#endif}} \\
\mbox{} \\
\mbox{}\ \ \ \ \ \ \ \ \ \ \ \ \textit{\textcolor{Brown}{//\ Comprobamos\ si\ se\ puede\ añadir\ la\ arista.}} \\
\mbox{}\ \ \ \ \ \ \ \ \ \ \ \ \textbf{\textcolor{Blue}{if}}\ \textcolor{BrickRed}{((}\textcolor{TealBlue}{not}\ actual\textcolor{BrickRed}{.}nodosa\textcolor{BrickRed}{.}\textbf{\textcolor{Black}{at}}\textcolor{BrickRed}{(}nueva$\_$arista\textcolor{BrickRed}{.}a\textcolor{BrickRed}{))}\ and \\
\mbox{}\ \ \ \ \ \ \ \ \ \ \ \ \ \ \ \ \textcolor{BrickRed}{(}\textcolor{TealBlue}{not}\ actual\textcolor{BrickRed}{.}nodosb\textcolor{BrickRed}{.}\textbf{\textcolor{Black}{at}}\textcolor{BrickRed}{(}nueva$\_$arista\textcolor{BrickRed}{.}b\textcolor{BrickRed}{))}\ and \\
\mbox{}\ \ \ \ \ \ \ \ \ \ \ \ \ \ \ \ \textcolor{BrickRed}{(}\textcolor{TealBlue}{not}\ actual\textcolor{BrickRed}{.}nodosc\textcolor{BrickRed}{.}\textbf{\textcolor{Black}{at}}\textcolor{BrickRed}{(}nueva$\_$arista\textcolor{BrickRed}{.}c\textcolor{BrickRed}{)))} \\
\mbox{}\ \ \ \ \ \ \ \ \ \ \ \ \textcolor{Red}{\{} \\
\mbox{}\ \ \ \ \ \ \ \ \ \ \ \ \ \ \ \ con$\_$nueva\textcolor{BrickRed}{.}nodosa\textcolor{BrickRed}{[}nueva$\_$arista\textcolor{BrickRed}{.}a\textcolor{BrickRed}{]}\ \textcolor{BrickRed}{=}\ \textbf{\textcolor{Blue}{true}}\textcolor{BrickRed}{;} \\
\mbox{}\ \ \ \ \ \ \ \ \ \ \ \ \ \ \ \ con$\_$nueva\textcolor{BrickRed}{.}nodosb\textcolor{BrickRed}{[}nueva$\_$arista\textcolor{BrickRed}{.}b\textcolor{BrickRed}{]}\ \textcolor{BrickRed}{=}\ \textbf{\textcolor{Blue}{true}}\textcolor{BrickRed}{;} \\
\mbox{}\ \ \ \ \ \ \ \ \ \ \ \ \ \ \ \ con$\_$nueva\textcolor{BrickRed}{.}nodosc\textcolor{BrickRed}{[}nueva$\_$arista\textcolor{BrickRed}{.}c\textcolor{BrickRed}{]}\ \textcolor{BrickRed}{=}\ \textbf{\textcolor{Blue}{true}}\textcolor{BrickRed}{;} \\
\mbox{}\textbf{\textcolor{RoyalBlue}{\ \ \ \ \ \ \ \ \ \ \ \ \ \ \ \ \#ifdef}}\ BBOUND \\
\mbox{}\ \ \ \ \ \ \ \ \ \ \ \ \ \ \ \ \textit{\textcolor{Brown}{//\ Sumamos\ la\ preferencia}} \\
\mbox{}\ \ \ \ \ \ \ \ \ \ \ \ \ \ \ \ con$\_$nueva\textcolor{BrickRed}{.}valor\textcolor{BrickRed}{+=}\ preferencias\textcolor{BrickRed}{[}indice\textcolor{BrickRed}{];} \\
\mbox{}\textbf{\textcolor{RoyalBlue}{\ \ \ \ \ \ \ \ \ \ \ \ \ \ \ \ \#else}} \\
\mbox{}\ \ \ \ \ \ \ \ \ \ \ \ \ \ \ \ con$\_$nueva\textcolor{BrickRed}{.}valor\textcolor{BrickRed}{++;} \\
\mbox{}\textbf{\textcolor{RoyalBlue}{\ \ \ \ \ \ \ \ \ \ \ \ \ \ \ \ \#endif}} \\
\mbox{}\ \ \ \ \ \ \ \ \ \ \ \ \ \ \ \ posibles$\_$particiones\textcolor{BrickRed}{.}\textbf{\textcolor{Black}{push}}\textcolor{BrickRed}{(}con$\_$nueva\textcolor{BrickRed}{);} \\
\mbox{}\ \ \ \ \ \ \ \ \ \ \ \ \textcolor{Red}{\}} \\
\mbox{}\ \ \ \ \ \ \ \ \textcolor{Red}{\}} \\
\mbox{}\ \ \ \ \textcolor{Red}{\}} \\
\mbox{}\ \ \ \ \textbf{\textcolor{Blue}{return}}\ solucion\textcolor{BrickRed}{;} \\
\mbox{}\textcolor{Red}{\}} \\
\mbox{}\textcolor{ForestGreen}{int}\ \textbf{\textcolor{Black}{main}}\ \textcolor{BrickRed}{()}\ \textcolor{Red}{\{} \\
\mbox{}\ \ \ \ \textit{\textcolor{Brown}{//\ Bloque\ de\ entradas}} \\
\mbox{}\ \ \ \ \textcolor{TealBlue}{vector\textless{}Arista\textgreater{}}\ aristas\textcolor{BrickRed}{;} \\
\mbox{}\ \ \ \ \textcolor{TealBlue}{vector\textless{}int\textgreater{}}\ preferencias\textcolor{BrickRed}{;} \\
\mbox{}\ \ \ \ \textcolor{ForestGreen}{int}\ preferencia\textcolor{BrickRed}{;} \\
\mbox{}\ \ \ \ \textcolor{ForestGreen}{int}\ a\textcolor{BrickRed}{,}b\textcolor{BrickRed}{,}c\textcolor{BrickRed}{;} \\
\mbox{} \\
\mbox{}\textbf{\textcolor{RoyalBlue}{\ \ \ \ \#ifdef}}\ BBOUND \\
\mbox{}\ \ \ \ \textbf{\textcolor{Blue}{while}}\ \textcolor{BrickRed}{(}cin\ \textcolor{BrickRed}{\textgreater{}\textgreater{}}\ a\ \textcolor{BrickRed}{\textgreater{}\textgreater{}}\ b\ \textcolor{BrickRed}{\textgreater{}\textgreater{}}\ c\ \textcolor{BrickRed}{\textgreater{}\textgreater{}}\ preferencia\textcolor{BrickRed}{)}\textcolor{Red}{\{} \\
\mbox{}\ \ \ \ \ \ \ \ aristas\textcolor{BrickRed}{.}\textbf{\textcolor{Black}{push$\_$back}}\textcolor{BrickRed}{(}\textbf{\textcolor{Black}{Arista}}\textcolor{BrickRed}{(}a\textcolor{BrickRed}{,}b\textcolor{BrickRed}{,}c\textcolor{BrickRed}{));} \\
\mbox{}\ \ \ \ \ \ \ \ preferencias\textcolor{BrickRed}{.}\textbf{\textcolor{Black}{push$\_$back}}\textcolor{BrickRed}{(}preferencia\textcolor{BrickRed}{);} \\
\mbox{}\ \ \ \ \textcolor{Red}{\}} \\
\mbox{}\textbf{\textcolor{RoyalBlue}{\ \ \ \ \#else}} \\
\mbox{}\ \ \ \ \textbf{\textcolor{Blue}{while}}\ \textcolor{BrickRed}{(}cin\ \textcolor{BrickRed}{\textgreater{}\textgreater{}}\ a\ \textcolor{BrickRed}{\textgreater{}\textgreater{}}\ b\ \textcolor{BrickRed}{\textgreater{}\textgreater{}}\ c\textcolor{BrickRed}{)} \\
\mbox{}\ \ \ \ \ \ \ \ aristas\textcolor{BrickRed}{.}\textbf{\textcolor{Black}{push$\_$back}}\textcolor{BrickRed}{(}\textbf{\textcolor{Black}{Arista}}\textcolor{BrickRed}{(}a\textcolor{BrickRed}{,}b\textcolor{BrickRed}{,}c\textcolor{BrickRed}{));} \\
\mbox{}\textbf{\textcolor{RoyalBlue}{\ \ \ \ \#endif}} \\
\mbox{} \\
\mbox{}\ \ \ \ \textit{\textcolor{Brown}{//\ Bloque\ de\ cómputos}} \\
\mbox{}\ \ \ \ \textit{\textcolor{Brown}{//\ El\ tamaño\ de\ cada\ tabla\ de\ nodos\ usados\ será\ la\ mayor\ de\ sus\ componentes.}} \\
\mbox{}\ \ \ \ \textbf{\textcolor{Blue}{for}}\ \textcolor{BrickRed}{(}\textbf{\textcolor{Blue}{auto}}\ arista\ \textcolor{BrickRed}{:}\ aristas\textcolor{BrickRed}{)}\ \textcolor{Red}{\{} \\
\mbox{}\ \ \ \ \ \ \ \ \textbf{\textcolor{Blue}{if}}\ \textcolor{BrickRed}{(}sizea\ \textcolor{BrickRed}{\textless{}}\ arista\textcolor{BrickRed}{.}a\textcolor{BrickRed}{+}\textcolor{Purple}{1}\textcolor{BrickRed}{)} \\
\mbox{}\ \ \ \ \ \ \ \ \ \ \ \ sizea\ \textcolor{BrickRed}{=}\ arista\textcolor{BrickRed}{.}a\textcolor{BrickRed}{+}\textcolor{Purple}{1}\textcolor{BrickRed}{;} \\
\mbox{}\ \ \ \ \ \ \ \ \textbf{\textcolor{Blue}{if}}\ \textcolor{BrickRed}{(}sizeb\ \textcolor{BrickRed}{\textless{}}\ arista\textcolor{BrickRed}{.}b\textcolor{BrickRed}{+}\textcolor{Purple}{1}\textcolor{BrickRed}{)} \\
\mbox{}\ \ \ \ \ \ \ \ \ \ \ \ sizeb\ \textcolor{BrickRed}{=}\ arista\textcolor{BrickRed}{.}b\textcolor{BrickRed}{+}\textcolor{Purple}{1}\textcolor{BrickRed}{;} \\
\mbox{}\ \ \ \ \ \ \ \ \textbf{\textcolor{Blue}{if}}\ \textcolor{BrickRed}{(}sizec\ \textcolor{BrickRed}{\textless{}}\ arista\textcolor{BrickRed}{.}c\textcolor{BrickRed}{+}\textcolor{Purple}{1}\textcolor{BrickRed}{)} \\
\mbox{}\ \ \ \ \ \ \ \ \ \ \ \ sizec\ \textcolor{BrickRed}{=}\ arista\textcolor{BrickRed}{.}c\textcolor{BrickRed}{+}\textcolor{Purple}{1}\textcolor{BrickRed}{;} \\
\mbox{}\ \ \ \ \textcolor{Red}{\}} \\
\mbox{} \\
\mbox{}\ \ \ \ \textcolor{TealBlue}{Matching}\ solucion\ \textcolor{BrickRed}{=}\ \textbf{\textcolor{Black}{resolver}}\textcolor{BrickRed}{(}aristas\textcolor{BrickRed}{,}\ preferencias\textcolor{BrickRed}{);} \\
\mbox{} \\
\mbox{}\ \ \ \ \textit{\textcolor{Brown}{//\ Bloque\ de\ salidas}} \\
\mbox{}\ \ \ \ \textit{\textcolor{Brown}{//\ Escribe\ la\ solución.}} \\
\mbox{}\ \ \ \ cout\ \textcolor{BrickRed}{\textless{}\textless{}}\ \texttt{\textcolor{Red}{"{}Solución:}}\texttt{\textcolor{CarnationPink}{\textbackslash{}n}}\texttt{\textcolor{Red}{"{}}}\textcolor{BrickRed}{;} \\
\mbox{}\ \ \ \ \textbf{\textcolor{Blue}{for}}\ \textcolor{BrickRed}{(}\textcolor{TealBlue}{uint}\ i\textcolor{BrickRed}{=}\textcolor{Purple}{0}\textcolor{BrickRed}{;}\ i\textcolor{BrickRed}{\textless{}}aristas\textcolor{BrickRed}{.}\textbf{\textcolor{Black}{size}}\textcolor{BrickRed}{();}\ i\textcolor{BrickRed}{++)} \\
\mbox{}\ \ \ \ \ \ \ \ \textbf{\textcolor{Blue}{if}}\ \textcolor{BrickRed}{(}solucion\textcolor{BrickRed}{.}aristas\textcolor{BrickRed}{[}i\textcolor{BrickRed}{])} \\
\mbox{}\ \ \ \ \ \ \ \ \ \ \ \ cout\ \textcolor{BrickRed}{\textless{}\textless{}}\ aristas\textcolor{BrickRed}{[}i\textcolor{BrickRed}{]}\ \textcolor{BrickRed}{\textless{}\textless{}}\ endl\textcolor{BrickRed}{;} \\
\mbox{}\textbf{\textcolor{RoyalBlue}{\ \ \ \ \#ifdef}}\ BBOUND \\
\mbox{}\ \ \ \ cout\ \textcolor{BrickRed}{\textless{}\textless{}}\ \texttt{\textcolor{Red}{"{}Satisfacción:\ "{}}}\textcolor{BrickRed}{;} \\
\mbox{}\textbf{\textcolor{RoyalBlue}{\ \ \ \ \#else}} \\
\mbox{}\ \ \ \ cout\ \textcolor{BrickRed}{\textless{}\textless{}}\ \texttt{\textcolor{Red}{"{}Cardinalidad:\ "{}}}\textcolor{BrickRed}{;} \\
\mbox{}\textbf{\textcolor{RoyalBlue}{\ \ \ \ \#endif}} \\
\mbox{}\ \ \ \ cout\ \textcolor{BrickRed}{\textless{}\textless{}}\ solucion\textcolor{BrickRed}{.}valor\ \textcolor{BrickRed}{\textless{}\textless{}}\ endl\textcolor{BrickRed}{;} \\
\mbox{}\textcolor{Red}{\}} \\
\mbox{}
}
        \normalsize
  \subsection{El problema de la asignación cuadrática (QAP)}
        \small
  	%\texttt{\input{terminales.tex}}
        \normalsize

    
\end{document}