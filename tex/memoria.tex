%%%%
% Modificación de una plantilla de Latex para adaptarla al castellano.
%%%

%%%%%%%%%%%%%%%%%%%%%%%%%%%%%%%%%%%%%%%%%
% Thin Sectioned Essay
% LaTeX Template
% Version 1.0 (3/8/13)
%
% This template has been downloaded from:
% http://www.LaTeXTemplates.com
%
% Original Author:
% Nicolas Diaz (nsdiaz@uc.cl) with extensive modifications by:
% Vel (vel@latextemplates.com)
%
% License:
% CC BY-NC-SA 3.0 (http://creativecommons.org/licenses/by-nc-sa/3.0/)
%
%%%%%%%%%%%%%%%%%%%%%%%%%%%%%%%%%%%%%%%%%

%----------------------------------------------------------------------------------------
%	PACKAGES AND OTHER DOCUMENT CONFIGURATIONS
%----------------------------------------------------------------------------------------

\documentclass[a4paper, 11pt]{article} % Font size (can be 10pt, 11pt or 12pt) and paper size (remove a4paper for US letter paper)

\usepackage[protrusion=true,expansion=true]{microtype} % Better typography
\usepackage{graphicx} % Required for including pictures
\usepackage[usenames,dvipsnames]{color} % Coloring code
\usepackage{wrapfig} % Allows in-line images
\usepackage[utf8]{inputenc}

% Imágenes
\usepackage{graphicx} 

\usepackage{amsmath}
% para importar svg
%\usepackage[generate=all]{svgfig}

% sudo apt-get install texlive-lang-spanish
\usepackage[spanish]{babel} % English language/hyphenation
\selectlanguage{spanish}
% Hay que pelearse con babel-spanish para el alineamiento del punto decimal
\decimalpoint
\usepackage{dcolumn}
\newcolumntype{d}[1]{D{.}{\esperiod}{#1}}
\makeatletter
\addto\shorthandsspanish{\let\esperiod\es@period@code}
\makeatother

\usepackage{longtable}
\usepackage{tabu}
\usepackage{supertabular}

\usepackage{multicol}
\newsavebox\ltmcbox

% Para algoritmos
\usepackage{algorithm}
\usepackage{algorithmic}
\usepackage{amsthm}
\floatname{algorithm}{Algoritmo}
\renewcommand{\listalgorithmname}{Lista de algoritmos}
\renewcommand{\algorithmicrequire}{\textbf{Entrada:}}
\renewcommand{\algorithmicensure}{\textbf{Salida:}}
\renewcommand{\algorithmicend}{\textbf{fin}}
\renewcommand{\algorithmicif}{\textbf{si}}
\renewcommand{\algorithmicthen}{\textbf{entonces}}
\renewcommand{\algorithmicelse}{\textbf{en otro caso}}
\renewcommand{\algorithmicelsif}{\algorithmicelse,\ \algorithmicif}
\renewcommand{\algorithmicendif}{\algorithmicend\ \algorithmicif}
\renewcommand{\algorithmicfor}{\textbf{para }}
\renewcommand{\algorithmicforall}{\textbf{para cada}}
\renewcommand{\algorithmicdo}{\textbf{}}
\renewcommand{\algorithmicendfor}{\algorithmicend\ \algorithmicfor}
\renewcommand{\algorithmicwhile}{\textbf{mientras}}
\renewcommand{\algorithmicendwhile}{\algorithmicend\ \algorithmicwhile}
\renewcommand{\algorithmicloop}{\textbf{repetir}}
\renewcommand{\algorithmicendloop}{\algorithmicend\ \algorithmicloop}
\renewcommand{\algorithmicrepeat}{\textbf{repetir}}
\renewcommand{\algorithmicuntil}{\textbf{hasta que}}
\renewcommand{\algorithmicprint}{\textbf{imprimir}} 
\renewcommand{\algorithmicreturn}{\textbf{devolver}} 
\renewcommand{\algorithmictrue}{\textbf{true }} 
\renewcommand{\algorithmicfalse}{\textbf{false }} 
\renewcommand{\algorithmicand}{\textbf{y}}
\renewcommand{\algorithmicor}{\textbf{o}}


% Para matrices
\usepackage{amsmath}

% Símbolos matemáticos
\usepackage{amssymb}
\let\oldemptyset\emptyset
\let\emptyset\varnothing

\usepackage[section]{placeins} % Para gráficas en su sección.
\usepackage{mathpazo} % Use the Palatino font
\usepackage[T1]{fontenc} % Required for accented characters
\newenvironment{allintypewriter}{\ttfamily}{\par}
\setlength{\parindent}{0pt}
\parskip=8pt
\linespread{1.05} % Change line spacing here, Palatino benefits from a slight increase by default

\makeatletter
\renewcommand\@biblabel[1]{\textbf{#1.}} % Change the square brackets for each bibliography item from '[1]' to '1.'
\renewcommand{\@listI}{\itemsep=0pt} % Reduce the space between items in the itemize and enumerate environments and the bibliography
\newcommand{\imagen}[2]{\begin{center} \includegraphics[width=90mm]{#1} \\#2 \end{center}}

\renewcommand{\maketitle}{ % Customize the title - do not edit title and author name here, see the TITLE block below
\begin{flushright} % Right align
    {\LARGE\@title} % Increase the font size of the title
    
    \vspace{50pt} % Some vertical space between the title and author name
    
    {\large\@author} % Author name
    \\\@date % Date
    
    \vspace{40pt} % Some vertical space between the author block and abstract
\end{flushright}
}


%Basado en: http://en.wikibooks.org/wiki/LaTeX/Theorems
\usepackage{amsthm}
\newtheorem*{mydef}{Definición}
\newtheorem{mydefn}{Definición}
\newtheorem{theorem}{Teorema}
\everymath{\displaystyle} % Displaystyle por defecto

%----------------------------------------------------------------------------------------
%	TITLE
%----------------------------------------------------------------------------------------

\title{\textbf{Práctica 4}\\ % Title
Backtracking y Branch \& Bound} % Subtitle

\author{\textsc{Óscar Bermúdez,\\Francisco David Charte,\\Ignacio Cordón,\\José Carlos Entrena,\\Mario Román} % Author
\\{\textit{Universidad de Granada}}} % Institution

\date{\today} % Date

%----------------------------------------------------------------------------------------

\begin{document}
    
    \maketitle % Print the title section
    
    \renewcommand{\abstractname}{Resumen} % Uncomment to change the name of the abstract to something else
    \begin{abstract}
      Estudio de algoritmos de backtracking y branch\&bound para distintos problemas de optimización.
      Se explican espacios de soluciones, cálculo de cotas, estrategias de ramificación, poda y vuelta atrás;
      así como algoritmos concretos con sus correspondientes implementaciones en pseudocódigo.
    \end{abstract}
    {\parskip=2pt
    \tableofcontents
    }
    \pagebreak
    
    
    \section{El problema de la mochila}
      \subsection{Enunciado}
	El problema de la mochila es un problema de maximización, en el que tenemos un contenedor con un tamaño 
	fijo $M$ y un conjunto de $n$ elementos que queremos almacenar en nuestro contenedor. Dichos elementos tendrán un peso $\omega_i$
	determinado, y obtendremos un cierto beneficio $b_i$ por cada objeto almacenado.
	
	Nuestro objetivo es maximizar la suma de beneficios,
	\begin{equation}
	 \sum_{i=1}^{n} b_i\delta_i
	\end{equation}
	(siendo $\delta_i = 1$ si el i-ésimo objeto está en la mochila, 0 en caso contrario) con la restricción de que la mochila solo puede llevar un peso máximo, determinado por 
	su tamaño:
	\begin{equation}
	  \sum_{i=1}^{n} \omega_i\delta_i \leq M.
	\end{equation}

    
      \subsection{Backtracking}
	Representamos una mochila como un vector de datos booleanos, $\delta$, de tamaño $n$. Esto implica 
	que nuestro espacio de soluciones tendrá un tamaño de $2^n$. La función de poda toma un vector 
	$\delta = (\delta_1, \dots, \delta_n)$ y devuelve falso en caso de que $\sum_{i=1}^{n} \omega_i\delta_i > M$, 
	asegurando así que solo se van considerando en el proceso candidatas a solución.
    
	\newpage
    
	\subsubsection{Algoritmo}
	  Nuestro algoritmo es de tipo iterativo, por lo que usa una pila en la que almacenamos las mochilas que 
	  vamos procesando. La poda se produce cuando al añadir un objeto, la mochila resultante supera el peso 
	  límite, en cuyo caso, esta no es añadida a la pila de candidatas. Cuando encontremos una mochila completa, 
	  se compara con la solución actual, y se sustituye en caso de mejorarla. 
    
	  \small
	    \texttt{% Generator: GNU source-highlight, by Lorenzo Bettini, http://www.gnu.org/software/src-highlite
\noindent
\mbox{}\textbf{\textcolor{Blue}{def}}\ mochila\textcolor{BrickRed}{(}capacidad\textcolor{BrickRed}{,}\ pesos\textcolor{BrickRed}{,}\ beneficios\textcolor{BrickRed}{)} \\
\mbox{}\ \ \ \ solucion\ \textcolor{BrickRed}{=}\ \textcolor{BrickRed}{[}\textbf{\textcolor{Blue}{false}}\textcolor{BrickRed}{,}\dots\textcolor{BrickRed}{,}\textbf{\textcolor{Blue}{false}}\textcolor{BrickRed}{]} \\
\mbox{}\ \ \ \ posibles$\_$mochilas\textcolor{BrickRed}{.}push\textcolor{BrickRed}{([])} \\
\mbox{}\ \ \ \ \textbf{\textcolor{Blue}{while}}\ \textcolor{BrickRed}{(}\textbf{\textcolor{Blue}{not}}\ posibles$\_$mochilas\textcolor{BrickRed}{.}empty?\textcolor{BrickRed}{)} \\
\mbox{}\ \ \ \ \ \ \ \ actual\ \textcolor{BrickRed}{=}\ posibles$\_$mochilas\textcolor{BrickRed}{.}pop \\
\mbox{}\ \ \ \ \ \ \ \ \textbf{\textcolor{Blue}{if}}\ \textcolor{BrickRed}{(}\#actual\ \textcolor{BrickRed}{==}\ \#pesos\textcolor{BrickRed}{)} \\
\mbox{}\ \ \ \ \ \ \ \ \ \ \ \ \textbf{\textcolor{Blue}{if}}\ \textcolor{BrickRed}{(}beneficio\textcolor{BrickRed}{(}actual\textcolor{BrickRed}{)}\ \textcolor{BrickRed}{\textgreater{}}\ beneficio\textcolor{BrickRed}{(}solucion\textcolor{BrickRed}{))} \\
\mbox{}\ \ \ \ \ \ \ \ \ \ \ \ \ \ \ \ solucion\ \textcolor{BrickRed}{=}\ actual \\
\mbox{}\ \ \ \ \ \ \ \ \textbf{\textcolor{Blue}{else}}\  \\
\mbox{}\ \ \ \ \ \ \ \ \ \ \ \ con$\_$nuevo\ \textcolor{BrickRed}{=}\ sin$\_$nuevo\ \textcolor{BrickRed}{=}\ actual \\
\mbox{}\ \ \ \ \ \ \ \ \ \ \ \ con$\_$nuevo\textcolor{BrickRed}{.}push\textcolor{BrickRed}{(}\textbf{\textcolor{Blue}{true}}\textcolor{BrickRed}{)} \\
\mbox{}\ \ \ \ \ \ \ \ \ \ \ \ sin$\_$nuevo\textcolor{BrickRed}{.}push\textcolor{BrickRed}{(}\textbf{\textcolor{Blue}{false}}\textcolor{BrickRed}{)} \\
\mbox{}\ \ \ \ \ \ \ \ \ \ \ \ \textbf{\textcolor{Blue}{if}}\ \textcolor{BrickRed}{(}peso\textcolor{BrickRed}{(}con$\_$nuevo\textcolor{BrickRed}{)}\ \textcolor{BrickRed}{\textless{}=}\ capacidad\textcolor{BrickRed}{)} \\
\mbox{}\ \ \ \ \ \ \ \ \ \ \ \ \ \ \ \ posibles$\_$mochilas\textcolor{BrickRed}{.}push\textcolor{BrickRed}{(}con$\_$nuevo\textcolor{BrickRed}{)} \\
\mbox{}\ \ \ \ \ \ \ \ \ \ \ \ posibles$\_$mochilas\textcolor{BrickRed}{.}push\textcolor{BrickRed}{(}sin$\_$nuevo\textcolor{BrickRed}{)} \\
\mbox{}\ \ \ \ \ \ \ \  \\
\mbox{}\ \ \ \ \textbf{\textcolor{Blue}{return}}\ solucion \\
\mbox{}\textbf{\textcolor{Blue}{end}} \\
\mbox{} \\
\mbox{}\textbf{\textcolor{Blue}{def}}\ beneficio\textcolor{BrickRed}{(}mochila\textcolor{BrickRed}{)} \\
\mbox{}\ \ \ \ \textbf{\textcolor{Blue}{return}}\ $\mathtt{\sum_{i \in \mathbb{Z}_n} mochila_i * beneficio_i}$ \\ \\
\mbox{}\textbf{\textcolor{Blue}{end}} \\
\mbox{} \\
\mbox{}\textbf{\textcolor{Blue}{def}}\ peso\textcolor{BrickRed}{(}mochila\textcolor{BrickRed}{)} \\
\mbox{}\ \ \ \ \textbf{\textcolor{Blue}{return}}\ $\mathtt{\sum_{i \in \mathbb{Z}_n} mochila_i * peso_i}$ \\
\mbox{}\textbf{\textcolor{Blue}{end}} \\
\mbox{} \\
\mbox{}\ \ \ \ \ \ \ \ \ \ \ \  \\
\mbox{}
}
	  \normalsize
	  
	\subsubsection{Árbol de Estados}
	Se incluye el árbol de estados para el ejemplo proporcionado en el guión. 
	\begin{itemize}
	\item $M = 61$
	\item $n = 5$
	\item $Pesos = (1, 11, 21, 23, 33)$
	\item $Beneficios = (11, 21, 31, 33, 43)$
	\end{itemize}
	 \tiny
	   	\texttt{\\
\\0 
\begin{itemize}
\item 0 0 


\begin{itemize}
\item 0 0 0 

\begin{itemize}
\item 0 0 0 0 

\begin{itemize}
\item 0 0 0 0 0 
\item 0 0 0 0 1
\end{itemize}
\item 0 0 0 1 


\begin{itemize}
\item 0 0 0 1 0 
\item 0 0 0 1 1
\end{itemize}
\end{itemize}
\item 0 0 1 


\begin{itemize}
\item 0 0 1 0 


\begin{itemize}
\item 0 0 1 0 0 
\item 0 0 1 0 1
\end{itemize}
\item 0 0 1 1 


\begin{itemize}
\item 0 0 1 1 0
\end{itemize}
\end{itemize}
\end{itemize}
\item 0 1 


\begin{itemize}
\item 0 1 0 


\begin{itemize}
\item 0 1 0 0 


\begin{itemize}
\item 0 1 0 0 0 
\item 0 1 0 0 1
\end{itemize}
\item 0 1 0 1 


\begin{itemize}
\item 0 1 0 1 0
\end{itemize}
\end{itemize}
\item 0 1 1 


\begin{itemize}
\item 0 1 1 0 


\begin{itemize}
\item 0 1 1 0 0
\end{itemize}
\item 0 1 1 1 


\begin{itemize}
\item 0 1 1 1 0
\end{itemize}
\end{itemize}
\end{itemize}
\end{itemize}
1 

\begin{itemize}
\item 1 0 


\begin{itemize}
\item 1 0 0 


\begin{itemize}
\item 1 0 0 0 


\begin{itemize}
\item 1 0 0 0 0 
\item 1 0 0 0 1
\end{itemize}
\item 1 0 0 1 


\begin{itemize}
\item 1 0 0 1 0 
\item 1 0 0 1 1
\end{itemize}
\end{itemize}
\item 1 0 1 


\begin{itemize}
\item 1 0 1 0 


\begin{itemize}
\item 1 0 1 0 0 
\item 1 0 1 0 1
\end{itemize}
\item 1 0 1 1 


\begin{itemize}
\item 1 0 1 1 0
\end{itemize}
\end{itemize}
\end{itemize}
\item 1 1 


\begin{itemize}
\item 1 1 0 


\begin{itemize}
\item 1 1 0 0 


\begin{itemize}
\item 1 1 0 0 0 
\item 1 1 0 0 1
\end{itemize}
\item 1 1 0 1 


\begin{itemize}
\item 1 1 0 1 0
\end{itemize}
\end{itemize}
\item 1 1 1 


\begin{itemize}
\item 1 1 1 0 


\begin{itemize}
\item 1 1 1 0 0
\end{itemize}
\item 1 1 1 1 


\begin{itemize}
\item 1 1 1 1 0
\end{itemize}
\end{itemize}
\end{itemize}
\end{itemize}

}
	 \normalsize
    
    
	\subsection{Branch \& Bound}
	  Para la versión Branch \& Bound del algoritmo, usamos la misma representación para la solución, 
	  un vector de booleanos, y por tanto tenemos el mismo espacio de soluciones. En este caso, un nodo 
	  intermedio será del tipo $(\delta_1, \dots, \delta_k)$, y sus hijos serán $(\delta_1, \dots, \delta_k, 1)$ 
	  y $(\delta_1, \dots, \delta_k, 0)$.
	  
	  Se realiza un cálculo del máximo beneficio a partir de un nodo intermedio $(\delta_1, \dots, \delta_k)$. 
	  Para hallar esta cota, tomamos el peso que queda por llenar y le añadimos el peso de los objetos que tengan
	  la mejor proporción \textit{beneficio}/\textit{peso}, hasta completar la capacidad de la mochila, y comparamos
	  el beneficio asociado a estos objetos con la mejor solución obtenida hasta ahora.
	  
	  La estrategia de poda que se sigue es podar cada rama en la que se encuentre una beneficio máximo (cota superior) 
	  menor que el beneficio de la mejor solución encontrada hasta el momento. La estrategia de ramificación utilizada es 
	  Maximum-Benefit FIFO, ya que usamos una cola con prioridad ordenada por el mayor beneficio.
      
	  \subsubsection{Algoritmo}
	    
	    \small
	    \texttt{% Generator: GNU source-highlight, by Lorenzo Bettini, http://www.gnu.org/software/src-highlite
\noindent
\mbox{}\textbf{\textcolor{Blue}{def}}\ mochila\textcolor{BrickRed}{(}capacidad\textcolor{BrickRed}{,}\ pesos\textcolor{BrickRed}{,}\ beneficios\textcolor{BrickRed}{)} \\
\mbox{}\ \ \ \ \textbf{\textcolor{Blue}{def}}\ generar\textcolor{BrickRed}{()} \\
\mbox{}\ \ \ \ \ \ \ \ posibles$\_$mochilas\ \textcolor{BrickRed}{=}\ new\ Queue \\
\mbox{}\ \ \ \ \ \ \ \ posibles$\_$mochilas\textcolor{BrickRed}{.}push\textcolor{BrickRed}{([])} \\
\mbox{} \\
\mbox{}\ \ \ \ \ \ \ \ \textbf{\textcolor{Blue}{while}}\ \textcolor{BrickRed}{(}\textbf{\textcolor{Blue}{not}}\ posibles$\_$mochilas\textcolor{BrickRed}{.}empty?\textcolor{BrickRed}{)} \\
\mbox{}\ \ \ \ \ \ \ \ \ \ \ \ actual\ \textcolor{BrickRed}{=}\ posibles$\_$mochilas\textcolor{BrickRed}{.}pop \\
\mbox{}\ \ \ \ \ \ \ \ \ \ \ \ \textbf{\textcolor{Blue}{if}}\ \textcolor{BrickRed}{(}actual\ \textcolor{BrickRed}{==}\ n\textcolor{BrickRed}{)} \\
\mbox{}\ \ \ \ \ \ \ \ \ \ \ \ \ \ \ \ \textbf{\textcolor{Blue}{if}}\ \textcolor{BrickRed}{(}beneficio\textcolor{BrickRed}{(}actual\textcolor{BrickRed}{)}\ \textcolor{BrickRed}{\textgreater{}}\ beneficio\textcolor{BrickRed}{(}solucion\textcolor{BrickRed}{))} \\
\mbox{}\ \ \ \ \ \ \ \ \ \ \ \ \ \ \ \ \ \ \ \ solucion\ \textcolor{BrickRed}{=}\ actual \\
\mbox{}\ \ \ \ \ \ \ \ \ \ \ \ \textbf{\textcolor{Blue}{else}} \\
\mbox{}\ \ \ \ \ \ \ \ \ \ \ \ \ \ \ \ \textbf{\textcolor{Blue}{if}}\ \textcolor{BrickRed}{(}peso\textcolor{BrickRed}{(}actual\ \textcolor{BrickRed}{+}\ \textcolor{BrickRed}{[}\textbf{\textcolor{Blue}{true}}\textcolor{BrickRed}{])}\ \textcolor{BrickRed}{$\le$}\ capacidad\textcolor{BrickRed}{)} \\
\mbox{}\ \ \ \ \ \ \ \ \ \ \ \ \ \ \ \ \ \ \ \ posibles$\_$mochilas\textcolor{BrickRed}{.}push\textcolor{BrickRed}{(}actual\ \textcolor{BrickRed}{+}\ \textcolor{BrickRed}{[}\textbf{\textcolor{Blue}{true}}\textcolor{BrickRed}{])} \\
\mbox{}\ \ \ \ \ \ \ \ \ \ \ \ \ \ \ \ \textbf{\textcolor{Blue}{if}}\ \textcolor{BrickRed}{(}beneficio\textcolor{BrickRed}{(}solucion\textcolor{BrickRed}{)}\ \textcolor{BrickRed}{\textless{}}\\
\mbox{}\ \ \ \ \ \ \ \ \ \ \ \ \ \ \ \ \ \ \ \ beneficio\textcolor{BrickRed}{(}actual\textcolor{BrickRed}{)}\ \textcolor{BrickRed}{+}\ calculaCota\textcolor{BrickRed}{(}actual\ \textcolor{BrickRed}{+}\ \textcolor{BrickRed}{[}\textbf{\textcolor{Blue}{false}}\textcolor{BrickRed}{]))} \\
\mbox{}\ \ \ \ \ \ \ \ \ \ \ \ \ \ \ \ \ \ \ \ posibles$\_$mochilas\textcolor{BrickRed}{.}push\textcolor{BrickRed}{(}actual\ \textcolor{BrickRed}{+}\ \textcolor{BrickRed}{[}\textbf{\textcolor{Blue}{false}}\textcolor{BrickRed}{])} \\
\mbox{} \\
\mbox{}\ \ \ \ \ \ \ \ \textbf{\textcolor{Blue}{return}}\ solucion \\
\mbox{}\ \ \ \ \textbf{\textcolor{Blue}{end}} \\
\mbox{} \\
\mbox{}\ \ \ \ \textbf{\textcolor{Blue}{def}}\ beneficio\textcolor{BrickRed}{(}nodo\textcolor{BrickRed}{)} \\
\mbox{}\ \ \ \ \ \ \ \ \textbf{\textcolor{Blue}{return}}\ $\sum_{i=0}^{\#beneficios} nodo[i]\times beneficios[i]$ \\
\mbox{}\ \ \ \ \textbf{\textcolor{Blue}{end}} \\
\mbox{} \\
\mbox{}\ \ \ \ \textbf{\textcolor{Blue}{def}}\ peso\textcolor{BrickRed}{(}nodo\textcolor{BrickRed}{)} \\
\mbox{}\ \ \ \ \ \ \ \ \textbf{\textcolor{Blue}{return}}\ $\sum_{i=0}^{\#pesos} nodo[i]\times pesos[i]$ \\
\mbox{}\ \ \ \ \textbf{\textcolor{Blue}{end}} \\
\mbox{} \\
\mbox{}\ \ \ \ \textbf{\textcolor{Blue}{def}}\ calculaCota\textcolor{BrickRed}{(}nodo\textcolor{BrickRed}{)} \\
\mbox{}\ \ \ \ \ \ \ \ w\ \textcolor{BrickRed}{=}\ $ \left\{\frac{beneficios[i+\#nodo]}{pesos[i+\#nodo]} : i \in [0..\#pesos]\right\} $ \\
\mbox{}\ \ \ \ \ \ \ \ beneficio$\_$extra\ \textcolor{BrickRed}{=}\ \textcolor{Purple}{0} \\
\mbox{}\ \ \ \ \ \ \ \ restante\ \textcolor{BrickRed}{=}\ capacidad\ \textcolor{BrickRed}{-}\ peso\textcolor{BrickRed}{(}nodo\textcolor{BrickRed}{)} \\
\mbox{} \\
\mbox{}\ \ \ \ \ \ \ \ \textbf{\textcolor{Blue}{while}}\ \textcolor{BrickRed}{(}restante\ \textcolor{BrickRed}{\textgreater{}}\ \textcolor{Purple}{0}\ \textbf{\textcolor{Blue}{and}}\ \textcolor{BrickRed}{!}w\textcolor{BrickRed}{.}empty\textcolor{BrickRed}{())} \\
\mbox{}\ \ \ \ \ \ \ \ \ \ \ \ restante\ \textcolor{BrickRed}{-=}\ pesos\textcolor{BrickRed}{[}w\textcolor{BrickRed}{.}max\textcolor{BrickRed}{()]} \\
\mbox{}\ \ \ \ \ \ \ \ \ \ \ \ beneficio$\_$extra\ \textcolor{BrickRed}{+=}\ beneficios\textcolor{BrickRed}{[}w\textcolor{BrickRed}{.}max\textcolor{BrickRed}{()]} \\
\mbox{}\ \ \ \ \ \ \ \ \ \ \ \ w\textcolor{BrickRed}{.}delete$\_$max\textcolor{BrickRed}{()} \\
\mbox{} \\
\mbox{}\ \ \ \ \ \ \ \ \textbf{\textcolor{Blue}{return}}\ beneficio$\_$extra \\
\mbox{}\ \ \ \ \textbf{\textcolor{Blue}{end}} \\
\mbox{} \\
\mbox{}\ \ \ \ n\ \textcolor{BrickRed}{=}\ \#pesos \\
\mbox{}\ \ \ \ solucion\ \textcolor{BrickRed}{=}\ \textcolor{BrickRed}{[}\textbf{\textcolor{Blue}{false}}\textcolor{BrickRed}{,}\ \textcolor{BrickRed}{...,}\ \textbf{\textcolor{Blue}{false}}\textcolor{BrickRed}{]} \\
\mbox{} \\
\mbox{}\ \ \ \ \textbf{\textcolor{Blue}{return}}\ generar\textcolor{BrickRed}{()} \\
\mbox{} \\
\mbox{}\textbf{\textcolor{Blue}{end}} \\
\mbox{}
}
	    \normalsize

	\subsubsection{Árbol de Estados}
	Se incluye el árbol de estados para el ejemplo proporcionado en el guión. 
    
    \tiny
	   	\texttt{\\\\
1 


\begin{itemize}
\item 1 1 


\begin{itemize}
\item 1 1 1 


\begin{itemize}
\item 1 1 1 1 


\begin{itemize}
\item 1 1 1 1 0
\end{itemize}
\item 1 1 1 0
\end{itemize}
\item 1 1 0 


\begin{itemize}
\item 1 1 0 1
\end{itemize}
\end{itemize}
\item 1 0 


\begin{itemize}
\item 1 0 1 


\begin{itemize}
\item 1 0 1 1
\end{itemize}
\end{itemize}
\end{itemize}
0 


\begin{itemize}
\item 0 1 


\begin{itemize}
\item 0 1 1 


\begin{itemize}
\item 0 1 1 1
\end{itemize}
\item 0 1 0 


\begin{itemize}
\item 0 1 0 1
\end{itemize}
\end{itemize}
\item 0 0 


\begin{itemize}
\item 0 0 1 


\begin{itemize}
\item 0 0 1 1
\end{itemize}
\end{itemize}
\end{itemize}
}
    \normalsize
        
    
    \section{Traveling Salesman Problem}
      \subsection{Enunciado}
	Dada una lista $S$ de $n$ ciudades, representadas como puntos en el plano:
	\begin{equation}
	    S = [(x_0,y_0), (x_1,y_1), \dots (x_{n-1},y_{n-1})] \subset \mathbb{R}^2
	\end{equation}
	Y definiendo la longitud de recorrer una lista como la suma de las distancias de cada ciudad a la siguiente:
	\begin{equation}
	    long(S) = \sum_{i \in \mathbb{Z}_n} dist((x_i,y_i), (x_{i+1}, y_{i+1})) = \sum_{i \in \mathbb{Z}_n} \sqrt{(x_i-x_{i+1})^2 + (y_i-y_{i+1})^2}
	\end{equation}
	El objetivo es encontrar la permutación de la lista $\sigma : \mathbb{Z}_n \leftrightarrow \mathbb{Z}_n$, verificando que su longitud sea mínima:
	\begin{equation}
	    long(\sigma(S)) = long([(x_{\sigma(1)},y_{\sigma(1)}), (x_{\sigma(2)},y_{\sigma(2)}), \dots, (x_{\sigma(n)},y_{\sigma(n)})])
    \end{equation}
    
      \subsection{Backtracking}
	Utilizaremos un vector de enteros para representar una solución, donde almacenaremos los 
	índices de las ciudades que se van recorriendo en la ruta. Como fijamos la primera ciudad 
	y únicamente podemos pasar una vez por cada una, el espacio de soluciones tendrá un tamaño 
	de $(n-1)!$.
	La poda se realiza generando únicamente vectores que representen posibles soluciones, esto es, 
	solo se consideran permutaciones del vector inicial $(1,\dots,n)$. De esta forma descartamos el 
	resto de posibles combinaciones de los elementos. Diremos que, para un nodo $(i_0, \dots, i_k)$, 
	la función de poda da falso si se repite alguno de los índices, es decir, $\exists p\ne q : i_p = i_q$.
    
    \newpage
    
    \subsubsection{Algoritmo}
    
    Para aplicar backtracking al problema del TSP, utilizamos un algoritmo recursivo, el cual recibe la lista de 
    ciudades del problema, la ruta con las ciudades que llevamos hasta el momento, su coste, y el índice de la 
    última ciudad añadida. Calcularemos todas las ramificaciones posibles de la ruta actual, y tomaremos la mejor 
    de estas. 
    
    
    \small
    \texttt{% Generator: GNU source-highlight, by Lorenzo Bettini, http://www.gnu.org/software/src-highlite
\noindent
\mbox{}\textbf{\textcolor{Blue}{def}}\ tsp\textcolor{BrickRed}{(}ciudades\textcolor{BrickRed}{)} \\
\mbox{}\ \ \ \ \textbf{\textcolor{Blue}{def}}\ recorrer\textcolor{BrickRed}{(}indice\textcolor{BrickRed}{)} \\
\mbox{}\ \ \ \ \ \ \ \ \textbf{\textcolor{Blue}{if}}\ \textcolor{BrickRed}{(}indice\ \textcolor{BrickRed}{==}\ $\#$ciudades\textcolor{BrickRed}{)} \\
\mbox{}\ \ \ \ \ \ \ \ \ \ \ \ \textbf{\textcolor{Blue}{if}}\ \textcolor{BrickRed}{(}coste\textcolor{BrickRed}{(}ruta\textcolor{BrickRed}{)}\ \textcolor{BrickRed}{\textless{}}\ mejor$\_$coste\textcolor{BrickRed}{)} \\
\mbox{}\ \ \ \ \ \ \ \ \ \ \ \ \ \ \ \ mejor$\_$coste\ \textcolor{BrickRed}{=}\ coste\textcolor{BrickRed}{(}ruta\textcolor{BrickRed}{)} \\
\mbox{}\ \ \ \ \ \ \ \ \ \ \ \ \ \ \ \ mejor$\_$ruta\ \ \textcolor{BrickRed}{=}\ ruta \\
\mbox{}\ \ \ \ \ \ \ \ \textbf{\textcolor{Blue}{else}} \\
\mbox{}\ \ \ \ \ \ \ \ \ \ \ \ \textbf{\textcolor{Blue}{for}}\ i\ \textbf{\textcolor{Blue}{in}}\ \textcolor{BrickRed}{[}indice\textcolor{BrickRed}{..}$\#$ciudades\textcolor{BrickRed}{]} \\
\mbox{}\ \ \ \ \ \ \ \ \ \ \ \ \ \ \ \ swap\textcolor{BrickRed}{(}ruta\textcolor{BrickRed}{[}i\textcolor{BrickRed}{],}\ ruta\textcolor{BrickRed}{[}indice\textcolor{BrickRed}{])} \\
\mbox{}\ \ \ \ \ \ \ \ \ \ \ \ \ \ \ \ recorrer\textcolor{BrickRed}{(}indice\textcolor{BrickRed}{+}\textcolor{Purple}{1}\textcolor{BrickRed}{)} \\
\mbox{}\ \ \ \ \ \ \ \ \ \ \ \ \ \ \ \ swap\textcolor{BrickRed}{(}ruta\textcolor{BrickRed}{[}indice\textcolor{BrickRed}{],}\ ruta\textcolor{BrickRed}{[}i\textcolor{BrickRed}{])} \\
\mbox{}\ \ \ \ \textbf{\textcolor{Blue}{end}} \\
\mbox{} \\
\mbox{}\ \ \ \ mejor$\_$coste\ \textcolor{BrickRed}{=}\ $\infty$ \\
\mbox{}\ \ \ \ ruta\ \textcolor{BrickRed}{=}\ \textcolor{BrickRed}{[}\textcolor{Purple}{1}\textcolor{BrickRed}{..}$\#$ciudades\textcolor{BrickRed}{]} \\
\mbox{}\ \ \ \ recorrer\textcolor{BrickRed}{(}\textcolor{Purple}{1}\textcolor{BrickRed}{)} \\
\mbox{} \\
\mbox{}\ \ \ \ \textbf{\textcolor{Blue}{return}}\ mejor$\_$coste \\
\mbox{}\textbf{\textcolor{Blue}{end}} \\
\mbox{} \\
\mbox{}\textbf{\textcolor{Blue}{def}}\ coste\textcolor{BrickRed}{(}ruta\textcolor{BrickRed}{)} \\
\mbox{}\ \ \ \ \textbf{\textcolor{Blue}{return}}\ $\mathtt{\sum_{i \in \mathbb{Z}_n} dist(ciudades[ruta[i]], ciudades[ruta[i+1]])}$ \\
\mbox{}\textbf{\textcolor{Blue}{end}} \\
\mbox{}
}
    \normalsize
    
    \subsection{Branch \& Bound}
      De nuevo, representaremos una solución como un vector de enteros de tamaño $n$, obteniendo el mismo espacio de soluciones de tamaño $(n-1)!$.
      
      Dado un nodo intermedio $(i_0, \dots, i_k)$, los posibles hijos serán del tipo\\ $(i_0, \dots, i_k, i_{k+1})$ con $i_{k+1} \notin \{i_0, \dots, i_k\}$, 
      ya que no aceptaremos repeticiones de ciudades (para evitar generar vectores que no sean permutaciones).
      
      La cota que se utiliza para la poda es el cálculo del coste (la suma de distancias entre ciudades) hasta el momento. Este coste resulta ser una cota 
      mínima del coste de cualquier posible solución que se obtenga del nodo actual.
      La poda se realiza en el momento en que la cota mínima supera al coste de la mejor solución hallada hasta el momento.
      
      La estrategia de ramificación utilizada en este caso será Least-Cost FIFO, ya que buscamos explorar las soluciones ordenadas por el menor coste.
      
      \subsubsection{Algoritmo}
      
      \small
      \texttt{% Generator: GNU source-highlight, by Lorenzo Bettini, http://www.gnu.org/software/src-highlite
\noindent
\mbox{}\textbf{\textcolor{Blue}{def}}\ tsp\textcolor{BrickRed}{(}ciudades\textcolor{BrickRed}{)} \\
\mbox{}\ \ \ \ \textbf{\textcolor{Blue}{def}}\ recorrer\textcolor{BrickRed}{()} \\
\mbox{}\ \ \ \ \ \ \ \ mejor\ \textcolor{BrickRed}{=}\ inicial \\
\mbox{}\ \ \ \ \ \ \ \ actual\ \textcolor{BrickRed}{=}\ posibles\textcolor{BrickRed}{.}pop \\
\mbox{} \\
\mbox{}\ \ \ \ \ \ \ \ \textbf{\textcolor{Blue}{if}}\ \textcolor{BrickRed}{(}indice\ \textcolor{BrickRed}{==}\ \#ciudades\textcolor{BrickRed}{)} \\
\mbox{}\ \ \ \ \ \ \ \ \ \ \ \ \textbf{\textcolor{Blue}{if}}\ \textcolor{BrickRed}{(}coste\textcolor{BrickRed}{(}actual\textcolor{BrickRed}{)}\ \textcolor{BrickRed}{\textless{}}\ coste\textcolor{BrickRed}{(}mejor\textcolor{BrickRed}{))} \\
\mbox{}\ \ \ \ \ \ \ \ \ \ \ \ \ \ \ \ mejor\ \textcolor{BrickRed}{=}\ actual \\
\mbox{} \\
\mbox{}\ \ \ \ \ \ \ \ \textbf{\textcolor{Blue}{else}}\ \textbf{\textcolor{Blue}{if}}\ \textcolor{BrickRed}{(}cota$\_$inferior\textcolor{BrickRed}{(}actual\textcolor{BrickRed}{)}\ \textcolor{BrickRed}{\textless{}}\ coste\textcolor{BrickRed}{(}mejor\textcolor{BrickRed}{))} \\
\mbox{}\ \ \ \ \ \ \ \ \ \ \ \ \textbf{\textcolor{Blue}{for}}\ i\ \textbf{\textcolor{Blue}{in}}\ \textcolor{BrickRed}{[}actual\textcolor{BrickRed}{.}indice\ \textcolor{BrickRed}{...}\ \#ciudades\textcolor{BrickRed}{]} \\
\mbox{}\ \ \ \ \ \ \ \ \ \ \ \ \ \ \ \ hija\ \textcolor{BrickRed}{=}\ actual \\
\mbox{}\ \ \ \ \ \ \ \ \ \ \ \ \ \ \ \ swap\textcolor{BrickRed}{(}hija\textcolor{BrickRed}{[}i\textcolor{BrickRed}{],}\ hija\textcolor{BrickRed}{[}hija\textcolor{BrickRed}{.}indice\textcolor{BrickRed}{])} \\
\mbox{}\ \ \ \ \ \ \ \ \ \ \ \ \ \ \ \ hija\textcolor{BrickRed}{.}indice\textcolor{BrickRed}{++} \\
\mbox{}\ \ \ \ \ \ \ \ \ \ \ \ \ \ \ \ posibles\textcolor{BrickRed}{.}push\textcolor{BrickRed}{(}hija\textcolor{BrickRed}{)} \\
\mbox{} \\
\mbox{}\ \ \ \ \ \ \ \ \textbf{\textcolor{Blue}{return}}\ mejor \\
\mbox{}\ \ \ \ \textbf{\textcolor{Blue}{end}} \\
\mbox{} \\
\mbox{}\ \ \ \ posibles\ \textcolor{BrickRed}{=}\ new\ Priority\ Queue \\
\mbox{}\ \ \ \ inicial\ \textcolor{BrickRed}{=}\ \textcolor{BrickRed}{[}\textcolor{Purple}{1}\ \textcolor{BrickRed}{...}\ \#ciudades\textcolor{BrickRed}{]} \\
\mbox{}\ \ \ \ inicial\textcolor{BrickRed}{.}indice\ \textcolor{BrickRed}{=}\ \textcolor{Purple}{1} \\
\mbox{}\ \ \ \ posibles\textcolor{BrickRed}{.}push\textcolor{BrickRed}{(}inicial\textcolor{BrickRed}{)} \\
\mbox{} \\
\mbox{}\ \ \ \ \textbf{\textcolor{Blue}{return}}\ recorrer\textcolor{BrickRed}{()} \\
\mbox{} \\
\mbox{}\ \ \ \ \textbf{\textcolor{Blue}{def}}\ cota$\_$inferior\textcolor{BrickRed}{(}ruta\textcolor{BrickRed}{)} \\
\mbox{}\ \ \ \ \ \ \ \ \textbf{\textcolor{Blue}{return}}\ coste\ \textcolor{BrickRed}{+}\ $\mathtt{\sum_{i=1}^{\#ciudades} \min_{j=1}^{\#ciudades} dist(i,j)}$ \\
\mbox{}\ \ \ \ \textbf{\textcolor{Blue}{end}} \\
\mbox{} \\
\mbox{}\ \ \ \ \textbf{\textcolor{Blue}{def}}\ coste\textcolor{BrickRed}{(}ruta\textcolor{BrickRed}{)} \\
\mbox{}\ \ \ \ \ \ \ \ \textbf{\textcolor{Blue}{return}}\ $\mathtt{\sum_{i \in \mathbb{Z}_n} dist(ciudades[ruta[i]], ciudades[ruta[i+1]])}$ \\
\mbox{}\ \ \ \ \textbf{\textcolor{Blue}{end}} \\
\mbox{}\textbf{\textcolor{Blue}{end}} \\
\mbox{}
}
      \normalsize
      
      \subsubsection{Variante: Branch \& Bound con 2-OPT}
	Una condición para podar una rama, cuando se está realizando la exploración del árbol, puede ser que exista algún cruce de ciudades en el grafo.
	Entendemos un cruce de ciudades como un caso en el que al permutar dos caminos del problema (cambiando $(a,b),(c,d)$ por $(a,d),(c,b)$), obtenemos
	un menor coste.
	
	En ese caso se pueden descartar las soluciones resultantes, que no son óptimas porque no son mínimos locales del problema.
    
    \subsubsection{Algoritmo}
    
    \small
    \texttt{% Generator: GNU source-highlight, by Lorenzo Bettini, http://www.gnu.org/software/src-highlite
\noindent
\mbox{}\textbf{\textcolor{Blue}{def}}\ tsp\textcolor{BrickRed}{(}ciudades\textcolor{BrickRed}{)} \\
\mbox{}\ \ \ \ \textbf{\textcolor{Blue}{def}}\ recorrer\textcolor{BrickRed}{()} \\
\mbox{}\ \ \ \ \ \ \ \ mejor\ \textcolor{BrickRed}{=}\ inicial \\
\mbox{}\ \ \ \ \ \ \ \ actual\ \textcolor{BrickRed}{=}\ posibles\textcolor{BrickRed}{.}pop \\
\mbox{} \\
\mbox{}\ \ \ \ \ \ \ \ \textbf{\textcolor{Blue}{if}}\ \textcolor{BrickRed}{(}indice\ \textcolor{BrickRed}{==}\ \#ciudades\textcolor{BrickRed}{)} \\
\mbox{}\ \ \ \ \ \ \ \ \ \ \ \ \textbf{\textcolor{Blue}{if}}\ \textcolor{BrickRed}{(}coste\textcolor{BrickRed}{(}actual\textcolor{BrickRed}{)}\ \textcolor{BrickRed}{\textless{}}\ coste\textcolor{BrickRed}{(}mejor\textcolor{BrickRed}{))} \\
\mbox{}\ \ \ \ \ \ \ \ \ \ \ \ \ \ \ \ mejor\ \textcolor{BrickRed}{=}\ actual \\
\mbox{} \\
\mbox{}\ \ \ \ \ \ \ \ \textbf{\textcolor{Blue}{else}}\ \textbf{\textcolor{Blue}{if}}\ \textcolor{BrickRed}{(}cota$\_$inferior\textcolor{BrickRed}{(}actual\textcolor{BrickRed}{)}\ \textcolor{BrickRed}{\textless{}}\ coste\textcolor{BrickRed}{(}mejor\textcolor{BrickRed}{))} \\
\mbox{}\ \ \ \ \ \ \ \ \ \ \ \ \textbf{\textcolor{Blue}{for}}\ i\ \textbf{\textcolor{Blue}{in}}\ \textcolor{BrickRed}{[}actual\textcolor{BrickRed}{.}indice\ \textcolor{BrickRed}{...}\ \#ciudades\textcolor{BrickRed}{]} \\
\mbox{}\ \ \ \ \ \ \ \ \ \ \ \ \ \ \ \ \textbf{\textcolor{Blue}{if}}\ \textcolor{BrickRed}{(}\textbf{\textcolor{Blue}{not}}\ hay$\_$cruce\textcolor{BrickRed}{(}i\textcolor{BrickRed}{,}\ actual\textcolor{BrickRed}{.}indice\textcolor{BrickRed}{))} \\
\mbox{}\ \ \ \ \ \ \ \ \ \ \ \ \ \ \ \ \ \ \ \ hija\ \textcolor{BrickRed}{=}\ actual \\
\mbox{}\ \ \ \ \ \ \ \ \ \ \ \ \ \ \ \ \ \ \ \ swap\textcolor{BrickRed}{(}hija\textcolor{BrickRed}{[}i\textcolor{BrickRed}{],}\ hija\textcolor{BrickRed}{[}hija\textcolor{BrickRed}{.}indice\textcolor{BrickRed}{])} \\
\mbox{}\ \ \ \ \ \ \ \ \ \ \ \ \ \ \ \ \ \ \ \ hija\textcolor{BrickRed}{.}indice\textcolor{BrickRed}{++} \\
\mbox{}\ \ \ \ \ \ \ \ \ \ \ \ \ \ \ \ \ \ \ \ posibles\textcolor{BrickRed}{.}push\textcolor{BrickRed}{(}hija\textcolor{BrickRed}{)} \\
\mbox{} \\
\mbox{}\ \ \ \ \ \ \ \ \textbf{\textcolor{Blue}{return}}\ mejor \\
\mbox{}\ \ \ \ \textbf{\textcolor{Blue}{end}} \\
\mbox{} \\
\mbox{}\ \ \ \ posibles\ \textcolor{BrickRed}{=}\ new\ Priority\ Queue \\
\mbox{}\ \ \ \ inicial\ \textcolor{BrickRed}{=}\ \textcolor{BrickRed}{[}\textcolor{Purple}{1}\textcolor{BrickRed}{..}\#ciudades\textcolor{BrickRed}{]} \\
\mbox{}\ \ \ \ inicial\textcolor{BrickRed}{.}indice\ \textcolor{BrickRed}{=}\ \textcolor{Purple}{1} \\
\mbox{}\ \ \ \ posibles\textcolor{BrickRed}{.}push\textcolor{BrickRed}{(}inicial\textcolor{BrickRed}{)} \\
\mbox{}\ \ \ \  \\
\mbox{}\ \ \ \ \textbf{\textcolor{Blue}{return}}\ recorrer\textcolor{BrickRed}{()} \\
\mbox{} \\
\mbox{}\ \ \ \ \textbf{\textcolor{Blue}{def}}\ cota$\_$inferior\textcolor{BrickRed}{(}ruta\textcolor{BrickRed}{)} \\
\mbox{}\ \ \ \ \ \ \ \ \textbf{\textcolor{Blue}{return}}\ coste\ \textcolor{BrickRed}{+}\ $\mathtt{\sum_{i=1}^{\#ciudades} \min_{j=1}^{\#ciudades} dist(i,j)}$ \\
\mbox{}\ \ \ \ \textbf{\textcolor{Blue}{end}} \\
\mbox{} \\
\mbox{}\ \ \ \ \textbf{\textcolor{Blue}{def}}\ coste\textcolor{BrickRed}{(}ruta\textcolor{BrickRed}{)} \\
\mbox{}\ \ \ \ \ \ \ \ \textbf{\textcolor{Blue}{return}}\ $\mathtt{\sum_{i \in \mathbb{Z}_n} dist(ciudades[ruta[i]], ciudades[ruta[i+1]])}$ \\
\mbox{}\ \ \ \ \textbf{\textcolor{Blue}{end}} \\
\mbox{} \\
\mbox{}\ \ \ \ \textbf{\textcolor{Blue}{def}}\ hay$\_$cruces\textcolor{BrickRed}{(}i\textcolor{BrickRed}{,}\ indice\textcolor{BrickRed}{)} \\
\mbox{}\ \ \ \ \ \ \ \ opt2\ \textcolor{BrickRed}{=}\ \textbf{\textcolor{Blue}{false}} \\
\mbox{}\ \ \ \ \ \ \ \ \ \ \ \ \textbf{\textcolor{Blue}{for}}\ j\ \textbf{\textcolor{Blue}{in}}\ \textcolor{BrickRed}{[}\textcolor{Purple}{1}\textcolor{BrickRed}{..}indice\textcolor{BrickRed}{]} \\
\mbox{}\ \ \ \ \ \ \ \ \ \ \ \ \ \ \ \ opt2\ \textcolor{BrickRed}{=}\ opt2\ \textcolor{BrickRed}{\&\&} \\
\mbox{}\ \ \ \ \ \ \ \ \ \ \ \ \ \ \ \ \ \ \ \ \textcolor{BrickRed}{(}dist\textcolor{BrickRed}{(}ruta\textcolor{BrickRed}{[}i\textcolor{BrickRed}{],}ruta\textcolor{BrickRed}{[}indice\textcolor{BrickRed}{])}\ \textcolor{BrickRed}{+}\ dist\textcolor{BrickRed}{(}ruta\textcolor{BrickRed}{[}j\textcolor{BrickRed}{],}ruta\textcolor{BrickRed}{[}j\textcolor{BrickRed}{-}\textcolor{Purple}{1}\textcolor{BrickRed}{])}\ \textcolor{BrickRed}{\textgreater{}} \\
\mbox{}\ \ \ \ \ \ \ \ \ \ \ \ \ \ \ \ \ \ \ \ dist\textcolor{BrickRed}{(}ruta\textcolor{BrickRed}{[}i\textcolor{BrickRed}{],}ruta\textcolor{BrickRed}{[}j\textcolor{BrickRed}{])}\ \textcolor{BrickRed}{+}\ dist\textcolor{BrickRed}{(}ruta\textcolor{BrickRed}{[}indice\textcolor{BrickRed}{],}ruta\textcolor{BrickRed}{[}j\textcolor{BrickRed}{-}\textcolor{Purple}{1}\textcolor{BrickRed}{]))} \\
\mbox{} \\
\mbox{}\ \ \ \ \ \ \ \ \textbf{\textcolor{Blue}{return}}\ opt2 \\
\mbox{}\ \ \ \ \textbf{\textcolor{Blue}{end}} \\
\mbox{}\textbf{\textcolor{Blue}{end}} \\
\mbox{}
}
    \normalsize
    
    \section{Planificación en multiprocesadores}
    \subsection{Enunciado}
    Tenemos un conjunto de $n$ tareas con un tiempo de ejecución $t_i$ asociado a cada una, existiendo una 
    relación de precedencia entre algunas tareas, de forma que unas han de ejecutarse antes que otras. 
    El objetivo es asignar las tareas entre $k$ procesadores, manteniendo las relaciones de dependencia,
    de manera que el tiempo de ejecución sea mínimo. 
    
    
    \subsection{Backtracking}
    En este caso, la representación de la solución se realiza mediante un vector de estructuras con los siguientes datos:
    \begin{itemize}
        \item Tarea asignada
        \item Tiempo de inicio
        \item Núcleo (\textit{core}) asignado
    \end{itemize}
    Además, podemos entender la \textit{tarea} como un tipo de dato que comprende tanto el tiempo de ejecución como referencias (por ejemplo, índices a un vector) de sus dependencias.
    
    Al tener esta representación de la solución, no es posible encontrar el tamaño del espacio de soluciones, pues el tiempo de inicio de cada tarea nos impide poder cuantificar dicho tamaño a partir del natural que representa la magnitud del problema. 
    
    La función de poda en este caso devuelve falso si la tarea que se va a añadir tiene una dependencia pendiente, esto es, incluye en sus dependencias a otra tarea que aún no se ha terminado de ejecutar.
    
    \subsubsection{Algoritmo}
    
      \small
      \texttt{% Generator: GNU source-highlight, by Lorenzo Bettini, http://www.gnu.org/software/src-highlite
\noindent
\mbox{}\textbf{\textcolor{Blue}{def}}\ planifica\textcolor{BrickRed}{(}ncores\textcolor{BrickRed}{,}\ tareas\textcolor{BrickRed}{)} \\
\mbox{}\ \ \ \ \textbf{\textcolor{Blue}{def}}\ resolver\textcolor{BrickRed}{()} \\
\mbox{}\ \ \ \ \ \ \ \ posibles\ \textcolor{BrickRed}{=}\ new\ Stack \\
\mbox{}\ \ \ \ \ \ \ \ solución\ \textcolor{BrickRed}{=}\ \textcolor{BrickRed}{[]} \\
\mbox{}\ \ \ \ \ \ \ \ tiempo\ \textcolor{BrickRed}{=}\ $\infty$ \\
\mbox{} \\
\mbox{}\ \ \ \ \ \ \ \ posibles\textcolor{BrickRed}{.}push\textcolor{BrickRed}{(}\textcolor{Red}{\{} \\
\mbox{}\ \ \ \ \ \ \ \ \ \ \ \ historial\textcolor{BrickRed}{:}\ \textcolor{BrickRed}{[]} \\
\mbox{}\ \ \ \ \ \ \ \ \ \ \ \ ejecutando\textcolor{BrickRed}{:}\ \textcolor{BrickRed}{[]} \\
\mbox{}\ \ \ \ \ \ \ \ \ \ \ \ restantes\textcolor{BrickRed}{:}\ tareas \\
\mbox{}\ \ \ \ \ \ \ \ \textcolor{Red}{\}}\textcolor{BrickRed}{)} \\
\mbox{} \\
\mbox{}\ \ \ \ \ \ \ \ \textbf{\textcolor{Blue}{while}}\ \textcolor{BrickRed}{(}\textbf{\textcolor{Blue}{not}}\ posibles\textcolor{BrickRed}{.}empty?\textcolor{BrickRed}{)} \\
\mbox{}\ \ \ \ \ \ \ \ \ \ \ \ actual\ \textcolor{BrickRed}{=}\ posibles\textcolor{BrickRed}{.}pop \\
\mbox{} \\
\mbox{}\ \ \ \ \ \ \ \ \ \ \ \ \textbf{\textcolor{Blue}{if}}\ \textcolor{BrickRed}{(}\#actual\textcolor{BrickRed}{.}historial\ \textcolor{BrickRed}{==}\ \#tareas\ \textcolor{BrickRed}{\&\&}\ actual\textcolor{BrickRed}{.}ejecutando\textcolor{BrickRed}{.}empty?\textcolor{BrickRed}{)} \\
\mbox{}\ \ \ \ \ \ \ \ \ \ \ \ \ \ \ \ solucion\ \textcolor{BrickRed}{=}\ actual\ \textbf{\textcolor{Blue}{if}}\ t$\_$ejecucion\textcolor{BrickRed}{(}actual\textcolor{BrickRed}{)}\ \textcolor{BrickRed}{\textless{}}\ t$\_$ejecucion\textcolor{BrickRed}{(}solucion\textcolor{BrickRed}{)} \\
\mbox{}\ \ \ \ \ \ \ \ \ \ \ \ \textbf{\textcolor{Blue}{else}} \\
\mbox{}\ \ \ \ \ \ \ \ \ \ \ \ \ \ \ \ libre\ \textcolor{BrickRed}{=}\ get$\_$core\textcolor{BrickRed}{(}actual\textcolor{BrickRed}{.}ejecutando\textcolor{BrickRed}{)} \\
\mbox{} \\
\mbox{}\ \ \ \ \ \ \ \ \ \ \ \ \ \ \ \ \textbf{\textcolor{Blue}{if}}\ \textcolor{BrickRed}{(}libre\ \textcolor{BrickRed}{\textgreater{}}\ \textcolor{BrickRed}{-}\textcolor{Purple}{1}\textcolor{BrickRed}{)} \\
\mbox{}\ \ \ \ \ \ \ \ \ \ \ \ \ \ \ \ \ \ \ \ \textbf{\textcolor{Blue}{for}}\ \textcolor{BrickRed}{(}nueva\ \textbf{\textcolor{Blue}{in}}\ actual\textcolor{BrickRed}{.}restantes\textcolor{BrickRed}{)} \\
\mbox{}\ \ \ \ \ \ \ \ \ \ \ \ \ \ \ \ \ \ \ \ \ \ \ \ \textbf{\textcolor{Blue}{if}}\ \textcolor{BrickRed}{(}\textbf{\textcolor{Blue}{not}}\ tiene$\_$dependencias\textcolor{BrickRed}{(}nueva\textcolor{BrickRed}{,}\ actual\textcolor{BrickRed}{))} \\
\mbox{}\ \ \ \ \ \ \ \ \ \ \ \ \ \ \ \ \ \ \ \ \ \ \ \ \ \ \ \ copia\ \textcolor{BrickRed}{=}\ actual \\
\mbox{}\ \ \ \ \ \ \ \ \ \ \ \ \ \ \ \ \ \ \ \ \ \ \ \ \ \ \ \ copia\textcolor{BrickRed}{.}ejecutando\textcolor{BrickRed}{[}libre\textcolor{BrickRed}{]}\ \textcolor{BrickRed}{=}\ nueva \\
\mbox{}\ \ \ \ \ \ \ \ \ \ \ \ \ \ \ \ \ \ \ \ \ \ \ \ \ \ \ \ copia\textcolor{BrickRed}{.}restantes\textcolor{BrickRed}{.}delete\textcolor{BrickRed}{(}j\textcolor{BrickRed}{)} \\
\mbox{}\ \ \ \ \ \ \ \ \ \ \ \ \ \ \ \ \ \ \ \ \ \ \ \ \ \ \ \ copia\textcolor{BrickRed}{.}historial\textcolor{BrickRed}{.}push\textcolor{BrickRed}{(}\textcolor{Red}{\{} \\
\mbox{}\ \ \ \ \ \ \ \ \ \ \ \ \ \ \ \ \ \ \ \ \ \ \ \ \ \ \ \ \ \ \ \ tarea\textcolor{BrickRed}{:}\ nueva \\
\mbox{}\ \ \ \ \ \ \ \ \ \ \ \ \ \ \ \ \ \ \ \ \ \ \ \ \ \ \ \ \ \ \ \ t$\_$inicio\textcolor{BrickRed}{:}\ t$\_$ejecucion\textcolor{BrickRed}{(}copia\textcolor{BrickRed}{)} \\
\mbox{}\ \ \ \ \ \ \ \ \ \ \ \ \ \ \ \ \ \ \ \ \ \ \ \ \ \ \ \ \ \ \ \ core\textcolor{BrickRed}{:}\ libre \\
\mbox{}\ \ \ \ \ \ \ \ \ \ \ \ \ \ \ \ \ \ \ \ \ \ \ \ \ \ \ \ \textcolor{Red}{\}}\textcolor{BrickRed}{)} \\
\mbox{} \\
\mbox{}\ \ \ \ \ \ \ \ \ \ \ \ \ \ \ \ \ \ \ \ \ \ \ \ \ \ \ \ posibles\textcolor{BrickRed}{.}push\textcolor{BrickRed}{(}copia\textcolor{BrickRed}{)} \\
\mbox{} \\
\mbox{}\ \ \ \ \ \ \ \ \ \ \ \ \ \ \ \ \textbf{\textcolor{Blue}{if}}\ \textcolor{BrickRed}{(}\textbf{\textcolor{Blue}{not}}\ actual\textcolor{BrickRed}{.}ejecutando\textcolor{BrickRed}{.}empty?\textcolor{BrickRed}{)} \\
\mbox{}\ \ \ \ \ \ \ \ \ \ \ \ \ \ \ \ \ \ \ \ min$\_$restante\ \textcolor{BrickRed}{=}\ $min\left\{tarea.tiempo: tarea \in actual.ejecutando\right\}$ \\
\mbox{} \\
\mbox{}\ \ \ \ \ \ \ \ \ \ \ \ \ \ \ \ \ \ \ \ \textbf{\textcolor{Blue}{for}}\ \textcolor{BrickRed}{(}tarea\ \textbf{\textcolor{Blue}{in}}\ actual\textcolor{BrickRed}{.}restantes\textcolor{BrickRed}{)} \\
\mbox{}\ \ \ \ \ \ \ \ \ \ \ \ \ \ \ \ \ \ \ \ \ \ \ \ tarea\textcolor{BrickRed}{.}tiempo\ \textcolor{BrickRed}{-=}\ min$\_$restante \\
\mbox{} \\
\mbox{}\ \ \ \ \ \ \ \ \ \ \ \ \ \ \ \ \ \ \ \ posibles\textcolor{BrickRed}{.}push\textcolor{BrickRed}{(}actual\textcolor{BrickRed}{)} \\
\mbox{} \\
\mbox{}\ \ \ \ \ \ \ \ \textbf{\textcolor{Blue}{return}}\ solucion \\
\mbox{}\ \ \ \ \textbf{\textcolor{Blue}{end}} \\
\mbox{} \\
\mbox{}\ \ \ \ \textbf{\textcolor{Blue}{def}}\ t$\_$ejecucion\textcolor{BrickRed}{(}planificacion\textcolor{BrickRed}{)} \\
\mbox{}\ \ \ \ \ \ \ \ \textbf{\textcolor{Blue}{return}}\ $max\left\{ planificacion[i].t\_inicio : i \in \{1, \dots \#planificacion \} \right\}$ \\
\mbox{}\ \ \ \ \textbf{\textcolor{Blue}{end}} \\
\mbox{} \\
\mbox{}\ \ \ \ \textbf{\textcolor{Blue}{def}}\ get$\_$core\textcolor{BrickRed}{(}multiprocesador\textcolor{BrickRed}{)} \\
\mbox{}\ \ \ \ \ \ \ \ \textbf{\textcolor{Blue}{return}}\ $min\left\{i < ncores : multiprocesador[i].empty? \right\}$ \\
\mbox{}\ \ \ \ \textbf{\textcolor{Blue}{end}} \\
\mbox{} \\
\mbox{}\ \ \ \ \textbf{\textcolor{Blue}{def}}\ tiene$\_$dependencias\textcolor{BrickRed}{(}tarea\textcolor{BrickRed}{,}\ plan\textcolor{BrickRed}{)} \\
\mbox{}\ \ \ \ \ \ \ \ \textbf{\textcolor{Blue}{return}}\ $\exists i : tarea \in plan.ejecutando[i].dependencias$ \textbf{\textcolor{Blue}{or}} \\
\mbox{}\ \ \ \ \ \ \ \ \ \ \ \ $\exists i : tarea \in plan.restantes[i].dependencias$ \\
\mbox{}\ \ \ \ \textbf{\textcolor{Blue}{end}} \\
\mbox{} \\
\mbox{}\ \ \ \ \textbf{\textcolor{Blue}{return}}\ resolver\textcolor{BrickRed}{()} \\
\mbox{}\textbf{\textcolor{Blue}{end}} \\
\mbox{}
}
      \normalsize
    
    \subsection{Branch \& Bound}
      Para el algoritmo siguiendo la estrategia Branch \& Bound, utilizamos la misma representación para la 
      solución. De nuevo no podemos encontrar el tamaño del espacio de soluciones. 
      
      Los hijos posibles de un nodo intermedio $(a_1, \dots, a_m)$ (con $a_i$ asignaciones válidas) son del tipo 
      $(a_1, \dots, a_m, a_{m+1})$, donde $a_{m+1}$ es una asignación asociada a una tarea que no está ya en 
      $\{a_1, \dots, a_m\}$.
      
      La cota que se utiliza para minorar el tiempo de ejecución de una posible solución es el tiempo de ejecución 
      que lleva hasta el momento. Si en un nodo intermedio, este valor es mayor que la mejor solución hasta el 
      momento, es seguro que no encontraremos una solución óptima al desarrollar el nodo.
      
      La estrategia de ramificación es LC-FIFO, usando una cola con prioridad para obtener las posibles 
      planificaciones de menor a mayor tiempo de ejecución.
    
      \subsubsection{Algoritmo}
      
        \small
        \texttt{% Generator: GNU source-highlight, by Lorenzo Bettini, http://www.gnu.org/software/src-highlite
\noindent
\mbox{}\textbf{\textcolor{Blue}{def}}\ planifica\textcolor{BrickRed}{(}ncores\textcolor{BrickRed}{,}\ tareas\textcolor{BrickRed}{)} \\
\mbox{}\ \ \ \ \textbf{\textcolor{Blue}{def}}\ resolver\textcolor{BrickRed}{()} \\
\mbox{}\ \ \ \ \ \ \ \ posibles\ \textcolor{BrickRed}{=}\ new\ Priority Queue \\
\mbox{}\ \ \ \ \ \ \ \ solución\ \textcolor{BrickRed}{=}\ \textcolor{BrickRed}{[]} \\
\mbox{}\ \ \ \ \ \ \ \ tiempo\ \textcolor{BrickRed}{=}\ $\infty$ \\
\mbox{} \\
\mbox{}\ \ \ \ \ \ \ \ posibles\textcolor{BrickRed}{.}push\textcolor{BrickRed}{(}\textcolor{Red}{\{} \\
\mbox{}\ \ \ \ \ \ \ \ \ \ \ \ historial\textcolor{BrickRed}{:}\ \textcolor{BrickRed}{[]} \\
\mbox{}\ \ \ \ \ \ \ \ \ \ \ \ ejecutando\textcolor{BrickRed}{:}\ \textcolor{BrickRed}{[]} \\
\mbox{}\ \ \ \ \ \ \ \ \ \ \ \ restantes\textcolor{BrickRed}{:}\ tareas \\
\mbox{}\ \ \ \ \ \ \ \ \textcolor{Red}{\}}\textcolor{BrickRed}{)} \\
\mbox{} \\
\mbox{}\ \ \ \ \ \ \ \ \textbf{\textcolor{Blue}{while}}\ \textcolor{BrickRed}{(}\textbf{\textcolor{Blue}{not}}\ posibles\textcolor{BrickRed}{.}empty?\textcolor{BrickRed}{)} \\
\mbox{}\ \ \ \ \ \ \ \ \ \ \ \ actual\ \textcolor{BrickRed}{=}\ posibles\textcolor{BrickRed}{.}pop \\
\mbox{} \\
\mbox{}\ \ \ \ \ \ \ \ \ \ \ \ \textbf{\textcolor{Blue}{if}}\ \textcolor{BrickRed}{(}\#actual\textcolor{BrickRed}{.}historial\ \textcolor{BrickRed}{==}\ \#tareas\ \textcolor{BrickRed}{\&\&}\ actual\textcolor{BrickRed}{.}ejecutando\textcolor{BrickRed}{.}empty?\textcolor{BrickRed}{)} \\
\mbox{}\ \ \ \ \ \ \ \ \ \ \ \ \ \ \ \ solucion\ \textcolor{BrickRed}{=}\ actual\ \textbf{\textcolor{Blue}{if}}\ t$\_$ejecucion\textcolor{BrickRed}{(}actual\textcolor{BrickRed}{)}\ \textcolor{BrickRed}{\textless{}}\ t$\_$ejecucion\textcolor{BrickRed}{(}solucion\textcolor{BrickRed}{)} \\
\mbox{}\ \ \ \ \ \ \ \ \ \ \ \ \textbf{\textcolor{Blue}{else}} \\
\mbox{}\ \ \ \ \ \ \ \ \ \ \ \ \ \ \ \ libre\ \textcolor{BrickRed}{=}\ get$\_$core\textcolor{BrickRed}{(}actual\textcolor{BrickRed}{.}ejecutando\textcolor{BrickRed}{)} \\
\mbox{} \\
\mbox{}\ \ \ \ \ \ \ \ \ \ \ \ \ \ \ \ \textbf{\textcolor{Blue}{if}}\ \textcolor{BrickRed}{(}libre\ \textcolor{BrickRed}{\textgreater{}}\ \textcolor{BrickRed}{-}\textcolor{Purple}{1}\textcolor{BrickRed}{)} \\
\mbox{}\ \ \ \ \ \ \ \ \ \ \ \ \ \ \ \ \ \ \ \ \textbf{\textcolor{Blue}{for}}\ \textcolor{BrickRed}{(}nueva\ \textbf{\textcolor{Blue}{in}}\ actual\textcolor{BrickRed}{.}restantes\textcolor{BrickRed}{)} \\
\mbox{}\ \ \ \ \ \ \ \ \ \ \ \ \ \ \ \ \ \ \ \ \ \ \ \ \textbf{\textcolor{Blue}{if}}\ \textcolor{BrickRed}{(}\textbf{\textcolor{Blue}{not}}\ tiene$\_$dependencias\textcolor{BrickRed}{(}nueva\textcolor{BrickRed}{,}\ actual\textcolor{BrickRed}{))} \\
\mbox{}\ \ \ \ \ \ \ \ \ \ \ \ \ \ \ \ \ \ \ \ \ \ \ \ \ \ \ \ copia\ \textcolor{BrickRed}{=}\ actual \\
\mbox{}\ \ \ \ \ \ \ \ \ \ \ \ \ \ \ \ \ \ \ \ \ \ \ \ \ \ \ \ copia\textcolor{BrickRed}{.}ejecutando\textcolor{BrickRed}{[}libre\textcolor{BrickRed}{]}\ \textcolor{BrickRed}{=}\ nueva \\
\mbox{}\ \ \ \ \ \ \ \ \ \ \ \ \ \ \ \ \ \ \ \ \ \ \ \ \ \ \ \ copia\textcolor{BrickRed}{.}restantes\textcolor{BrickRed}{.}delete\textcolor{BrickRed}{(}j\textcolor{BrickRed}{)} \\
\mbox{}\ \ \ \ \ \ \ \ \ \ \ \ \ \ \ \ \ \ \ \ \ \ \ \ \ \ \ \ copia\textcolor{BrickRed}{.}historial\textcolor{BrickRed}{.}push\textcolor{BrickRed}{(}\textcolor{Red}{\{} \\
\mbox{}\ \ \ \ \ \ \ \ \ \ \ \ \ \ \ \ \ \ \ \ \ \ \ \ \ \ \ \ \ \ \ \ tarea\textcolor{BrickRed}{:}\ nueva \\
\mbox{}\ \ \ \ \ \ \ \ \ \ \ \ \ \ \ \ \ \ \ \ \ \ \ \ \ \ \ \ \ \ \ \ t$\_$inicio\textcolor{BrickRed}{:}\ t$\_$ejecucion\textcolor{BrickRed}{(}copia\textcolor{BrickRed}{)} \\
\mbox{}\ \ \ \ \ \ \ \ \ \ \ \ \ \ \ \ \ \ \ \ \ \ \ \ \ \ \ \ \ \ \ \ core\textcolor{BrickRed}{:}\ libre \\
\mbox{}\ \ \ \ \ \ \ \ \ \ \ \ \ \ \ \ \ \ \ \ \ \ \ \ \ \ \ \ \textcolor{Red}{\}}\textcolor{BrickRed}{)} \\
\mbox{} \\
\mbox{}\ \ \ \ \ \ \ \ \ \ \ \ \ \ \ \ \ \ \ \ \ \ \ \ \ \ \ \ posibles\textcolor{BrickRed}{.}push\textcolor{BrickRed}{(}copia\textcolor{BrickRed}{)} \\
\mbox{} \\
\mbox{}\ \ \ \ \ \ \ \ \ \ \ \ \ \ \ \ \textbf{\textcolor{Blue}{if}}\ \textcolor{BrickRed}{(}\textbf{\textcolor{Blue}{not}}\ actual\textcolor{BrickRed}{.}ejecutando\textcolor{BrickRed}{.}empty?\textcolor{BrickRed}{)} \\
\mbox{}\ \ \ \ \ \ \ \ \ \ \ \ \ \ \ \ \ \ \ \ min$\_$restante\ \textcolor{BrickRed}{=}\ $min\left\{tarea.tiempo: tarea \in actual.ejecutando\right\}$ \\
\mbox{} \\
\mbox{}\ \ \ \ \ \ \ \ \ \ \ \ \ \ \ \ \ \ \ \ \textbf{\textcolor{Blue}{for}}\ \textcolor{BrickRed}{(}tarea\ \textbf{\textcolor{Blue}{in}}\ actual\textcolor{BrickRed}{.}restantes\textcolor{BrickRed}{)} \\
\mbox{}\ \ \ \ \ \ \ \ \ \ \ \ \ \ \ \ \ \ \ \ \ \ \ \ tarea\textcolor{BrickRed}{.}tiempo\ \textcolor{BrickRed}{-=}\ min$\_$restante \\
\mbox{} \\
\mbox{}\ \ \ \ \ \ \ \ \ \ \ \ \ \ \ \ \ \ \ \ \textbf{\textcolor{Blue}{if}}\ \textcolor{BrickRed}{(}t$\_$ejecucion\textcolor{BrickRed}{(}actual\textcolor{BrickRed}{)}\ \textcolor{BrickRed}{\textless{}}\ t$\_$ejecucion\textcolor{BrickRed}{(}solucion\textcolor{BrickRed}{)}\textcolor{BrickRed}{)} \\
\mbox{}\ \ \ \ \ \ \ \ \ \ \ \ \ \ \ \ \ \ \ \ \ \ \ \ 
posibles\textcolor{BrickRed}{.}push\textcolor{BrickRed}{(}actual\textcolor{BrickRed}{)} \\
\mbox{} \\
\mbox{}\ \ \ \ \ \ \ \ \textbf{\textcolor{Blue}{return}}\ solucion \\
\mbox{}\ \ \ \ \textbf{\textcolor{Blue}{end}} \\
\mbox{} \\
\mbox{}\ \ \ \ \textbf{\textcolor{Blue}{def}}\ t$\_$ejecucion\textcolor{BrickRed}{(}planificacion\textcolor{BrickRed}{)} \\
\mbox{}\ \ \ \ \ \ \ \ \textbf{\textcolor{Blue}{return}}\ $max\left\{ planificacion[i].t\_inicio : i \in \{0, \dots \#planificacion -1\} \right\}$ \\
\mbox{}\ \ \ \ \textbf{\textcolor{Blue}{end}} \\
\mbox{} \\
\mbox{}\ \ \ \ \textbf{\textcolor{Blue}{def}}\ get$\_$core\textcolor{BrickRed}{(}multiprocesador\textcolor{BrickRed}{)} \\
\mbox{}\ \ \ \ \ \ \ \ \textbf{\textcolor{Blue}{return}}\ $min\left\{i < ncores : multiprocesador[i].empty? \right\}$ \\
\mbox{}\ \ \ \ \textbf{\textcolor{Blue}{end}} \\
\mbox{} \\
\mbox{}\ \ \ \ \textbf{\textcolor{Blue}{def}}\ tiene$\_$dependencias\textcolor{BrickRed}{(}tarea\textcolor{BrickRed}{,}\ plan\textcolor{BrickRed}{)} \\
\mbox{}\ \ \ \ \ \ \ \ \textbf{\textcolor{Blue}{return}}\ $\exists i : tarea \in plan.ejecutando[i].dependencias$ \textbf{\textcolor{Blue}{or}} \\
\mbox{}\ \ \ \ \ \ \ \ \ \ \ \ $\exists i : tarea \in plan.restantes[i].dependencias$ \\
\mbox{}\ \ \ \ \textbf{\textcolor{Blue}{end}} \\
\mbox{} \\
\mbox{}\ \ \ \ \textbf{\textcolor{Blue}{return}}\ resolver\textcolor{BrickRed}{()} \\
\mbox{}\textbf{\textcolor{Blue}{end}} \\
\mbox{}
}
        \normalsize
     
    \subsection{Implementación en un Sistema Operativo}
    Al analizar la viabilidad de implementar el algoritmo de planificación como parte de un Sistema Operativo, hemos concluido que no es viable, ya que no conocemos de antemano los tiempos de ejecución de los programas, lo que nos impide planificar la tareas conforme a nuestro algoritmo. De igual forma, la misma planificación de las tareas podría suponer un coste en tiempo muy significativo, si el número de tareas es muy grande, lo que ralentizaría mucho el Sistema Operativo en general. 
      
    \section{3-Dimensional Matching}
    \subsection{Enunciado}
    Sean $X$,$Y$,$Z$ tres conjuntos finitos y sea $T \subset X \times Y \times Z$, subconjunto de tripletas válidas.
    Una asignación válida $M$ es un subconjunto de elementos disjuntos de $T$. Es decir, $M \subset T$ tal que:
    \begin{equation}
        \forall (x_1,y_1,z_1), (x_2,y_2,z_2) \in M : \quad (x_1 \neq x_2) \wedge (y_1 \neq y_2) \wedge (z_1 \neq z_2)
        \end{equation} 
        Buscamos la asignación válida de mayor cardinal, es decir la $M$ asignación válida maximizando la satisfacción.
        
        \subsection{Backtracking}
	  En este algoritmo consideraremos todas las asignaciones posibles y nos quedaremos con la asignación que tenga mayor satisfacción
	  que, en este caso, coincide con el cardinal.
	  
	  Para ello, consideraremos todas las aristas e iremos comprobando para cada una si es posible añadirla. De ser así, distinguiremos dos
	  casos: La arista puede pertenecer o no a la solución. Sobre estos dos casos aplicaremos el mismo algoritmo. Cuando no quede ninguna
	  arista que pueda ser añadida, compararemos los tamaños de la solución obtenida y la mejor solución hasta el momento, actualizando
	  la mejor solución si la actual la mejorase. 
	  
	  Representaremos la solución como un vector de booleanos, que nos indican si la arista asociada al índice se encuentra o 
	  no en la solución. Como cada arista tiene la opción de estar o no estar en la solución, la dimensión del espacio de 
	  soluciones es de $2^n$, siendo $n$ el número de aristas.
	  
	  \subsubsection{Algoritmo}
	    \small
	    \texttt{% Generator: GNU source-highlight, by Lorenzo Bettini, http://www.gnu.org/software/src-highlite
\noindent
\mbox{}\textbf{\textcolor{Blue}{def}}\ 3dmatching\textcolor{BrickRed}{(}aristas\textcolor{BrickRed}{)} \\
\mbox{}\ \ \ \ particiones\ \textcolor{BrickRed}{=}\ new\ Stack \\
\mbox{}\ \ \ \ particiones\textcolor{BrickRed}{.}push\textcolor{BrickRed}{([])} \\
\mbox{} \\
\mbox{}\ \ \ \ \textbf{\textcolor{Blue}{while}}\ \textbf{\textcolor{Blue}{not}}\ particiones\textcolor{BrickRed}{.}empty? \\
\mbox{}\ \ \ \ \ \ \ \ actual\ \textcolor{BrickRed}{=}\ particiones\textcolor{BrickRed}{.}pop \\
\mbox{}\ \ \ \ \ \ \ \ \textbf{\textcolor{Blue}{if}}\ $\#$actual\ \textcolor{BrickRed}{==}\ $\#$aristas \\
\mbox{}\ \ \ \ \ \ \ \ \ \ \ \ \textbf{\textcolor{Blue}{if}}\ valor\textcolor{BrickRed}{(}actual\textcolor{BrickRed}{)}\ \textcolor{BrickRed}{\textgreater{}}\ valor\textcolor{BrickRed}{(}solucion\textcolor{BrickRed}{)} \\
\mbox{}\ \ \ \ \ \ \ \ \ \ \ \ \ \ \ \ solucion\ \textcolor{BrickRed}{=}\ actual \\
\mbox{}\ \ \ \ \ \ \ \ \textbf{\textcolor{Blue}{else}} \\
\mbox{}\ \ \ \ \ \ \ \ \ \ \ \ \textbf{\textcolor{Blue}{if}}\ interseca?\textcolor{BrickRed}{(}$\mathtt{aristas[\#actual]}$\textcolor{BrickRed}{,}\ actual\textcolor{BrickRed}{)} \\
\mbox{}\ \ \ \ \ \ \ \ \ \ \ \ \ \ \ \ particiones\textcolor{BrickRed}{.}push\textcolor{BrickRed}{(}actual\ \textcolor{BrickRed}{++}\ \textcolor{BrickRed}{[}\textbf{\textcolor{Blue}{true}}\textcolor{BrickRed}{])} \\
\mbox{}\ \ \ \ \ \ \ \ \ \ \ \ particiones\textcolor{BrickRed}{.}push\textcolor{BrickRed}{(}actual\ \textcolor{BrickRed}{++}\ \textcolor{BrickRed}{[}\textbf{\textcolor{Blue}{false}}\textcolor{BrickRed}{])} \\
\mbox{} \\
\mbox{}\ \ \ \ \textbf{\textcolor{Blue}{return}}\ solucion \\
\mbox{}\textbf{\textcolor{Blue}{end}} \\
\mbox{} \\
\mbox{}\textbf{\textcolor{Blue}{def}}\ interseca?\textcolor{BrickRed}{(}particion\textcolor{BrickRed}{,}\ arista\textcolor{BrickRed}{)} \\
\mbox{}\ \ \ \ \textbf{\textcolor{Blue}{return}}\ $\mathtt{!(arista.x \in particion.usadosX)}$ \\ %$\bigcup_{\mathtt{particion[i]}} \left( \mathtt{aristas[i]} \cap \mathtt{arista} \right) == \empty$ \\
\mbox{}\ \ \ \ \ \ \ \textbf{\textcolor{Blue}{and}}\ $\mathtt{!(arista.y \in particion.usadosY)}$ \\
\mbox{}\ \ \ \ \ \ \ \textbf{\textcolor{Blue}{and}}\ $\mathtt{!(arista.z \in particion.usadosZ)}$ \\
\mbox{}\textbf{\textcolor{Blue}{end}} \\
\mbox{} \\
\mbox{}\textbf{\textcolor{Blue}{def}}\ valor\textcolor{BrickRed}{(}particion\textcolor{BrickRed}{)} \\
\mbox{}\ \ \ \ \textbf{\textcolor{Blue}{return}}\ $\mathtt{\sum_{\mathtt{i=1}}^{\mathtt{\#particion}} \mathtt{particion[i]}} $ \\
\mbox{}\textbf{\textcolor{Blue}{end}} \\
\mbox{}
}
	    \normalsize
	  
        \subsection{Branch \& Bound}
	  Para la versión Branch \& Bound del algoritmo, usaremos la misma representación para la solución, 
	  un vector de booleanos, por lo que tendremos el mismo espacio de soluciones.
	  
	  Los hijos posibles de un nodo intermedio $(a_1, \dots, a_m)$ (con $a_i$ asignaciones válidas) son del tipo 
	  $(a_1, \dots, a_m, a_{m+1})$, donde $a_{m+1}$ es una asignación válida que no está ya en $\{a_1, \dots, a_m\}$.
	  
	  Como función de poda sobre una solución parcial, consideraremos la satisfacción que obtendríamos con todas 
	  las aristas restantes como cota máxima de la satisfacción posible a obtener. Si la satisfacción máxima de 
	  la parte restante sumada con la satisfacción de la solución parcial es menor que la satisfacción de la mejor 
	  solución hasta el momento, podemos descartar la solución parcial y obviar todas las ramas derivadas. 
	  
	  En este caso, la satisfacción estará relacionada con la preferencia de cada arista. Dicha preferencia vendrá dada en un 
	  nuevo vector de valores reales, que utilizaremos para calcular el valor de cada arista en base a su preferencia. 
      
      Al tratarse de un problema de maximización, la estrategia de poda que se sigue es podar cada rama en la que se
      encuentre una satisfacción máxima (cota superior) menor que la satisfacción de la mejor solución encontrada hasta el
      momento. La estrategia de ramificación utilizada es Maximum-Benefit FIFO, ya que usamos una cola con prioridad
      ordenada por el mayor beneficio.
    
        \subsubsection{Algoritmo}    
        \small
        \texttt{% Generator: GNU source-highlight, by Lorenzo Bettini, http://www.gnu.org/software/src-highlite
\noindent
\mbox{}\textbf{\textcolor{Blue}{def}}\ 3dmatching\textcolor{BrickRed}{(}aristas\textcolor{BrickRed}{)} \\
\mbox{}\ \ \ \ particiones\ \textcolor{BrickRed}{=}\ new\ Priority\ Queue \\
\mbox{}\ \ \ \ particiones\textcolor{BrickRed}{.}push\textcolor{BrickRed}{([])} \\
\mbox{} \\
\mbox{}\ \ \ \ \textbf{\textcolor{Blue}{while}}\ \textbf{\textcolor{Blue}{not}}\ particiones\textcolor{BrickRed}{.}empty? \\
\mbox{}\ \ \ \ \ \ \ \ actual\ \textcolor{BrickRed}{=}\ particiones\textcolor{BrickRed}{.}pop \\
\mbox{}\ \ \ \ \ \ \ \ \textbf{\textcolor{Blue}{if}}\ \#actual\ \textcolor{BrickRed}{==}\ \#aristas \\
\mbox{}\ \ \ \ \ \ \ \ \ \ \ \ \textbf{\textcolor{Blue}{if}}\ valor\textcolor{BrickRed}{(}actual\textcolor{BrickRed}{)}\ \textcolor{BrickRed}{\textgreater{}}\ valor\textcolor{BrickRed}{(}solucion\textcolor{BrickRed}{)} \\
\mbox{}\ \ \ \ \ \ \ \ \ \ \ \ \ \ \ \ solucion\ \textcolor{BrickRed}{=}\ actual \\
\mbox{}\ \ \ \ \ \ \ \ \textbf{\textcolor{Blue}{else}} \\
\mbox{}\ \ \ \ \ \ \ \ \ \ \ \ \textbf{\textcolor{Blue}{if}}\ interseca?\textcolor{BrickRed}{(}$\mathtt{aristas[\#actual]}$\textcolor{BrickRed}{,}\ actual\textcolor{BrickRed}{)} \\
\mbox{}\ \ \ \ \ \ \ \ \ \ \ \ \ \ \ \ particiones\textcolor{BrickRed}{.}push\textcolor{BrickRed}{(}actual\ \textcolor{BrickRed}{++}\ \textcolor{BrickRed}{[}\textbf{\textcolor{Blue}{true}}\textcolor{BrickRed}{])} \\
\mbox{}\ \ \ \ \ \ \ \ \ \ \ \ \textbf{\textcolor{Blue}{if}}\ \textcolor{BrickRed}{(}valor\textcolor{BrickRed}{(}solucion\textcolor{BrickRed}{)}\ \textcolor{BrickRed}{\textless{}}\ valor\textcolor{BrickRed}{(}actual\textcolor{BrickRed}{)}\ \textcolor{BrickRed}{+}\ valorMaximo\textcolor{BrickRed}{(}actual\textcolor{BrickRed}{))} \\
\mbox{}\ \ \ \ \ \ \ \ \ \ \ \ \ \ \ \ particiones\textcolor{BrickRed}{.}push\textcolor{BrickRed}{(}actual\ \textcolor{BrickRed}{++}\ \textcolor{BrickRed}{[}\textbf{\textcolor{Blue}{false}}\textcolor{BrickRed}{])} \\
\mbox{} \\
\mbox{}\ \ \ \ \textbf{\textcolor{Blue}{return}}\ solucion \\
\mbox{}\textbf{\textcolor{Blue}{end}} \\
\mbox{} \\
\mbox{}\textbf{\textcolor{Blue}{def}}\ interseca?\textcolor{BrickRed}{(}particion\textcolor{BrickRed}{,}\ arista\textcolor{BrickRed}{)} \\
\mbox{}\ \ \ \ \textbf{\textcolor{Blue}{return}}\ $\mathtt{!(arista.x \in particion.usadosX)}$ \\
\mbox{}\ \ \ \ \ \ \ \textbf{\textcolor{Blue}{and}}\ $\mathtt{!(arista.y \in particion.usadosY)}$ \\
\mbox{}\ \ \ \ \ \ \ \textbf{\textcolor{Blue}{and}}\ $\mathtt{!(arista.z \in particion.usadosZ)}$ \\
\mbox{}\textbf{\textcolor{Blue}{end}} \\
\mbox{} \\
\mbox{} \\
\mbox{}\textbf{\textcolor{Blue}{def}}\ valor\textcolor{BrickRed}{(}particion\textcolor{BrickRed}{)} \\
\mbox{}\ \ \ \ \textbf{\textcolor{Blue}{return}}\ $\mathtt{\sum_{\mathtt{i=1}}^{\mathtt{\#particion}} \mathtt{particion[i]} * \mathtt{preferencias[i]}} $ \\
\mbox{}\textbf{\textcolor{Blue}{end}} \\
\mbox{} \\
\mbox{}\textbf{\textcolor{Blue}{def}}\ valorMaximo\textcolor{BrickRed}{(}particion\textcolor{BrickRed}{,} aristas\textcolor{BrickRed}{)} \\
\mbox{}\ \ \ \ \ \textbf{valor \textcolor{Red}{=} $\mathtt{0}$}\\
\mbox{}\ \ \ \ \ \textbf{\textcolor{Blue}{for}\textcolor{Red}{(}\textcolor{Blue}{int} i \textcolor{Red}{=} $\mathtt{0}$; 
$\mathtt{i \le \#aristas}$; $\mathtt{i}$++\textcolor{Red}{)}}\\
\mbox{}\ \ \ \ \ \ \ \ \textbf{\textcolor{Blue}{if}}\ \textcolor{Red}{(}$\mathtt{aristas[i] \notin particion}$\textcolor{Red}{)} \\
\mbox{}\ \ \ \ \ \ \ \ \ \ \ \textbf{valor \textcolor{Red}{+=} $\mathtt{preferencias[i]}$}\\
\mbox{}\ \ \ \ \ \textbf{\textcolor{Blue}{return} valor}\\
\mbox{}\textbf{\textcolor{Blue}{end}} \\
\mbox{}
}
        \normalsize        
    \end{document}
